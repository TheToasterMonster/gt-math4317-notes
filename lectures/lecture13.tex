\chapter{Oct.~3 -- Perfect Sets}

\begin{definition}
  A set $P \subseteq \R$ is \textbf{perfect} if it is closed
  and contains no isolated points.
\end{definition}

\begin{example}
  The closed intervals $[a, b]$ are perfect.
\end{example}

\section{The Cantor Set}
\begin{example}
  Define a sequence of sets inductively by $C_0 = [0, 1]$ and
  removing its middle third to get
  $C_1 = C_0 \setminus (\frac{1}{3}, \frac{2}{3})$, i.e.
  \[C_1 = \left[0, \frac{1}{3}\right] \cup \left[\frac{2}{3}, 1\right].\]
  Then we have
  \[C_2 = \left(\left[0, \frac{1}{9}\right] \cup \left[\frac{2}{9}, \frac{1}{3}\right]\right) \cup \left(\left[\frac{2}{3}, \frac{7}{9}\right] \cup \left[\frac{8}{9}, 1\right]\right),\]
  and so on, removing the middle third of each interval at
  each step. Note that $C_n$ is a set consisting of $2^n$
  closed intervals each of length $\frac{1}{3^n}$. Then
  the (middle third) \textit{Cantor set} $C$ is defined as
  \[C = \bigcap_{n = 0}^\infty C_n.\]
\end{example}

\begin{remark}
If we consider the sum of the lengths of the intervals that we
removed, we get
\[\frac{1}{3} + 2 \cdot \frac{1}{9} + 4 \cdot \frac{1}{27} + \dots + 2^{n - 1} \cdot \frac{1}{3^n} + \dots = \frac{1}{3}\left(\frac{1}{1 - \frac{2}{3}}\right) = \frac{1}{3} \cdot 3 = 1.\]
So, in some sense, the ``size''
\footnote{The (Lebesgue) \textit{measure}.}
of the Cantor set $C$ is 0.
However, $C$ is uncountable. In particular, the cardinality of
$C$ is the same as the cardinality of $\R$. A lot of
counterexamples in real analysis come from this Cantor set.
\end{remark}

\begin{remark}
  This means that the usual measure is not a good way to
  ``catch'' the Cantor set. Instead, we can consider
  \textit{fractional} (or \textit{fractal}) dimensions.
\end{remark}

\begin{theorem}
  The Cantor set $C$ is perfect.
\end{theorem}

\begin{proof}
  First note that $C$ is a countable intersection of
  closed sets, so $C$ is closed as well.
  To see that $C$ has no isolated points, take an arbitrary
  $x \in C$. Since $x \in C_n$ for all $n$, we can find
  $x_n \in C_n$ such that $x_n \ne x$ and
  $|x_n - x| < \frac{1}{3^n}$. Then $\lim_{n \to \infty} x_n = x$
  and $x_n \ne x$, so $x$ is a limit point of $C$.
  Thus $C$ is perfect.
\end{proof}

\section{Perfect Sets and Countability}

\begin{theorem}
  A nonempty perfect set is uncountable.
\end{theorem}

\begin{proof}
  Note that if $P$ is perfect, then $P$ is infinite
  (if we only have finitely many points, then they must be
  isolated). Now suppose that $P$ is only countably infinite.
  Then we can write
  \[P = \{x_1, x_2, \dots\}.\]
  Take $I_1$ to be a closed interval such that $x_1 \in I_1$
  and $x_1$ is not an endpoint of $I_1$.
  Since $x_1$ is not isolated in $P_1$, we can find $y_2 \in P$
  with $y_2 \ne x_1$
  such that $y_2 \in I_1$ and $y_2$ is not an endpoint of $I_1$.
  Then let $I_2 \subseteq I_1$ be a closed interval centered at
  $y_2$ such that $x_1 \notin I_2$. For example, we can do this
  by setting
  \[
    \epsilon = \frac{1}{2}\min\{y_2 - a, b - y_2, |x_1 - y_2|\}
  \]
  and letting $I_2 = [y_2 - \epsilon, y_2 + \epsilon]$.
  Since $y_2 \in P$, $y_2$ is not isolated, so we can find
  $y_3 \in P$ such that $y_3$ is not an endpoint of $I_2$ and
  $y_3 \ne x_2$. Pick $I_3$ centered at $y_3$ such that
  $x_2 \notin I_3$ and $I_3 \subseteq I_2$. Note that
  $I_3 \cap P \ne \emptyset$ since $y_3 \in I_3 \cap P$.
  From here, we continue by constructing
  $I_{n + 1} \subseteq I_n$ with $x_n \notin I_{n + 1}$ and
  $I_{n + 1} \cap P \ne \emptyset$. Let
  $k_n = I_n \cap P$. Clearly $k_n$ is closed, and $k_n$
  is also bounded since $k_n \subseteq I_n$. So $k_n$ is
  compact. By construction, $k_{n + 1} \subseteq k_n$, so
  by the nested interval property of compact sets,
  $\bigcap_{n = 1}^\infty k_n \ne \emptyset$. But
  $k_n \subseteq I_n$ and $x_n \notin I_{n + 1}$
  so $x_n \ne k_{n + 1}$. Since $k_n \subseteq P$, we must
  have $\bigcap_{n = 1}^\infty = \emptyset$. This is
  a contradiction, so $P$ must be uncountable.
\end{proof}
