\chapter{Oct.~19 -- Continuity}

\section{Algebraic Properties of Functional Limits}
\begin{theorem}
  Let $f : A \to \R$ and $c$ be a limit point of $A$.
  Then the following two statements are equivalent:
  \begin{enumerate}
    \item $\lim_{x \to c} f(x) = L$.
    \item For any sequence $\{x_n\} \subseteq A$ such that
      $x_n \ne c$ and
      $\lim_{n \to \infty} x_n = c$, we have
      $\lim_{n \to \infty} f(x_n) = L$.
  \end{enumerate}
\end{theorem}

\begin{proof}
  $(1) \Rightarrow (2)$: By assumption, for all $\epsilon > 0$,
  there exists $\delta > 0$ such that for any $x \in A$,
  $0 < |x - L| < \delta$ implies $|f(x) - L| < \epsilon$.
  Now let $\{x_n\} \subseteq A$ be an arbitrary sequence satisfying
  $x_n \ne c$ and $\lim_{n \to \infty} x_n = c$. Since $x_n$
  converges, for the
  $\delta$ given above we can find $N$ such that
  $|x_n - L| < \delta$ for all $n \ge N$. Then by $(1)$,
  we have $|f(x_n) - L| < \epsilon$. So for all $n \ge N$,
  we have $|f(x_n) - L| < \epsilon$, which means that
  $\lim_{n \to \infty} f(x_n) = L$.

  $(2) \Rightarrow (1)$: Argue by contradiction. Suppose that
  $(1)$ fails. So there exists some $\epsilon_0 > 0$ such that
  for all $\delta > 0$ and $|x - c| < \delta$, we have
  $|f(x) - L| > \epsilon_0$. Fix $\epsilon_0$ and take
  $\delta_n = \frac{1}{n}$. Pick $x_n \ne c$ such that
  $|x_n - c| < \frac{1}{n}$. By construction,
  $|f(x_n) - L | < \epsilon_0$. Also note that clearly
  $\lim_{n \to \infty} x_n = c$. Then by $(2)$, we have
  $\lim_{n \to \infty} f(x_n) = L$. Contradiction.
\end{proof}

\begin{corollary}[Algebraic limit theorem for functions]
  Given $\lim_{x \to c} f(x) = L$ and $\lim_{x \to c} g(x) = M$,
  \begin{enumerate}
    \item $\lim_{x \to c} kf(x) = kL$,
    \item $\lim_{x \to c} \left(f(x) + g(x)\right) = L + M$,
    \item $\lim_{x \to c} f(x)g(x) = LM$,
    \item $\lim_{x \to c} \frac{f(x)}{g(x)} = \frac{L}{M}$ if
      $M \ne 0$.
  \end{enumerate}
\end{corollary}

\begin{proof}
  Convert all the limits to sequences and apply the algebraic
  limit theorem.
\end{proof}

\begin{corollary}
  Let $f : A \to \R$ and $c$ be a limit point of $A$. If there
  exist sequences $\{x_n\}, \{y_n\} \subseteq A$ such that
  \[\lim_{n \to \infty} x_n = \lim_{n \to \infty} y_n = c\]
  but
  \[\lim_{n \to \infty} f(x_n) \ne \lim_{n \to \infty} f(y_n),\]
  then $\lim_{x \to c} f(x)$ does not exist.
\end{corollary}

\begin{example}
  Let $f(x) = \sin(\frac{1}{x})$ and
  $A = (-\infty, 0) \cup (0, \infty)$.
  Note that $c = 0$ is a limit point of $A$. Take
  \[x_n = \frac{1}{2n \pi}, \quad y_n = \frac{1}{2n\pi + \frac{\pi}{2}}.\]
  Then $\lim_{n \to \infty} x_n = \lim_{n \to \infty} y_n = 0$,
  but
  \[
  f(x_n) = \sin(2n \pi) = 0, \quad f(y_n) = \sin(2n \pi + \frac{\pi}{2}) = 1
  .\]
  So $\lim_{x \to 0} f(x)$ does not exist because $f$ oscillates
  too much at $x = 0$.
\end{example}

\begin{example}
  Let
  \[
    f(x) = \begin{cases}
      1, &  x < 0 \\
      2, &  x \ge 0.
    \end{cases}
  \]
  The limit does not exist at $x = 0$ since there is a jump
  discontinuity.
\end{example}

\begin{example}
  The function $f(x) = \frac{1}{x}$ does not have a limit at
  $x = 0$ since it diverges to infinity.
\end{example}

\section{Continuous Functions}
\begin{definition}
  A function $f : A \to \R$ is \textbf{continuous} at $c \in A$
  if for all $\epsilon > 0$, there exists $\delta > 0$ such that
  whenever $|x < c| < \delta$, we have
  $|f(x) - f(c)| < \epsilon$.
\end{definition}

\begin{remark}
  Here we require $f$ to be defined at $c$.
\end{remark}

\begin{definition}
  If $f$ is continuous at every point of $A$, then
  we say $f$ is \textbf{continuous on} $A$.
\end{definition}

\begin{theorem}
  Let $f : A \to \R$ and $c$ be a limit point of $A$. Then
  $f$ is continuous if one of the following hold:
  \begin{enumerate}
    \item For all $\epsilon > 0$, there exists $\delta > 0$ such
      that whenever $|x - c| < \delta$ for $x \in A$, then
      $|f(x) - f(c)| < \epsilon$.
    \item For all $\epsilon > 0$, there exists $\delta > 0$
      such that if $x \in V_\delta(c)$, then
      $f(x) \in V_\epsilon(f(x))$.
    \item For any sequence $\{x_n\} \subseteq A$ such that
      $\lim_{n \to \infty} x_n = c$, we have
      $\lim_{n \to \infty} f(x_n) = f(c)$.
  \end{enumerate}
\end{theorem}

\begin{proof}
  These follow from the topological and sequential definitions
  of limits.
\end{proof}

\begin{corollary}
  Let $f : A \to \R$ and $c$ be a limit point of $A$. If there
  exists $\{x_n\} \subseteq A$ with $\lim_{n \to \infty} x_n = c$
  but $f(x_n) \not\to f(c)$, then $f$ is not continuous at $c$.
\end{corollary}

\begin{theorem}
  Assume that $f, g : A \to \R$ are continuous at $c$. Then for
  every $k \in \R$, each of $kf, f + g, fg$ are also continuous at
  $c$. Also $\frac{f}{g}$ is continuous at $c$ if $g(c) \ne 0$.
\end{theorem}

\begin{proof}
  Convert all of these to sequences.
\end{proof}

\begin{theorem}
  Let $f : A \to \R$ and $g : B \to \R$ and assume that
  $f(A) = \{f(x) : x \in A\} \subseteq B$. Suppose that $f(x)$ is
  continuous at $c$ and $g$ is continuous at $f(c)$. Then
  $(g \circ f)(x) = g(f(x))$ is continuous at $c$.
\end{theorem}

\begin{proof}
  For all $\epsilon > 0$, since $g$ is continuous at $f(c)$, there
  exists $\eta > 0$ such that if $|y - f(c)| < \eta$,
  one has $|g(y) - g(f(c))| < \epsilon$. Since $f$ is continuous
  at $c$, for the $\eta$ given above, we can find $\delta > 0$
  such that if $|x - c| < \delta$, one has
  $|f(x) - f(c)| < \eta$. Therefore, for every $\epsilon > 0$,
  we can find $\delta$ such that if $|x - c| < \delta$, we have
  $|g(f(x)) - g(f(c))| < \epsilon$ since $|f(x) - f(c)| < \eta$.
  So $\lim_{x \to c} g(f(x)) = g(f(c))$.
\end{proof}

\section{Examples of Continuous Functions}
\begin{example}
  All polynomials are continuous on $\R$. To see this, first note
  that $f(x) = x$ is continuous on $\R$. For all $c > 0$, we have
  $f(c) = c$. Now fix a $c \in \R$. Then for all $\epsilon > 0$,
  take $\delta = \epsilon$. Then for all $x$ with
  $|x - c| < \delta$, we have
  \[|f(x) - f(c)| = |x - c| < \delta = \epsilon.\]
  So $f$ is continuous at $c$ and thus on all of $\R$. Now note
  that any
  polynomial $p(x)$ can be written as
  \[p(x) = a_0 + a_1 x + a_2x^2 + \dots + a_nx^n.\]
  Since constant functions are obviously continuous and we can
  write $x^2 = x \cdot x$, we conclude by the algebraic limit
  theorem that $p(x)$ is also continuous on $\R$.
\end{example}

\begin{example}
  Let
  \[
    g(x) = \begin{cases}
      x \sin(\frac{1}{x}), & x \ne 0 \\
      0, & x = 0.
    \end{cases}
  \]
  Then $g(x)$ is continuous at $x = 0$. This is because
  \[
  |g(x) - g(0)| = \left|x \sin\left(\frac{1}{x}\right) - 0\right|
  = \left|x\sin\left(\frac{1}{x}\right)\right| \le |x|
  .\]
  So taking $\delta = \epsilon$ works for any $\epsilon > 0$.
\end{example}

\begin{example}
  Let $A = \{x \in \R : x \ge 0\}$ and define $f : A \to \R$
  by $f(x) = \sqrt{x}$. Then $f$ is continuous on $A$.
  We can split the proof into two cases, $c = 0$ and $c > 0$.
  First let $c = 0$. Then for all $\epsilon > 0$, take
  $\delta = \epsilon^2$. Then for any $x \in A$ such that
  $|x - 0| < \delta$, we have
  \[
    |f(x) - f(0)| = |\sqrt{x} - \sqrt{0}|
    = |\sqrt{x}| < \sqrt{\delta} = \epsilon
  .\]
  So $f$ is continuous at $0$. Now consider $c > 0$. Note that
  \[|f(x) - f(c)| = |\sqrt{x} - \sqrt{c}| = \left|\frac{(\sqrt{x} - \sqrt{c})(\sqrt{x} + \sqrt{c})}{\sqrt{x} + \sqrt{c}}\right| = \frac{|x - c|}{\sqrt{x} + \sqrt{c}} \le \frac{|x - c|}{\sqrt{c}}.\]
  Then for any $\epsilon > 0$, take $\delta = \epsilon \sqrt{c}$.
  Then for any $x \in A$ such that $|x - c| < \delta$, we have
  \[
  |f(x) - f(c)| \le \frac{|x - c|}{\sqrt{c}} < \frac{\delta}{\sqrt{c}} = \epsilon
  .\]
  So $f$ is also continuous at $c$. Thus $f$ is continuous on
  all of $A$.
\end{example}

\begin{example}
  Let $h(x) = \lfloor x \rfloor$, where $h(x)$ takes the
  largest integer $n \le x$. Then we claim that $h$ is
  not continuous on $\Z$, but it is continuous on $\R \setminus \Z$.
  For the first part, let $m \in \Z$ and note that $h(m) = m$.
  Take $x_n = m - \frac{1}{n}$. Then $\lim_{n \to \infty} x_n = m$
  but $h(x_n) = m - 1$, so $\lim_{n \to \infty} h(x_n) \ne h(m)$.
  Thus $h$ is not continuous at $m$. Now for the second part,
  let $c \in \R \setminus \Z$. Then there exists $n \in \Z$ such
  that $n < c < n + 1$, so that $h(c) = n$. We can let
  $\delta = \min\{c - n, n + 1 - c\}$. Then for all
  $|x - c| < \delta$, we have $h(x) = n = h(c)$. So $h$ is
  continuous at $c$.
\end{example}
