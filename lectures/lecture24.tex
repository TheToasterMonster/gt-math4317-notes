\chapter{Nov.~28 -- Dini's Theorem, Metric Spaces}

\section{Dini's Theorem}

\begin{theorem}[Dini's theorem]
  Let $K$ be a compact set. Let $f : K \to \R$
  be a continuous function and $f_{n} : K \to \R$ for
  $n \in \N$
  be a sequence of continuous functions. If $f_n \to f$
  pointwise and $f_n(x) \ge f_{n + 1}(x)$, then $f_n \to f$
  uniformly.
\end{theorem}

\begin{proof}
  For each $n \in \N$, define $g_n(x) = f_n(x) - f(x)$. Now
  $\{g_n\}$ is a sequence of continuous functions and
  $g_n(x) \to 0$ for every $x \in K$. Also
  \[g_n(x) \ge g_{n + 1}(x) \ge 0\]
  since the $f_n$ is decreasing.
  Set $M_n = \sup \{g_n(x) : x \in K\}$. Then it suffices to
  show that $M_n \to 0$. For every $\epsilon > 0$, we define
  \[
    O_n = \{x \in K : g_n(x) \in (-\infty, \epsilon)\}.
  \]
  Observe that $O_n$ is open since we have
  $O_n = g_n^{-1}((-\infty, \epsilon))$, which is the
  preimage of an open set under a continuous function.
  Also $O_n \subseteq O_{n + 1}$ since $\{g_n\}$ is
  decreasing.
  For any fixed $x \in K$, since $g_n(x) \to x$, there
  exists $N_x$ such that $g_{N_x}(x) < \epsilon$. In
  particular, this means that $x \in O_{N_x}$. Then
  \[
    K = \bigcup_{n = 1}^\infty O_n.
  \]
  Observe that $\{O_n\}$ is an open cover for $K$ and $K$
  is compact, so
  we may find a finite subcover
  $\{O_{n_k}\}_{k = 1}^M$ satisfying
  \[
    K \subseteq \bigcup_{k = 1}^M O_{n_k} \subseteq O_{n_M}.
  \]
  Note that the last equality is from
  $O_n \subseteq O_{n + 1}$. So for all $n \ge n_M$,
  we have $K \subseteq O_n$. In particular, for all
  $x \in K$, we have $g_n(x) < \epsilon$. Furthermore
  $0 \le g_n(x)$ by construction, so $g_n \to 0$
  uniformly and $M_n \to 0$.
\end{proof}

\begin{example}
  Let's see what happens when the compactness condition
  is removed. Let $K = (0, 1)$ and $f_n(x) = x^n$.
  Clearly $f_{n + 1} \le f_n$ and $f_n$ is continuous
  for every $n$. Now note that
  \[\sup\{x^n : 0 < x < 1\} = 1,\]
  so this convergence is not uniform.
\end{example}

\begin{example}
  Let $K = \R$ and
  \[
    f_n(x) = \begin{cases}
      0 & \text{if } x \le n \\
      \frac{1}{n}(x - n) & \text{if } n < x < 2n \\
      1 & \text{if } x \ge 2n.
    \end{cases}
  \]
  Note that $f_n \ge f_{n + 1}$ and $f_n \to 0$ pointwise.
  But $\sup\{f_n\} = 1$, so we do not have $f_n \to f$
  uniformly.
\end{example}

\begin{example}
  Now let's try removing the continuity condition.
  Let $K = [0, 1]$ and
  \[
    f_n(x) = \begin{cases}
      -n (x - \frac{1}{n}) & \text{if } 0 \le x \le \frac{1}{n} \\
      0 & \text{if } \frac{1}{n} < x \le 1
    \end{cases}
  \]
  We have
  \[
    f_n(x) \to f(x)
    = \begin{cases}
      1 & \text{if } x = 0 \\
      0 & \text{if } 0 < x \le 1
    \end{cases}
  \]
  pointwise. Note that $f(x)$ is not continuous.
  As a result, we have
  \[\sup\{f_n(x) - f(x) : x \in K\} = 1,\]
  so we do not have $f_n \to f$ uniformly.
\end{example}

\begin{example}
  Finally we try removing the monotonicity condition.
  Let $K = [0, 1]$ and
  \[
    f_n(x) = \begin{cases}
      0 & \text{if } x \le \frac{1}{n} \\
      2n(x - \frac{1}{n}) & \text{if } \frac{1}{n} < x \le \frac{3}{2n} \\
      -2n(x - \frac{2}{n}) & \text{if } \frac{3}{2n} < x < \frac{2}{n} \\
      0 & \text{if } x \ge \frac{2}{n}.
    \end{cases}
  \]
  Observe that $f_n \to 0$ pointwise. However,
  $\sup\{f_n(x) : x \in K\} = 1$, so the convergence is
  not uniform. This is because $f_{n + 1} \le f_n$ fails.
\end{example}

\section{Metric Spaces}

\begin{definition}
  Given a set $X$, a function $d : X \times X \to \R$ is a
  \textbf{metric} on $X$ if for all $x, y \in X$,
  \begin{enumerate}[(i)]
    \item (Positive-definiteness)\ $d(x, y) \ge 0$, with $d(x, y) = 0$ if and only
      if $x = y$,
    \item (Symmetry)\ $d(x, y) = d(y, x)$,
    \item (Triangle inequality)\ for all $z \in X$,
      $d(x, y) \le d(x, z) + d(z, y)$.
  \end{enumerate}
  A \textbf{metric space} is a set $X$ with a metric $d$.
  We typically write this as a pair $(X, d)$.
\end{definition}

\begin{example}
  On $\R$, the absolute value $d(x, y) = |x - y|$ is a metric.
\end{example}

\begin{example}
  On $\R^2$, for points $x = (x_1, x_2)$ and $y = (y_1, y_2)$,
  the Euclidean distance
  \[d_1 (x, y) = \sqrt{(x_1 - y_1)^2 + (x_2 - y_2)^2}\]
  is a metric. The following
  \[
    d_2(x, y) = \max\{|x_1 - y_1|, |x_2 - y_2|\}
  \]
  is also a metric.
\end{example}

\begin{definition}
  Given $\epsilon > 0$ and $x \in X$ the
  \textbf{$\epsilon$-neighborhood} in $(X, d)$ is
  \[V_\epsilon(x) = \{y \in X : d(x, y) < \epsilon\}.\]
\end{definition}

\begin{example}
  For $\R^2$, the neighborhood
  $V_1(0)$ in $(\R^2, d_1)$ is the unit disk.
  However, the same
  neighborhood $V_1(0)$ in $(\R^2, d_2)$
  is a square of
  side length $2$.
\end{example}

\begin{definition}
  Let $(X, d)$ be a metric space. A sequence
  $\{x_n\} \subseteq X$ \textbf{converges} to
  $x \in X$ if for all
  $\epsilon > 0$, there exists $N$ such that for all
  $n \ge N$, we have $d(x, x_n) < \epsilon$.
\end{definition}

\begin{definition}
  A sequence $\{x_n\}$ in $(X, d)$ is \textbf{Cauchy} if
  for all $\epsilon > 0$, there exists $N$ such that
  for all $m, n \ge N$, we have $d(x_m, x_n) < \epsilon$.
\end{definition}

\begin{prop}
  Convergent sequences in a metric space are Cauchy.
\end{prop}

\begin{proof}
  Pretty much carry over the same proof from
  $| \cdot |$ in $\R$.
\end{proof}

\begin{definition}
  A metric space $(X, d)$ is \textbf{complete} if every
  Cauchy sequence in $X$ converges to an element in $X$.
\end{definition}

\begin{remark}
  The real numbers $\R$ form a complete metric space
  (recall the axiom of completeness),
  but the same is not true for $\Q$.
\end{remark}

\begin{definition}
  Let $(X, d_1)$ and $(Y, d_2)$ be metric spaces.
  A function $f : X \to Y$ is \textbf{continuous} at
  $x \in X$ if for all $\epsilon > 0$, there exists
  $\delta > 0$ such that
  \[d_2(f(x), f(y)) < \epsilon\]
  for all $y$ with $d_1(x, y) < \delta$.
\end{definition}

\begin{definition}
  A set $O \subseteq X$ is \textbf{open} if for every
  $x \in O$, we can find $\epsilon > 0$ such that
  $V_\epsilon(x) \subseteq O$.
\end{definition}

\begin{remark}
  We can similarly define closed sets, limit points, etc.
  on metric spaces.
\end{remark}

\begin{definition}
  A subset $K$ of a metric space $(X, d)$ is \textbf{compact}
  if every sequence in $K$ has a convergent subsequence
  that converges to a point in $K$.
\end{definition}
