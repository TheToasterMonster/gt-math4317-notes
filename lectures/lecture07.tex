\chapter{Sept.~12 -- The Cauchy Criterion}

\section{Cauchy Sequences}
\begin{definition}
  A sequence $\{a_n\}$ is called a \textbf{Cauchy}
  sequence if for every $\epsilon > 0$, there exists
  $N \in \N$ such that for every $m, n \ge N$,
  one has $|a_m - a_n| < \epsilon$.
  \footnote{The Cauchy condition controls the
    \textit{oscillation} of the \textit{tail} of a sequence.}
\end{definition}

\begin{theorem}
  \label{thm:cauchy-convergent}
  Every convergent sequence is a Cauchy sequence.
\end{theorem}

\begin{proof}
  Assume $\lim_{n \to \infty} a_n = a$. Then for every
  $\epsilon > 0$, we can find $N$ such that for
  every $n \ge N$, we have $|a_n - a| < \epsilon/2$.
  Then for every $m, n \ge N$, we have
  \[
  |a_m - a_n| = |a_m - a + a - a_n| \le
  |a_m - a| + |a - a_n| <
  \frac{\epsilon}{2} + \frac{\epsilon}{2} = \epsilon
  \]
  by the triangle inequality.
\end{proof}

\begin{lemma}
  Every Cauchy sequence is bounded.
\end{lemma}

\begin{proof}
  Suppose $\{x_n\}$ is a Cauchy sequence.
  Pick $\epsilon = 1$. Then there exists $N$ such that
  for all $m, n \ge N$, we have $|x_m - x_n| < 1$.
  Fixing $m = N$, we know that for all $n \ge N$,
  $|x_N - x_n| < 1$. So $|x_n| \le |x_N| + 1$ for all
  $n \ge N$. Set
  \[
    M = \max\{|x_1|, |x_2|, \dots, |x_{N-1}|, |x_N| + 1\}
  .\]
  Then $\sup |x_n| \le M$ by construction.
\end{proof}

\begin{theorem}[Cauchy criterion]
  A sequence converges if and only if it is a Cauchy
  sequence.
\end{theorem}

\begin{proof}
  $(\Rightarrow)$\, This is Theorem \ref{thm:cauchy-convergent}.

  $(\Leftarrow)$\, Suppose $\{a_n\}$ is a Cauchy sequence.
  Since $\{a_n\}$ is Cauchy, we know $\sup |a_n| \le M$
  for some $M \in \R$. Then by the Bolzano-Weierstrass
  theorem, we can find a convergent subsequence
  $\{a_{n_k}\}$ such that
  $\lim_{k \to \infty} a_{n_k} = a$. We show that
  we also have $\lim_{n \to \infty} a_n = a$.

  For every $\epsilon > 0$, we can find $N_1$ such that
  for all $m, n \ge N_1$, we have
  $|a_m - a_n| < \epsilon/2$. Since
  $\lim_{k \to \infty} a_{n_k} = a$, there is some
  $K$ such that for all $k \ge K$, we have
  $|a_{n_k} - a| < \epsilon/2$. Take
  \[N \ge \max\{N_1, n_K\}.\]
  We can find $K_0$ such that $n_{K_0} \ge N$.
  Then for every $n \ge N$,
  \[
    |a_n - a| = |a_n - a_{n_{K_0}} + a_{n_{K_0}} - a|
    \le |a_n - a_{n_{K_0}}| + |a_{n_{K_0}} - a|
    < \frac{\epsilon}{2} + \frac{\epsilon}{2} = \epsilon
  \]
  by the triangle inequality and the Cauchy condition.
\end{proof}

\begin{remark}
  The Cauchy condition allows us to show that a sequence
  converges without explicitly providing its limit.
\end{remark}

\section{Revisiting Completeness}
This is the way we have discussed completeness
(ordered by implication):
\begin{itemize}
  \item Axiom of Completeness
    \begin{itemize}
      \item Nested intervals property
        \begin{itemize}
          \item Bolzano-Weierstrass theorem
            \begin{itemize}
              \item Cauchy criterion
            \end{itemize}
        \end{itemize}
      \item Monotone convergence theorem.
    \end{itemize}
\end{itemize}

But this is not the only way to do so: We have several
ways of choosing axioms to define completeness. For example,
we can also prove the nested intervals property using
the monotone convergence theorem.

\begin{exercise}
  The monotone convergence theorem implies the nested
  intervals property.
\end{exercise}

\begin{proof}
  Let $I_n = [a_n, b_n]$ with $I_{n+1} \subseteq I_n$.
  In particular, $\{a_n\}$ is increasing and
  bounded ($b_1$ is an upper bound). So by
  the monotone convergence theorem,
  $\lim_{n \to \infty} a_n = a$ exists.

  Left as an exercise to show that $a \in I_n$ for
  all $n$.
\end{proof}

\begin{exercise}
  Given the Archimedean property,
  the nested intervals property implies the Axiom of
  Completeness.
\end{exercise}

\begin{proof}
  Note that $\frac{1}{2^n} \to 0$ as $n \to \infty$.
  This is because for
  every $\epsilon > 0$, we can find $N$ such that
  $\frac{1}{N} < \epsilon$ by the Archimidean property.
  Then
  \[
    \frac{1}{2^N} < \frac{1}{N}
  \]
  for all $N \in \N$. So
  $\lim_{n \to \infty} \frac{1}{2^n} = 0$.

  Now let $S$ be a nonempty set which is bounded above.
  Let $U$ be an upper bound for $S$. Take $s \in S$.
  Set $a_1 = s$, $b_1 = U$. Consider
  \[
    \frac{s + U}{2}
  .\]
  If $\frac{s + U}{2}$ is an upper bound for $S$, then
  we set $a_2 = a_1 = s$, $b_2 = \frac{s + U}{2}$. If
  $\frac{s + U}{2}$ is not an upper bound for $S$, then
  we set $a_2 = \frac{s + U}{2}$, $b_2 = b_1 = U$.
  Note that $[a_2, b_2] \subseteq [a_1, b_1]$.
  Repeat the same process for $a_n$ and $b_n$ to
  obtain the closed intervals
  \[[a_1, b_1] \supseteq [a_2, a_2] \supseteq [a_3, b_3] \supseteq \dots.\]
  By the nested interval properties, the intersection
  $\bigcap_{n = 1}^\infty [a_n, b_n]$ is nonempty.
  Note that
  \begin{align*}
    \left|[a_1, b_1]\right| &= |b_1 - a_1| = |U - s| \\
    \left|[a_2, b_2]\right| &= |b_2 - a_2| = \left|\frac{U - s}{2}\right| \\
                            &\ \ \vdots \\
    \left|[a_n, b_n]\right| &= |b_n - a_n| = \frac{2}{2^n}|U - s|
  \end{align*}
  So there is only one $x \in \R$ such that
  $x \in \bigcap_{n = 1}^\infty [a_n, b_n]$. We claim
  that $\sup S = x$.

  Note that $x \in [a_n, b_n]$ for all $n$. So
  $a_n$ is not an upper bound and $b_n$ is an upper bound.
  Suppose for contradiction that $x$ is not an upper bound.
  Then there exists $s_0 \in S$ such that $s_0 > x$.
  Since $|[a_n, b_n]| \to 0$, there exists an $N$ such
  that whenever $n \ge N$,
  \[
    |[a_n, b_n]| < \frac{1}{2}|s_0 - x|
  .\]
  Since $x \in [a_n, b_n]$, this implies that $s_0 > b_n$,
  which is a contradiction with $b_n$ being an upper bound.

  Use a similar idea to show that $x$ is the \textit{least}
  upper bound.
\end{proof}

\begin{remark}
  These are all different ways to understand the
  same idea of completeness.
\end{remark}
