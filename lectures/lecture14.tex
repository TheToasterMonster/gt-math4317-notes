\chapter{Oct.~12 -- Connected Sets}

\section{Discussion of Exam 1}
\begin{exercise}
  Investigate the convergence of the following series:
  \[\sum_{n=1}^\infty \frac{\sin(\pi n / 3)}{\sqrt{n}}.\]
  Either state the test/reason used to show convergence
  (and verify the hypotheses required), or show why the
  sum diverges.
\end{exercise}

\begin{proof}[Solution]
  Observe that
  \[
    \sin\left(\frac{\pi}{3}n\right) = \begin{cases}
      0, & n = 3k\\
      \frac{\sqrt{3}}{2}, & n = 6k + 1, 6k + 2 \\
      -\frac{\sqrt{3}}{2}, & n = 6k + 4, 6k + 5. \\
    \end{cases}
  \]
  But note that we cannot just regroup since the series
  is not absolutely convergent! Instead, look at the partial sums
  (here, the sum is finite so we can regroup)
  and show that
  \[
    |s_m - s_n| \le \frac{C}{\sqrt{n}}
  \]
  for some constant $C$. Then argue by the Cauchy criterion that
  the series converges.
\end{proof}

\section{Connected Sets}
\begin{definition}
  Two nonempty sets $A, B \subseteq \R$ are \textbf{separated}
  if $\overline{A} \cap B = \overline{B} \cap A = \varnothing$.
\end{definition}

\begin{definition}
  A set $E \subseteq \R$ is \textbf{disconnected} if $E$ can be
  written as $E = A \cup B$ such that $A$ and $B$ (both nonempty)
  are separated. A set which is not disconnected is
  \textbf{connected}.
\end{definition}

\begin{example}
  Let $A = (1, 2)$ and $B = (2, 5)$. The set
  $E = A \cup B = (1, 2) \cup (2, 5)$ is disconnected since
  \[\overline{A} \cap B = [1, 2] \cap (2, 5) = \varnothing = (1, 2) \cap [2, 5] = A \cap \overline{B}.\]
\end{example}

\begin{example}
  Let $C = (1, 2)$ and $D = [2, 5)$. Then $F = C \cup D$ is
  connected.
\end{example}

\begin{example}
  The set of rational numbers $\Q$ is disconnected.
  For example, we can take $A = \Q \cap (-\infty, \sqrt{2})$
  and $B = \Q \cap (\sqrt{2}, \infty)$. Clearly
  $A \cup B = \Q$ but $A$ and $B$ are separated.
\end{example}

\begin{theorem}
  A set $E \subseteq \R$ is connected if and only if for every
  $A, B \ne \varnothing$ such that $E = A \cup B$, there exists
  $\{x_n\} \subseteq A$ such that
  $\lim_{n \to \infty} x_n = x \in B$ or $\{x_n\} \subseteq B$
  such that $\lim_{n \to \infty} x_n = x \in A$.
\end{theorem}

\begin{proof}
  $(\Rightarrow)$\, Argue by contrapositve and suppose that
  the latter condition fails.
  Then we can find $A^0, B^0 \ne \varnothing$ such that
  for all $\{x_n\} \subseteq A^0$,  we have
  $\lim_{n \to \infty} x_n = x \ne B^0$. So
  $\overline{A^0} \cap B^0 = \varnothing$. By a similar argument,
  $\overline{B^0} \cap A^0 = \varnothing$. So $E$ is disconnected.

  $(\Leftarrow)$\, Use a similar argument by contrapositive.
\end{proof}

\begin{theorem}
  A set $E \subseteq \R$ is connected if and only if
  for any $a, b \in E$, we have $a < c < b$ implies $c \in E$.
\end{theorem}

\begin{proof}
  $(\Rightarrow)$\, Let $a, b \in E$ and pick $a < c < b$.
  Set $A = (-\infty, c) \cap E$ and $B = (c, \infty) \cap E$.
  Note that $A, B \ne \varnothing$ since $a \in A$ and $b \in B$.
  Furthermore,
  \[\overline{A} \cap B = A \cap \overline{B} = \varnothing.\]
  If $E = A \cup B$, then $E$ is disconnected, so
  $A \cup B \ne E$.
  But $(A \cup B)^c = \{c\}$, so we must have
  $c \in E$.

  $(\Leftarrow)$\, Let $E = A \cup B$
  such that $A, B \ne \varnothing$ and $A \cap B = \varnothing$.
  Take $a_0 \in A, b_0 \in B$. Without loss of generality,
  assume $a_0 < b_0$. Consider $I_0 = [a_0, b_0]$. Bisect
  $I_0$ let $x_0$ be the center of $I_0$. Since
  $a_0 < x_0 < b_0$ and $a_0, b_0 \in E$, we must have
  $x_0 \in E$. But $E = A \cup B$, so $x_0 \in A$ or $x_0 \in B$.
  If $x_0 \in A$, set $a_1 = x_0$ and $b_1 = b_0$. Otherwise,
  set $a_1 = a_0$ and $b_1 = x_0$. Then take $I_1 = [a_1, b_1]$.
  Repeat this process to construct a sequence of nested intervals
  with $I_n = [a_n, b_n]$ such that $a_n \in A$ and $b_n \in B$.
  By the nested interval property, we have
  $\bigcap_{n = 1}^\infty I_n \ne \varnothing$. But
  \[|I_n| = |b_n - a_n| = \frac{1}{2^n}(b_0 - a_0) \to 0,\]
  so $\bigcap_{n = 1}^\infty I_n = \{x\}$ for some $x \in \R$.
  So $\lim_{n \to \infty} a_n = \lim_{n \to \infty} b_n = x$.
  Since $I_n \subseteq E$, we have $x \in E = A \cup B$. So
  $x \in A$ or $x \in B$. If $x \in A$, then
  $A \cup \overline{B} \ne \varnothing$, and if $x \in B$, then
  $\overline{A} \cap B \ne \varnothing$. Thus $E$ is connected.
\end{proof}

\section{Density in a Set}
\begin{definition}
  A set $A \subseteq \R$ is called an \textbf{$F_\sigma$ set}
  if it can be written as a countable union of closed sets.
  A set $A \subseteq \R$ is called a \textbf{$G_\delta$ set}
  if it can be written as a countable intersection of open sets.
\end{definition}

\begin{remark}
  Recall that the countable union of closed sets is not necessarily
  closed, and the countable intersection of open sets is not
  necessarily open.
\end{remark}

\begin{definition}
  A set $G \subseteq \R$ is \textbf{dense} in $\R$ if for every
  $a, b \in \R$, there exists $x \in G$ such that $a < x < b$.
\end{definition}

\begin{remark}
  Recall that $\Q$ is dense in $\R$.
\end{remark}

\begin{theorem}
  \label{thm:countable-intersection-dense}
  If $\{G_1, G_2, \dots\}$ is a countable collection of dense
  open sets, then
  $\bigcap_{n = 1}^\infty G_n \ne \varnothing$.
\end{theorem}

\begin{proof}
  Take $x_1 \in G_1$. Since $G_1$ is open, there exists
  $\epsilon_1$ such that
  $(x_1 - \epsilon_1, x_1 + \epsilon_1) \subseteq G_1$.
  Take
  \[I_1 = \left[x_1 - \frac{\epsilon_1}{2}, x_1 + \frac{\epsilon}{2}\right] \subseteq G_1.\]
  Since $G_2$ is dense, $I_1 \cap G_2 \ne \varnothing$.
  Take $x_2 \in I_1 \cap G_2$ such that $x_2$ is not an
  endpoint of $I_1$. Since $G_2$ is open,
  we can similarly find $\epsilon_2$ such that
  $(x_2 - \epsilon_2, x_2 + \epsilon_2) \subseteq G_2$.
  Then take
  \[I_2 = \left[x_2 - \frac{\epsilon_2}{2}, x_2 + \frac{\epsilon_2}{2}\right] \cap I_1 \subseteq G_2.\]
  Repeat this to get the sequence of nested intervals
  $I_1 \supseteq I_2 \supseteq I_3 \supseteq \dots$. Then
  $\varnothing \ne \bigcap_{n = 1}^\infty I_n \subseteq \bigcap_{n = 1}^\infty G_n$.
\end{proof}

\begin{corollary}
  The set of real numbers $\R$ cannot be written as
  \[\R = \bigcup_{n = 1}^\infty F_n\]
  such that $F_n$ is closed and $F_n$ contains no open nonempty
  interval.
\end{corollary}

\begin{proof}
  Suppose
  \[\underbrace{\R^c}_{= \varnothing} = \left(\bigcup_{n = 1}^\infty F_n\right)^c = \bigcap_{n = 1}^\infty F_n^c\]
  by de Morgan's laws. Then $F_n^c$ is open and dense in $\R$. To
  see
  that $F_n^c$ is dense in $\R$, let $a, b \in \R$. Note that
  $(a, b) \notin F_n$, so $(a, b) \cap F_n^c \ne \varnothing$.
  Hence there exists $x \in F_n^c$ such that $a < x < b$. Then
  this is a contradiction with Theorem
  \ref{thm:countable-intersection-dense}.
\end{proof}

\begin{definition}
  A set $E$ is \textbf{nowhere dense} if $\overline{E}$ contains no
  nonempty open interval.
\end{definition}

\begin{remark}
  This is equivalent to saying that $\overline{E}^c$ is dense.
\end{remark}
