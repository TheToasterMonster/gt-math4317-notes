\chapter{Sept.~21 -- Iterated Sums, Topology}

\section{Double Sums}
Given a set of doubly indexed real numbers
$\{a_{ij} : i, j \in \N\}$, consider the
sums
\[
  \sum_{j = 1}^\infty \sum_{i = 1}^\infty a_{ij}, \quad
  \sum_{i = 1}^\infty \sum_{j = 1}^\infty a_{ij}.
\]
Are these sums equal? The answer is no, in general.

\begin{example}
  Define $a_{ij}$ by
  \[
    a_{ij} = \begin{cases}
      \left(\frac{1}{2}\right)^{j - i} & \text{if $j > i$}, \\
      -1 & \text{if $i = j$}, \\
      0 & \text{if $j < i$}.
    \end{cases}
  \]
  This looks like
  \[
    \begin{array}{c|ccccc}
      & a_{11} & a_{12} & a_{13} & a_{14} & \cdots \\
      \hline
      a_{11} & -1 & \frac{1}{2} & \frac{1}{4} & \left(\frac{1}{2}\right)^3 & \cdots \\
      a_{21} & 0 & -1 & \frac{1}{2} & \left(\frac{1}{2}\right)^2 & \cdots \\
      a_{31} & 0 & 0 & -1 & \frac{1}{2} & \cdots \\
      \vdots & \vdots & \vdots & \vdots & \vdots & \ddots
    \end{array}
  \]
  Then the first sum (over the columns) is
  \[
    \sum_{j = 1}^\infty \sum_{i = 1}^\infty a_{ij}
    = \sum_{j = 1}^\infty -\left(\frac{1}{2}\right)^{j - 1}
    = -\frac{1}{1 - \frac{1}{2}} = -2
  .\]
  Meanwhile, the second sum (over the rows) is
  \[
    \sum_{i = 1}^\infty \sum_{j = 1}^\infty a_{ij}
    = \sum_{i = 1}^\infty \left(-1 + \sum_{k = 1}^\infty \left(\frac{1}{2}\right)^{k}\right)
    = \sum_{i = 1}^\infty \left(-1 + 1\right)
    = \sum_{i = 1}^\infty 0 = 0
  .\]
  Notice that these two sums are not the same.
\end{example}

\begin{remark}
  We cannot always exchange the order of a double
  sum.
  \footnote{\textit{Fubini's theorem} gives conditions under which we can do this for iterated \textit{integrals} (when the integrand is \textit{absolutely integrable}).}
\end{remark}

\subsection{Convergence of Double Sums}
\begin{definition}
  We say that $\sum_{i = 1} \sum_{j = 1} |a_{ij}|$
  \textbf{converges}
  if for all $i \in \N$,
  $\sum_{j = 1}^\infty |a_{ij}|$ converges to some real
  number $b_i$ and $\sum_{i = 1}^\infty b_i$
  converges.
\end{definition}

\begin{theorem}
  Consider $\{a_{ij} : i, j \in \N\}$. If
  \[
    \sum_{i = 1}^\infty \sum_{j = 1}^\infty |a_{ij}|
  \]
  converges, then both
  \[
    \sum_{i = 1}^\infty \sum_{j = 1}^\infty a_{ij} \quad
    \text{and} \quad
    \sum_{j = 1}^\infty \sum_{i = 1}^\infty a_{ij}
  \]
  converge to the same limit, i.e.
  \[
    \sum_{i = 1}^\infty \sum_{j = 1}^\infty a_{ij}
    = \sum_{j = 1}^\infty \sum_{i = 1}^\infty a_{ij}
    = \lim_{n \to \infty} S_{nn}
  \]
  where $S_{nn} = \sum_{i = 1}^n \sum_{j = 1}^n a_{ij}$.
\end{theorem}

\begin{proof}
  Go look this up.
\end{proof}

\section{Basic Topology in \texorpdfstring{$\R$}{R}}
\subsection{Open Sets}
\begin{definition}
  For all $a \in \R$, $\epsilon > 0$, we define
  \[V_\epsilon(a) = \{x \in \R : |x - a| < \epsilon\}\]
  to be the \textbf{$\epsilon$-neighborhood} of $a$.
\end{definition}

\begin{definition}
  A set $U \subseteq \R$ is \textbf{open} if for all
  $a \in U$, there exists $\epsilon > 0$ such that
  $V_\epsilon(a) \subseteq U$.
\end{definition}

\begin{example}
  The set $\R$ is open: Simply take $\epsilon = 1$ for
  any choice of $a \in \R$.
\end{example}

\begin{example}
  The open interval $(c, d)$ is open: For any
  $x \in (c, d)$, take $\epsilon = \min\{x - c, d - x\}$.
\end{example}

\begin{theorem}\leavevmode
  \begin{enumerate}
    \item The union of an arbitrary collection of open sets
      is open.
    \item The intersection of a finite collection of
      open sets is open.
  \end{enumerate}
\end{theorem}

\begin{proof}
  (1)\, Let $\{U_\lambda : \lambda \in \Lambda\}$ be a
  collection of open sets and consider
  $\bigcup_{\lambda \in \Lambda} U_\lambda$. For
  every $a \in \bigcup_{\lambda \in \Lambda} U_\lambda$,
  there is $\lambda'$ such that $a \in U_{\lambda'}$.
  Since $U_{\lambda'}$ is open, there exists
  $\epsilon > 0$ such that
  \[V_\epsilon(a) \subseteq U_{\lambda'} \subseteq \bigcup_{\lambda \in \Lambda} U_\lambda.\]
  So $\bigcup_{\lambda \in \Lambda} U_\lambda$ is open.

  (2)\, Let $U_1, \ldots, U_n$ be a collection of
  open sets and consider $\bigcap_{j = 1}^n U_j$.
  For every $a \in \bigcap_{j = 1}^n U_j$, note that
  $a \in U_j$ for all $j = 1, \ldots, n$. Since
  $U_j$ is open, there exists $\epsilon_j > 0$ such that
  $V_{\epsilon_j}(a) \subseteq U_j$. Then take
  \[\epsilon = \min_{i \le j \le n}\{\epsilon_j\},\]
  which exists since the collection is finite.
  Since $\epsilon_j > 0$, we have $\epsilon > 0$ as well.
  By construction,
  \[V_\epsilon(a) \subseteq V_{\epsilon_j}(a) \subseteq U_j\]
  for all $j$. So
  $V_\epsilon(a) \subseteq \bigcap_{j = 1}^n U_j$, and
  thus $\bigcap_{j = 1}^n U_j$ is open.
\end{proof}

\begin{example}
  Consider the family of open sets
  $U_n = (-\frac{1}{n}, \frac{1}{n})$ for $n \in \N$. Notice
  that their intersection
  \[\bigcap_{n = 1}^\infty U_n = \{0\}\]
  is not open.
\end{example}

\subsection{Limit Points}
\begin{definition}
  A point $x$ is a \textbf{limit point} of a set
  $A$ if for all $\epsilon > 0$, 
  we have
  \[(V_\epsilon(x) \cap A) \setminus \{x\} \ne \varnothing,\]
  i.e.~there is some other point in the
  $\epsilon$-neighborhood of $x$ that is also in $A$.
\end{definition}

\begin{theorem}
  \label{thm:limit-point}
  A point $x$ is a limit point of $A$ if and only
  if $x = \lim_{n \to \infty} a_n$ for some
  sequence $\{a_n\}$ with $a_n \ne x$ and $a_n \in A$.
\end{theorem}

\begin{proof}
  ($\Rightarrow$)\, Suppose $x$ is a limit point of $A$.
  Take $\epsilon = 1 / n$ and pick
  $a_n \in (V_{1 / n}(x) \cap A)$ such that $a_n \ne x$.
  For such a sequence $\{a_n\}$, for all $\epsilon > 0$,
  if $N \ge 1 / \epsilon$, then for all $n \ge N$,
  we have
  \[|a_n - x| \le \frac{1}{N} < \epsilon.\]
  So $\lim_{n \to \infty} a_n = x$.

  ($\Leftarrow$)\, Assume such a sequence $\{a_n\}$ exists.
  Then for any $\epsilon > 0$, there exists $N$
  such that $|a_n - x| < \epsilon$ for all $n \ge N$.
  Note that $a_N \in V_\epsilon(x)$, and also
  $a_N \in A$ and $a_N \ne x$. So
  $a_N \in (V_\epsilon(x) \cap A) \setminus \{x\}$,
  i.e.~this set is not empty. Thus $x$ is a limit point
  of $A$.
\end{proof}
