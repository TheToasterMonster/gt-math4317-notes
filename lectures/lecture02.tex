\chapter{Aug.~24 -- The Axiom of Completeness}

The number system $\Q$ is pretty good (it is a field),
but recall that we are unable to take limits. For
instance, take the sequence $x_0 = 2$ and
\[x_{n+1} = \frac{1}{2}\left(x_n + \frac{2}{x_n}\right)\]
for $n \ge 1$. All the $x_i$ are rational, but
$x_n \to \sqrt{2} \notin \Q$. This shows that there are
gaps in $\Q$. The real numbers $\R$ will fill
these gaps (completeness).

\begin{axiom}[Axiom of completeness]
Every nonempty set of real numbers that are bounded
above has a least upper bound.
\end{axiom}

Note that this least upper bound is \textit{unique}.

\section{Suprema and Infima}

\begin{definition}
  Let $S \subseteq \R$. The set $S$ is
  \textbf{bounded above} if there exists $u \in R$
  such that  $s \le u$ for all $s \in S$.
  We say that $u$ is an \textbf{upper bound} of $S$.
\end{definition}

We define \textbf{bounded below} and \textbf{lower bound}
similarly.

\begin{definition}
  $S$ is said to be \textbf{bounded} if it is
  both bounded above and below. Otherwise we say that
  $S$ is \textbf{unbounded}.
\end{definition}

\begin{example}
  $\N = \{1, 2, 3, \dots\}$ is bounded below but not
  above.
\end{example}

\begin{example}
  The set
  \[\left\{\frac{1}{k} : k \in \N\right\} = \left\{1, \frac{1}{2}, \frac{1}{3}, \dots\right\}\]
  is bounded.
\end{example}

\begin{example}
  $\emptyset$ is bounded.
\end{example}

\begin{definition}
  We say $u \in \R$ is the \textbf{least upper bound}
  or \textbf{supremum} of a nonempty set $S \subseteq \R$
  if
  \begin{enumerate}
    \item $u$ is an upper bound of $S$.
    \item $u \le v$ for any upper bound $v$ of $S$.
  \end{enumerate}
  We write $u = \sup S$.
\end{definition}

The \textbf{greatest lower bound} or \textbf{infimum}
of $S$ is defined similarly, denoted $\inf S$.

\begin{example}
  \[S = \left\{\frac{1}{k} : k \in \N\right\}.\]
  $\sup S = 1$,  $\inf S = 0$.
\end{example}

\begin{definition}
  Let $S \subseteq \R$. We say a real number
  $M \in S$ is a \textbf{maximal element} or
  \textbf{maximum} of $S$ if $s \le M$ for all
  $s \in S$.
\end{definition}

The \textbf{minimal element} or \textbf{minimum}
is defined similarly.

\begin{example}
  $[0, 1)$ is bounded, but has no maximum. The minimum is
  $0$.
\end{example}

\begin{example}
  The set
  \[\{2^{-n} : n \in \N\} = \left\{\frac{1}{2}, \frac{1}{4}, \frac{1}{8}, \dots\right\}\]
  is bounded, but has no minimum. The maximum is
  $\frac{1}{2}$.
\end{example}

\begin{example}
  $\emptyset$ is bounded but has no minimum or maximum.
\end{example}

\begin{exercise}
  Let $A \subseteq \R$ be bounded above. Let  $c \in \R$
  and define
  \[c + a := \{a + c : a \in A\}.\]
  Then  $\sup (A + c) = c + \sup A$.
\end{exercise}

\begin{proof}
  Let $s = \sup A$. By definition, we know  $a \le s$
  for all  $a \in A$, which implies
  $a + c \le s + c$. So $s + c$ is an upper bound
  for $c + A$.
  Now let  $b$ be an arbitrary upper bound
  for $c + A$.  For all $a \in A$, we have
  $a + c \le b$, which implies $a \le b - c$. So
  $b - c$ is an upper bound for  $A$. By construction,
  $s \le b - c$, so $s + c \le b$. Therefore
  $s + c = \sup (A + c)$.
\end{proof}

\begin{lemma}
  Assume $s \in \R$ is an upper bound for a set
  $A \subseteq \R$. Then $s = \sup A$ if and only
  if for every $\epsilon > 0$, there exists
  $a \in A$ such that  $s - \epsilon < a$.
\end{lemma}

\begin{proof}\leavevmode\\
$(\Longrightarrow)$: Suppose $\sup A = s$. Then given any
$\epsilon > 0$,  $s - \epsilon$ cannot be an upper
bound for $A$. So there exists $a \in A$ such that
$a > s - \epsilon$.

$(\Longleftarrow)$: Let $b$ be an arbitrary upper bound
for $A$. Suppose for contradiction that $b < s$. Set
$\epsilon = s - b > 0$. Then by assumption we can
find $a \in A$ such that $a > s - \epsilon = b$.
Contradiction. Therefore $b \ge s$, whence $\sup A = s$.
\end{proof}

\section{Consequences of Completeness}
\subsection{1st Consequence: Nested Interval Properties}
\begin{theorem}[Nested interval properties]
  For any $n \in \N$, assume that we are given a closed
  interval
  \[
    I_n = [a_n, b_n] = \{x \in \R : a_n \le x \le b_n\}
  .\]
  Assume $I_n \supseteq I_{n+1}$. Then the resulting
  nested sequence of closed intervals
  \[I_1 \supseteq I_2 \supseteq I_3 \supseteq \dots\]
  has a nonempty intersection:
  \[\bigcap_{n=1}^\infty I_n \ne \emptyset.\]
\end{theorem}

\begin{proof}
  Define $A = \{a_n\}$. Note that  $A \ne \emptyset$.
  For any $n$, $a_n \le b_n \le b_1$. So $x = \sup A$
  exists. Furthermore, for any $n$, $b_n$ is an upper
  bound for $A$. So  $x \le b_n$. Since  $x = \sup A$,
  $a_n \le x$. So $x \in [a_n, b_n]$ for any $n$,
  whence
  \[x \in \bigcap_{n = 1}^\infty I_n.\]
\end{proof}

\subsection{2nd Consequence: Archimedean Properties}
\begin{theorem}[Archimedean properties]\leavevmode
  \begin{enumerate}
    \item Given any $x \in \R$, there is an $n \in \N$
      such that $n > x$.
    \footnote{This is saying that $\N$ is not bounded
      above.}
    \item Given any real number $y > 0$, there is an
      $\N$ such that  $\frac{1}{n} < y$.
  \end{enumerate}
\end{theorem}

\begin{proof}[Proof of (1)]
  Argue by contradiction. Suppose $\N$ is bounded above.
  Then by the axiom of completeness, $\alpha = \sup N$
  exists. By construction, $\alpha - 1$ is not an
  upper bound for  $\N$. So we can find  $n \in N$
  such that  $\alpha - 1 < n$, which implies
  $\alpha < n + 1 \in \N$. Contradiction.
\end{proof}

\begin{proof}[Proof of (2)]
  Follows from (1) by setting $x = \frac{1}{y}$.
\end{proof}
