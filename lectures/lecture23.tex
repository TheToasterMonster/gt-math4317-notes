\chapter{Nov.~14 -- Review for Exam 2}

\section{Problems from Homework 9}
\begin{exercise}[Abbott 6.2.1]
  Let
  \[
  f_n(x) = \frac{nx}{1 + nx^2}
  .\]
  \begin{enumerate}[(a)]
    \item Find the pointwise limit of $f_n$.
    \item Is the convergence uniform on $(0, \infty)$?
    \item Is the convergence uniform on $(0, 1)$?
    \item Is the convergence uniform on $(1, \infty)$?
  \end{enumerate}
\end{exercise}

\begin{proof}[Solution]
  (a)\, We have
  \[
    f_n(x) = \frac{x}{\frac{1}{n} + x^2} \to \frac{x}{x^2}
    = \frac{1}{x} = f(x).
  \]
  Also observe that
  \[
    |f_n(x) - f(x)|
    = \left|\frac{x}{\frac{1}{n} + x^2} - \frac{x}{x^2}\right|
    = \left|\frac{x(-\frac{1}{n})}{x^2 (\frac{1}{x} + x^2)}\right|
    = \frac{1}{n} \frac{1}{x(\frac{1}{n} + x^2)}.
  \]
  (c)\, No, the convergence is problematic near $x = 0$. We
  can see that
  \[
    \frac{1}{n}\frac{1}{x(1 + nx^2)} < \epsilon
    \iff \frac{x}{1 + nx^2} < \epsilon
    \iff 1 + nx^2 > \frac{1}{x\epsilon}
    \iff nx^2 > \frac{1}{x\epsilon} - 1
    \iff n > \frac{1}{x^2}\left(\frac{1}{x\epsilon} - 1\right).
  \]
  Since this is true for all $x \in (0, 1)$, we have
  $n$ arbitrarily large, which is not allowed.

  (d)\, Notice that
  \[
    \frac{1}{n} \frac{1}{x(\frac{1}{n} + x^2)}
    \le \frac{1}{n}
  \]
  when $x \ge 1$. So taking $n > 1 / \epsilon$
  works.

  (b)\, No, since $(0, 1) \subseteq (0, \infty)$ and
  the convergence is not uniform on $(0, 1)$ by
  (c).
\end{proof}

\begin{exercise}[Abbott 6.2.2]
  Let
  \[
    f_n(x) = \begin{cases}
      1 & \text{if } x = 1, \frac{1}{2}, \dots, \frac{1}{n} \\
      0 & \text{otherwise}
    \end{cases}
  \]
  and let $f$ be the pointwise limit of $f_n$.
  \begin{enumerate}[(a)]
    \item Is $f_n$ is continuous at $0$?
    \item Does $f_n \to f$ uniformly?
    \item Is $f$ continuous at $0$?
  \end{enumerate}
\end{exercise}

\begin{proof}[Solution]
  (a)\, Note that $f_n(0) = 0$. For any
  $\epsilon > 0$, pick $\delta < \frac{1}{n}$.
  Then
  \[
    |f_n(x) - f(0)| = |0 - 0| = 0 < \epsilon.
  \]
  So $f_n$ is continuous at $x = 0$.

  (c)\, Observe that
  \[
    f = \begin{cases}
      x & \text{if } x = \frac{1}{q} \text{ for some } q \in \N \\
      0 & \text{otherwise}.
    \end{cases}
  \]
  Take $\epsilon_0 = 1 / 2$. For all $\delta > 0$,
  there exists $q_0 \in \N$ such that
  $1 / q_0 \in (0, \delta)$. Then
  \[
    \left|f(1 / q_0) - f(0)\right| = 1 > \frac{1}{2}.
  \]
  So $f$ is not continuous at $0$.

  (b)\, Since $f$ is not continuous despite
  $f_n$ being continuous, the convergence cannot be
  uniform.
\end{proof}

\section{Weierstrass \texorpdfstring{$M$}{M}-test}

\begin{corollary}[Weierstrass $M$-test]
  Let $f_n$ be a function defined on $A \subseteq \R$
  for each $n \in \N$ and suppose that
  $|f_n(x)| \le M_n$ for all $x \in A$.
  If $\sum_{n = 0}^\infty M_n < \infty$, then
  $\sum_{n = 0}^\infty f_n$ converges uniformly.
\end{corollary}

\begin{proof}
  Apply the Cauchy criterion twice, once on $\sum M_n$
  and again on $\sum f_n$.
\end{proof}

\section{More Problems}

\begin{exercise}
  True of false?
  \begin{enumerate}[(a)]
    \item We have $f^{-1}(B)$ is finite whenever $B$ is finite.
    \item We have $f^{-1}(K)$ is compact if
      $K$ is compact.
    \item We have $f^{-1}(A)$ is bounded if $A$ is bounded.
    \item If $F$ is closed, then $f^{-1}(F)$ is closed.
  \end{enumerate}
\end{exercise}

\begin{proof}[Solution]
  (a)\, Very false. Just let $f \equiv 0$.

  (b)\, Still very false. Use the same $f \equiv 0$.

  (c)\, Still false, use $f \equiv 0$ again.

  (d)\, False. Define $f \equiv 0$ on $(0, 1)$ and
  $f \equiv 1$ on $\R \setminus (0, 1)$.
  Take $F = \{0\}$ with $f^{-1}(F) = (0, 1)$.
\end{proof}
