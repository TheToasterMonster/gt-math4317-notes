\chapter{Aug.~29 -- Completeness and Countability}

\section{Consequences of Completeness}

\subsection{3rd Consequence: Density of \texorpdfstring{$\Q$}{Q} in \texorpdfstring{$\R$}{R}}
\begin{theorem}[Density of $\Q$ in $\R$]
For all $a, b \in \R$, $a < b$, there exists
$r \in \Q$ such that $a < r < b$.
\end{theorem}

\begin{proof}
  We want to find $m \in \Z$, $n \in \N$ such that
  \[
  a < \frac{m}{n} < b
  .\]
  By (ii) of the Archimedean properties, we can find
  $n \in \N$ such that
  \[
  \frac{1}{n} < b - a
  .\]
  Fix such an $n$. Then let $m$ be the smallest integer
  such that $m - 1 \le na < m$. By construction,
  \[
  \frac{m}{n} - \frac{1}{n} \le a < \frac{m}{n},
  \]
  \[\frac{m}{n} \le a + \frac{1}{n} < b.\]
  Therefore, $a < \frac{m}{n} < b$.
\end{proof}

\begin{corollary}
  For all $a, b \in \Q$, $a < b$, there exists
   $t \in \R \setminus \Q$ such that  $a < t < b$.
\end{corollary}

\subsection{4th Consequence: Existence of \texorpdfstring{$\sqrt{2}$}{sqrt(2)}}
\begin{theorem}[Existence of $\sqrt{2}$]
  There exists $s \in \R$, $s > 0$ such that $s^2 = 2$.
\end{theorem}

\begin{proof}
  Define
  \[S = \{x > 0 : x^2 < 2\} \subseteq \R.\]
  $x = 1 \in S$, so $S \ne \emptyset$. $2$ is an upper
  bound for $S$, so $S$ is bounded above. Then by the
  axiom of completeness, $s = \sup S$ exists.
  We claim that $s^2 = 2$. 

  Suppose otherwise that $s^2 < 2$. Then we can find
  $\epsilon > 0$ such that $s + \epsilon \in S$.
  Define $\delta = 2 - s^2 > 0$. Note that
  \[(s + \epsilon)^2 - 2 = s^2 + 2s\epsilon + \epsilon^2 - 2 = -\delta + 2s\epsilon + \epsilon^2.\]
  We know $s \le 2$ since $2$ is an upper bound. 
  Pick
  \[\epsilon = \frac{\delta}{100000000000},\]
  \[
  2s\epsilon + \epsilon \le 4\epsilon + \epsilon^2 < \frac{\delta}{2}
  .\]
  Then
  \[
  (s + \epsilon)^2 - 2 < -\delta + \frac{\delta}{2}
  = -\frac{\delta}{2} < 0
  .\]
  So $s + \epsilon \in S$, which contradicts with
  $s = \sup S$.

  $s^2 > 2$ also leads to a contradiction
  (left as an exercise). Thus we must have $s^2 = 2$.
\end{proof}

\section{Countability}
\begin{definition}
  We say two sets $A$ and $B$ have the same
  \textbf{cardinality}
  if there is a bijection $f : A \to B$.
  We write $A \sim B$.
\end{definition}

\begin{definition}
  We say that a set $A$  is \textbf{finite} if
  $A \sim \{1, 2, \dots, n\}$ for some integer $n$.
  We say that a set $A$ is \textbf{countable}
  (or countably infinite) if
  $A \sim \N$.
  If a set $A$ is not countable, then we say it is
  \textbf{uncountable}.
\end{definition}

\begin{tcolorbox}
  $E = \{2, 4, 6, 8, \dots\}$.

  $E$ is not finite but it is countable: $E \sim \N$.
  We can define $f : \N \to E$ by $f(n) = 2n$.
\end{tcolorbox}

\begin{tcolorbox}
  $\N \sim \Z$. 

  The bijection $f : \N \to \Z$ is given by
  \[
  f(n) =
  \begin{cases}
    \frac{n-1}{2} & \text{$n$ is odd} \\
    -\frac{n}{2} & \text{$n$ is even}.
  \end{cases}
  \]
\end{tcolorbox}

\begin{tcolorbox}
  $(-1, 1) \sim \R$.

  The bijection  $f : (-1, 1) \to \R$ is given by
  \[x \mapsto \frac{x}{x^2 - 1}.\]
\end{tcolorbox}

\begin{theorem}\leavevmode
  \begin{enumerate}
    \item $\Q$ is countable.
    \item  $\R$ is uncountable.
  \end{enumerate}
\end{theorem}

\begin{proof}[Proof of (1)]
  Set $A_1 = \{0\}$ and for $n \ge 2$,
  \[A_n = \left\{\pm \frac{p}{q} : p, q \in \N,\, \text{$p, q$ in lowest terms},\, p + q = n\right\}.\]
  So the first few $A_n$ are:
  \[A_2 = \left\{\frac{1}{1}, \frac{-1}{1}\right\},\]
  \[A_3 = \left\{\frac{1}{2}, \frac{2}{1}, \frac{-1}{2}, \frac{-2}{1}\right\},\]
  etc. Note that $A_n$ is finite and
  for all
  $x \in \Q$, there is an $n \in \N$ such that
  $x \in A_n$.
  We can list elements in $A_1, \dots, A_n$ and label
  them with
  integers in $\N$. Any element of $A_n$ will be
  listed eventually. Then this pairing gives
  a bijection since the $A_n$ are disjoint.
  So $\Q \sim \N$.
\end{proof}

\begin{proof}[Proof of (2)]
  Argue by contradiction. Suppose $f$ is one-to-one from
  $\N \to \R$.
  Set $x_1 = f(1)$, $x_2 = f(2)$, etc. We can write
  \[\R = \{x_1, x_2, \dots\}.\]
  Let $I_1$ be a closed interval such that
  $x_1 \notin I_1$. Pick $I_2 \subseteq I_1$ such that
  $x_2 \notin I_2$. Continue this process such that
  $I_{n+1} \subseteq I_n$ is a closed interval where
  $x_{n+1} \notin I_{n+1}$.
  By construction,
  \[I_1 \supseteq I_2 \supseteq \dots \supseteq I_n \supseteq \dots.\]
  We know that
  \[\bigcap_{n = 1}^\infty I_n \ne \emptyset.\]
  So we can find $n_0$ such that
  \[x_{n_0} \in \bigcap_{n = 1}^\infty I_n.\]
  This is a contradiction with $x_{n_0} \notin I_{n_0}$.
  Thus such an $f$ cannot exist and $\R$ is
  uncountable.
\end{proof}

\begin{theorem}\leavevmode
  \begin{enumerate}
    \item Let $A \subseteq B$. If $B$ is countable, then
      $A$ is either finite or countable.
    \item If $A_n$ is a countable set, then
      \[\bigcup_{n=1}^\infty A_n\]
      is also countable.
  \end{enumerate}
\end{theorem}

\begin{theorem}[Cantor's theorem]
  The open interval
  \[(0, 1) = \{x \in \R : 0 < x < 1\}\]
  is uncountable.
\end{theorem}

\begin{proof}
  Argue by contradiction. Assume $f : \N \to (0, 1)$
  is one-to-one and onto. Then for $m \in \N$,
  we can write (decimal expansion)
  \[f(m) = 0.a_{m1}a_{m2}a_{m3}\ldots \in (0, 1).\]
  For every $m, n \in \N$, $a_{mn} \in \{0, \dots, 9\}$
  is the $n$th digit in the decimal expansion of
  $f(m)$. We can write in a table
  \[1 \quad f(1) \quad a_{11} \quad a_{12} \quad a_{13} \quad \dots\]
  \[2 \quad f(2) \quad a_{21} \quad a_{22} \quad a_{23} \quad \dots\]
  \[3 \quad f(3) \quad a_{31} \quad a_{32} \quad a_{33} \quad \dots\]
  \[\vdots\]
  Take $x = 0.b_1b_2b_3\ldots$ where
  \[
    b_n =
    \begin{cases}
      2 & \text{if $a_{nn} \ne 2$} \\
      3 & \text{if $a_{nn} = 2$}.
    \end{cases}
  \]
  Then $x \ne f(m)$ for any $m \in \N$
  (since $b_m \ne a_{mm}$). This is a
  contradiction.
\end{proof}
