\chapter{Sept.~26 -- Closed Sets}

\section{Closed Sets}

\begin{definition}
  Let $A \subseteq \R$. An element $x \in A$ is
  an \textbf{isolated point} of $A$ if it is not
  a limit point of $A$, i.e.~there exists $\epsilon > 0$
  such that $V_\epsilon(x) \cap A = \{x\}$.
\end{definition}

\begin{definition}
  A set $A \subseteq \R$ is \textbf{closed}
  if it contains all of its limit points.
\end{definition}

\begin{example}
  The empty set and $\R$ are closed. Moreover, any
  set without limit points is closed.
\end{example}

\begin{theorem}
  A set $A \subseteq \R$ is closed if and only if
  every Cauchy sequence in $A$ converges to a
  limit in $A$.
\end{theorem}

\begin{proof}
  $(\Rightarrow)$\, Suppose $A$ is closed and
  $\{a_n\}$ is Cauchy with $a_n \in A$ for all $n$.
  Since Cauchy sequences are convergent, let
  $x = \lim_{n \to \infty} a_n$. Now consider two cases.
  If there exists an $n$ such that $x = a_n$, then we're
  done since $x = a_n \in A$. Otherwise, we have
  $a_n \ne x$ for all $n$. By Theorem \ref{thm:limit-point},
  $x$ is a limit point of $A$. So $x \in A$ as
  $A$ is closed.

  $(\Leftarrow)$\, Let $x$ be a limit point of $A$.
  Then there exists a sequence $\{a_n\}$ with $a_n \in A$
  and $a_n \ne x$ for all $n$ such that
  $\lim_{n \to \infty} a_n = x$. This means that
  $\{a_n\}$ is Cauchy, so by assumption,
  $x = \lim_{n \to \infty} a_n \in A$. Thus every limit
  point of $A$ is in $A$, so $A$ is closed.
\end{proof}

\begin{example}
  Consider the set
  \[
    A = \left\{\frac{1}{n} : n \in \N\right\}
  .\]
  First $x \ne 0$, we look at the following cases:
  \begin{enumerate}
    \item If $x < 0$, let $\epsilon = |x|$. Then
    $V_\epsilon(x) = (2x, 0)$, and
    $V_\epsilon(x) \cap A = \varnothing$ since
    $A \subseteq \R^+$. So $x < 0$ is not
    is not a limit point of $A$.
  \item If $x > 1$, let $\epsilon = x - 1$. Then
    $V_\epsilon(x) = (1, 2x - 1)$, so
    $V_\epsilon(x) \cap A = \varnothing$ as all $y \in A$
    satisfies $y \in (0, 1]$.
  \item If $x \in (0, 1]$, then there exists $n \in N$
    such that $n > 1 / x$. Let
    $n_0 = \min\{n \in \N : n > 1 / x\}$, which exists
    by the well-ordering principle. Noting that
    $n_0 \ge 2$, we have
    \[
    \frac{1}{n_0} < x \le \frac{1}{n_0 - 1}
    .\]
    Now we look at two more cases:
    \begin{enumerate}
      \item If $x = \frac{1}{n_0 - 1}$, let
        \[\epsilon = x - \frac{1}{n_0} = \frac{1}{n_0 - 1} - \frac{1}{n_0}.\]
        Then we have
        \[
        V_\epsilon(x) = \left(\frac{1}{n_0}, \frac{2}{n_0 - 1} - \frac{1}{n_0}\right)
        .\]
        Note that
        $(V_\epsilon \cap A) \setminus \{x\} = \varnothing$ if $n_0 = 2$.
        Otherwise, $n_0 > 2$ and we have
        \[\frac{2}{n_0 - 1} - \frac{1}{n_0} - \frac{1}{n_0 - 2} = \dots = \frac{-2}{n_0(n_0 - 2)(n_0 - 1)} < 0.\]
        So $V_\epsilon(x) \subseteq \left(\frac{1}{n_0}, \frac{1}{n_0} - 2\right)$, which means that
        $V_\epsilon(x) \cap A = \{x\}$.
      \item Otherwise, $x \in \left(\frac{1}{n_0}, \frac{1}{n_0 - 1}\right)$
        and let
        \[\epsilon = \min\left\{x - \frac{1}{n_0}, \frac{1}{n_0 - 1} - x\right\}.\]
        Then (left as exercise)
        $V_\epsilon(x) \subseteq \left(\frac{1}{n_0}, \frac{1}{n_0 - 1}\right)$,
        which implies $V_\epsilon(x) \cap A = \varnothing$.
    \end{enumerate}
  \end{enumerate}
  So $x \ne 0$ is not a limit point of $A$. However,
  $x = 0 = \lim_{n \to \infty} \frac{1}{n}$, so $0$ is
  a limit point of $A$. But $0 \notin A$, so $A$ is
  not closed.
\end{example}

\begin{example}
  Let $A = [a, b]$. For any Cauchy sequence
  $\{x_n\} \subseteq A$, let
  $x = \lim_{n \to \infty} x_n$. Since
  $x_n \ge a$, we have $x = \lim_{n \to \infty} x_n \ge a$.
  Similarly, $x_n \le b$ implies that $x \le b$. So
  $x \in [a, b] = A$, and thus $A$ is closed.
\end{example}

\begin{example}
  Consider $\Q$. For any $x \in \R$, for all $n \in N$
  there exists $a_n$ such that $a_n \in \Q$ with
  \[
  \frac{1}{2n} < |a_n - x| < \frac{1}{n}
  .\]
  Thus $a_n \ne x$ and $a_n \in \Q$ for all $n$, so
  $\lim_{n \to \infty} a_n = x$ is a limit point of
  $\Q$. So $\Q$ is not closed.
\end{example}

\begin{remark}
  We can also define the real numbers as equivalence
  classes of Cauchy sequences.
  \footnote{Using \textit{Dedekind cuts} is another such way.}
  Note that Cauchy sequences
  do not require an ordering (only a metric), so we can
  easily extend this definition to higher dimensions.
\end{remark}

\section{The Closure of a Set}

\begin{definition}
  Let $A \subseteq \R$. Define the \textbf{closure}
  of $A$ as
  \[
    \overline{A} = \{x : x \in A \text{ or $x$ is a limit point of $A$}\}
  .\]
\end{definition}

\begin{theorem}
  The closure of a set $A$ is closed. Furthermore,
  if $B$ is closed and $A \subseteq B$, then
  $\overline{A} \subseteq B$.
  \footnote{So $\overline{A}$ is the smallest closed set
    containing $A$.}
\end{theorem}

\begin{proof}
  Let $x$ be a limit point of $\overline{A}$.
  We want to show that $x$ is also a limit point of $A$
  (so we will have $x \in \overline{A}$). If $x \in A$,
  then we're done. Otherwise, $x \notin A$, so for
  all $\epsilon > 0$, we have
  $(V_{\epsilon / 2}(x) \cap \overline{A}) \setminus \{x\} \ne \varnothing$,
  so let $y \in (V_{\epsilon / 2}(x) \cap \overline{A}) \setminus \{x\}$.
  If $y \in A$, then $(V_{\epsilon}(x) \cap A) \setminus \{x\}
  = \varnothing$ and we're done. Otherwise, $y \notin A$,
  so $y \in \overline{A}$ implies that $y$ is a limit
  point of $A$. So there exists
  $z \in (V_{\epsilon / 2} \cap A) \setminus \{y\}$.
  Then
  \[|x - z| \le |x - y| + |y - z| < \frac{\epsilon}{2} + \frac{\epsilon}{2} = \epsilon.\]
  So $z \in (V_\epsilon(x) \cap A) \setminus \{x\}$.
  Since this is true for all $\epsilon > 0$,
  $x$ is a limit point of $A$.
  Thus $x \in \overline{A}$, so $\overline{A}$ is closed.

  For the second part, let $x \in \overline{A}$. If
  $x \in A$, then $A \subseteq B$ implies that $x \in B$.
  If $x \notin A$, then $x$ is a limit point of $A$.
  So there exists a sequence $\{a_n\}$ with $a_n \in A$
  for all $n$ such that $a_n \to x$.
  Since $a_n \in A \subseteq B$, we have $a_n \in B$ for
  all $n$. Since $a_n \to x$, we must have
  $x \in B$ since $B$ is closed. Thus
  $A \subseteq B$.
\end{proof}

\begin{corollary}
  If $A \subseteq B$, then $\overline{A} \subseteq \overline{B}$.
\end{corollary}

\begin{proof}
  Note that
  Cauchy sequences in $A$ are also Cauchy sequences in $B$.
\end{proof}
