\chapter{Sept.~5 -- Limits and Limit Theorems}

\section{Review of Limits}

\begin{example}
  Find
  \[
    \lim_{n \to \infty} \frac{1 + \sqrt{n}}{\sqrt{n}}
  .\]
\end{example}

\begin{proof}
  We want to show that
  \[
    \lim_{n \to \infty} \frac{1 + \sqrt{n}}{\sqrt{n}} = 1
  .\]
  Fix $\epsilon > 0$ and take $N \in \N$ such that
  $N > \frac{1}{\epsilon^2}$. Then for any $n > N$,
  \[
    \left\lvert \frac{1 + \sqrt{n}}{\sqrt{n}} - 1 \right\rvert \le
    \left\lvert \frac{1}{\sqrt{n}} \right\rvert
    \le \frac{1}{\sqrt{N}} < \epsilon
  ,\]
  as desired.
\end{proof}

How can we understand this using the topological definition?
For all $\epsilon > 0$, take $V_\epsilon(1)$. Pick
$N > \frac{1}{\epsilon^2}$. Then we claim that
$V_\epsilon(1)$ contains all but at most $N$ elements of
$\left\{\frac{\sqrt{n} + 1}{\sqrt{n}}\right\}$. When
$n \ge N$, we have
\[
  \left\lvert \frac{\sqrt{n} + 1}{\sqrt{n}} - 1 \right\rvert
  < \epsilon
,\]
i.e.~$\frac{\sqrt{n} + 1}{\sqrt{n}} \in V_\epsilon(1)$. So
at most $N$ elements might not be in $V_\epsilon(1)$.

\section{Limit Theorems}
\subsection{Algebraic Facts About Limits}
\begin{definition}
  A sequence $\{x_n\}$ is said to be \textbf{bounded} if there
  exists $M$ such that $|x_n| \le M$ for all $n$.
  Alternatively, $\sup_n |x_n| \le M$.
\end{definition}

\begin{theorem}
  Every convergent sequence is bounded.
\end{theorem}

\begin{proof}
  Suppose
  \[
    \lim_{n \to \infty} x_n = l
  .\]
  Take $\epsilon = 1$, we can find $N$ such that for all
  $n \ge N$, $|x_n - l| < 1$. By the triangle inequality,
  $|x_n| < |l| + 1$ for $n \ge N$. Take
  \[M = \max\{|x_1|, |x_2|, \dots, |x_{N - 1}|, |l| + 1\}.\]
  Then $|x_n| \le M$ for all $n \in \N$.
\end{proof}

\begin{theorem}[Algebraic limit theorem]
  If
  \[
    \lim_{n \to \infty} a_n = a \quad \text{and} \quad
    \lim_{n \to \infty} b_n = b
  ,\]
  then for all $c \in \R$,
  \[
    (1)\, \lim_{n \to \infty} ca_n = ca, \quad
    (2)\, \lim_{n \to \infty} (a_n + b_n) = a + b, \quad \text{and} \quad
    (3)\, \lim_{n \to \infty} a_nb_n = ab
  .\]
  Furthermore, if $b \ne 0$, then
  \[
    \lim_{n \to \infty} \frac{a_n}{b_n} = \frac{a}{b} \tag{4}
  .\]
\end{theorem}

\begin{proof}
  (1)\, When $c = 0$, the result is trivial. When $c \ne 0$, for
  all $\epsilon > 0$, we set
  $\epsilon' = \frac{\epsilon}{|c|}$.
  Since $\lim_{n \to \infty} a_n = a$, we can find
  $N_{\epsilon'}$ such that for all $n \ge N_{\epsilon'}$,
  $|a_n - a| < \epsilon'$.
  When $n > N_{\epsilon'}$, we have
  \[
  |ca_n - ca| = |c||a_n - a| < |c|e' = |c| \frac{\epsilon}{|c|} = \epsilon
  .\]
  So $\lim_{n \to \infty} ca_n = ca$.

  (2)\, For all $\epsilon > 0$, since $a_n \to a$ and $b_n \to b$,
  we can find $N_1$ and $N_2$ such that when
  \begin{align*}
    n \ge N_1, &\quad |a_n - a| < \frac{\epsilon}{2}, \\
    n \ge N_2, &\quad |b_n - b| < \frac{\epsilon}{2}.
  \end{align*}
  Take $N = \max\{N_1, N_2\}$. Then for all $n \ge N$,
   \[
  |a_n + b_n - (a + b)| = |a_n - a + b_n - b| \le
  |a_n - a| + |b_n - b| < \frac{\epsilon}{2} + \frac{\epsilon}{2}
  = \epsilon
  .\]
  Therefore $\lim_{n \to \infty} (a_n + b_n) = a + b$.
\end{proof}

\subsection{Order Theorem}
\begin{theorem}[Order theorem]
  Let $\{a_n\}$ and $\{b_n\}$ be sequences such that
  \[
    \lim_{n \to \infty} a_n = a \quad \text{and} \quad
    \lim_{n \to \infty} b_n = b
  .\]
  (5)\, If $a_n \ge 0$ for every $n$, then $a \ge 0$.
  (6)\, If $a_n \le b_n$, then $a \le b$.
  (7)\, If $a_n \ge c$, then $a \ge c$.
\end{theorem}

\begin{proof}
  (5)\, Argue by contradiction. Suppose $a < 0$. Take
  $\epsilon = \frac{|a|}{2}$. Since
  $\lim_{n \to \infty} a_n = a$, we can find $N$ such
  that when $n \ge N$, $|a_n - a| < \epsilon$. Note that
  this means
  \[-\epsilon < a_n - a < \epsilon \]
  Then we have
  \[a_n < \epsilon + a = \frac{-a}{2} + a = \frac{a}{2} < 0.\]
  Contradiction.
\end{proof}

\subsection{Monotone Convergence Theorem}

\begin{definition}
  A sequence $\{a_n\}$ is \textbf{increasing} if
  $a_n \le a_{n + 1}$ for every $n$ and \textbf{decreasing}
  if $a_n \ge a_{n + 1}$ for every $n$. A sequence is
  \textbf{monotone} if it is either increasing or decreasing.
\end{definition}

\begin{theorem}[Monotone convergence theorem]
  If a sequence is monotone and bounded, then it converges.
\end{theorem}

\begin{proof}
  Let $\{a_n\}$ be increasing and bounded. Set
  $A = \{a_n : n \in \N\}$.
  Note that $A \ne \emptyset$ and $A$ is bounded. Therefore, by the
  axiom of completeness, $s = \sup A \in \R$ exists.
  Then we claim that $\lim_{n \to \infty} a_n = s$.
  For
  every $\epsilon > 0$, $s - \epsilon$ is not an upper bound
  for $A$, so we can find $N$ such that
  $s - \epsilon < a_{N} \le s$. 
  Since $\{a_n\}$ is increasing, for all $n \ge N$, we know
  $s - \epsilon < a_N \le a_n \le s$,
  i.e.~$|a_n - s| < \epsilon$. Therefore
  $\lim_{n \to \infty} a_n = s$.

  For $\{a_n\}$ decreasing and bounded, simply let apply the
  previous result to $\{-a_n\}$.
\end{proof}
