\chapter{Oct.~17 -- Functional Limits}

\section{The Limit of a Function}
\begin{definition}
  Let $A \subseteq \R$ and $f : A \to \R$. Further let
  $c$ be a limit point of $A$ (not necessarily in $A$).
  We say that
  \[\lim_{x \to c} f(x) = L\]
  provided that for all $\epsilon > 0$, there exists
  $\delta > 0$ such that for all $x \in A$, if
  $0 < |x - c| < \delta$, then
  $|f(x) - L| < \epsilon$.
\end{definition}

\begin{remark}
  We do not require $f$ to be defined at $c$.
\end{remark}

\begin{example}
  For the function $f : \R \to \R$ given by
  \[
    f = \begin{cases}
      1 & \text{if $x \ne 0$} \\
      5 & \text{if $x = 0$},
    \end{cases}
  \]
  we have $\lim_{x \to 0} f(x) = 1$.
\end{example}

\begin{example}
  Take $A = (-\infty, 0) \cup (0, \infty)$ and $f = 1$
  in $A$. Then $\lim_{x \to 0} f(x) = 1$.
\end{example}

We can alternatively use the following topological
definition:
\begin{definition}
  We say that
  \[\lim_{x \to c}f(x) = L\]
  if for all $\epsilon > 0$, there exists $\delta > 0$
  such that if $x \in A$ with $x \ne c$ and
  $x \in V_\delta(c)$, then $f(x) = V_\epsilon(L)$.
\end{definition}

\begin{exercise}
  Let $f(x) = 3x + 1$. Then $\lim_{x \to 2} f(x) = 7$.
\end{exercise}

\begin{proof}
  Fix $\epsilon > 0$ and let $\delta = \frac{\epsilon}{3}$.
  Then for all $0 < |x - 2| < \delta$, we have
  \[|f(x) - 7| = |3x + 1 - 7| = |3x - 6| = 3|x - 2|
  < 3\delta = \epsilon.\]
  Therefore $\lim_{x \to 2} f(x) = 7$.
\end{proof}

\begin{exercise}
  Let $f(x) = x^2$. Then $\lim_{x \to 2} f(x) = 4$.
\end{exercise}

\begin{proof}
  Note that
  \[|f(x) - 4| = |x^2 - 4| = |x - 2| |x + 2|.\]
  Since we only care about $f(x)$ when $x$ is near 2,
  we take $|x - 2| \le 1$. In this region,
  $|x + 2| \le 5$.

  Now fix $\epsilon > 0$ and take
  $\delta_1 = \frac{\epsilon}{5}$. Then in the region
  $|x - 2| \le 1$, for any $0 < |x - 2| < \delta_1$, we have
  \[|f(x) - 4| = |x + 2||x - 2| < 5 \delta_1 = 5 \cdot \frac{\epsilon}{5} = \epsilon.\]
  Taking $\delta = \min\{1, \delta_1\}$ does the job
  for any arbitrary $0 < |x - 2| < \delta$. So
  $\lim_{x \to 2} f(x) = 4$.
\end{proof}

\begin{exercise}
  Let $f(x) = x^3$. Then $\lim_{x \to 2} f(x) = 8$.
\end{exercise}

\begin{proof}
  Note that
  \[|x^3 - 8| = |x - 2| |x^2 + 2x + 4|.\]
  First restrict $|x - 2| < 1$. Then
  \[|x^2 + 2x + 4| \le 9 + 6 + 4 = 19\]
  in this region. So take
  $\delta_1 = \frac{\epsilon}{19}$ and
  $\delta = \min\{1, \delta_1\}$.
\end{proof}

\section{Algebraic Properties of Functional Limits}
\begin{theorem}
  Let $f : A \to \R$ and $c$ be a limit point of $A$.
  Then the following two statements are equivalent:
  \begin{enumerate}
    \item $\lim_{x \to c} f(x) = L$.
    \item For any sequence $\{x_n\} \subseteq A$ such that
      $x_n \ne c$ and
      $\lim_{n \to \infty} x_n = c$, we have
      $\lim_{n \to \infty} f(x_n) = L$.
  \end{enumerate}
\end{theorem}

\begin{proof}
  Next class.
\end{proof}
