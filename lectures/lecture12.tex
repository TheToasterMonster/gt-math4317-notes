\chapter{Sept.~28 -- Compact Sets}

\section{Another Characterization of Closed Sets}
\begin{definition}
  Given a set $A \subseteq \R$, its \textbf{complement} is
  $A^c = \{x \in \R : x \notin A\}$.
\end{definition}

\begin{theorem}
  A set $A \subseteq \R$ is closed if and only if $A^c$
  is open.
\end{theorem}

\begin{proof}
  $(\Rightarrow)$\, Suppose $A$ is closed. Take any $x \in A^c$.
  Since $A$ is closed and $x \notin A$, we know $x$ is not a
  limit point of $A$. So there is $\epsilon > 0$ such that
  $(V_\epsilon(x) \cap A) \setminus \{x\} = \varnothing$.
  Since $x \notin A$, this means that
  $V_\epsilon(x) \cap A = \varnothing$, which means that
  $V_\epsilon(x) \subseteq A^c$. So $x$ is an interior point
  of $A^c$, and thus $A^c$ is open.

  $(\Leftarrow)$\, Suppose $A^c$ is open. Let $x$ be a
  limit point of $A$. Assume that $x \notin A$, i.e.~$x \in A^c$.
  Since $A^c$ is open, there is $\epsilon > 0$ such that
  $V_\epsilon(x) \subseteq A^c$. But then
  $V_\epsilon(x) \cap A = \varnothing$, which is a contradiction
  with $x$ being a limit point of $A$. So $x \in A$, and
  thus $A$ is closed.
\end{proof}

\begin{corollary}
  A set $A \subseteq \R$ is open if and only if $A^c$
  is closed.
\end{corollary}

\begin{remark}
  This is used in topology, where a collection of open sets
  that satisfies certain conditions
  \footnote{The collection must be closed under finite intersection and arbitrary union.}
  is called a \textit{topology}
  of a space, and closed sets are defined as their complements.
  Furthermore, metrics are not necessary in this setting.
\end{remark}

\begin{theorem}\leavevmode
  \begin{enumerate}
    \item Let $A_1, A_2, \dots, A_n \subseteq \R$ be closed.
      Then $\bigcup_{i = 1}^n A_i$ is closed.
    \item Let $A_\lambda \subseteq \R$, $\lambda \in \Lambda$
      be a family of closed subsets of $\R$ indexed by
      $\lambda \in \Lambda$, where $\Lambda$ is an index
      set. Then $\bigcup_{\lambda \in \Lambda} A_\lambda$
      is closed.
  \end{enumerate}
\end{theorem}

\begin{proof}
  Left as an exercise.
\end{proof}

\section{Compactness}

\begin{definition}
  A set $A \subseteq \R$ is \textbf{compact} if any sequence
  $\{a_n\}$ in $A$ has a convergent subsequence
  $\{a_{n_k}\}$ such that $\lim_{k \to \infty} a_{n_k} \in A$.
  \footnote{This definition is sometimes called \textit{sequential compactness}.}
\end{definition}

\begin{remark}
  Suppose we want to solve the differential equation
  \[
  \begin{cases}
    x' = f(t, x) \\
    x(t_0) = x_0
  \end{cases}
  .\]
  We can first transform this into the integral equation
  \[
    x(t) = x_0 + \int_{t_0}^t f(s, x(s))\, ds
  \]
  with a test function $f$.
  Then we perform Picard iterations to continue. However, this
  method requires $f \in C^{1}(\R)$ (i.e.~$f$ is continuously
  differentiable), since that is what
  guarantees that the sequence of functions converges
  (closedness).
  But even without this condition (if $f$ is only continuous),
  if we can
  show that the set of functions lies in a compact set,
  then we can find a subsequence of functions that do converge
  (though solutions may no longer be unique).
\end{remark}

\begin{example}
  The interval $(0, 1]$ is not compact since it is not closed:
  it does not contain all of its limit points.
\end{example}

\begin{example}
  The set $\R$ is not compact since it is not bounded:
  an arbitrary sequence may not even have a convergent
  subsequence.
\end{example}

\begin{theorem}
  \label{thm:compactness}
  A set $A \subseteq \R$ is compact if and only if
  $A$ is bounded and closed.
  \footnote{This applies to all Euclidean spaces (and pretty much only Euclidean spaces).}
\end{theorem}

\begin{proof}
  $(\Rightarrow)$\, Assume $A$ is not bounded. Then for any
  $n \in \N$, there exists an $x_n \in A$ such that
  $|x_n| > M$. Then $\{x_n\}$ is a sequence in $A$.
  Since $A$ is compact, there exists a subsequence
  \[\{x_{n_k}\} = \{x_{n_1}, x_{n_2}, x_{n_3}, \dots\}\]
  such
  that $x = \lim_{k \to \infty} x_{n_k} \in A$. But this
  implies that $\{x_{n_k}\}$ is bounded, which contradicts
  the fact that $|x_{n_k}| > n_k$ and $n_k \to \infty$
  and $k \to \infty$. Hence $A$ must be bounded.

  Now assume $A$ is not closed. Then there exists a limit
  point $x$ of $A$ such that $x \notin A$. Since $x$ is
  a limit point of $A$, there exists a sequence
  $\{a_n\}$ such that $a_n \in A$ and
  $\lim_{n \to \infty} a_n = x$. Now since $A$ is compact,
  there exists a convergent subsequence $\{a_{n_k}\}$ such that
  $\lim_{k \to \infty} a_{n_k} \in A$. But since
  $\{a_n\}$ converges, we have
  \[x = \lim_{n \to \infty} a_n = \lim_{k \to \infty} a_{n_k} \in A.\]
  This is a contradiction with $x \notin A$. Hence $A$ is closed.

  $(\Leftarrow)$\, Suppose $A$ is bounded and closed. Let
  $\{a_n\}$ be a sequence in $A$. By the Bolzano-Weierstrass
  theorem, there exists a convergent subsequence $\{a_{n_k}\}$.
  Let $x = \lim_{k \to \infty} a_{n_k}$. But $\{a_n\}$ is
  a convergent sequence in $A$, which is closed. So $x \in A$,
  and hence $A$ is compact.
\end{proof}

\begin{example}
  The union of intervals $[1, 2] \cup [3, 4]$ is compact.
\end{example}

\begin{example}
  The set
  \[A = \left\{\frac{1}{n} : n \in \N\right\} \cup \{0\}\]
  is compact (since we added the limit point $0$).
\end{example}

\begin{remark}
  Why do we need the concept of compactness? Because Theorem
  \ref{thm:compactness} is no longer true in infinite dimensions.
\end{remark}

\begin{theorem}
  Suppose
  \[k_1 \supseteq k_2 \supseteq k_3 \supseteq \dots\]
  are non-empty compact sets. Then
  \[
    \bigcap_{n = 1}^\infty k_n \ne \varnothing
  .\]
\end{theorem}

\begin{proof}
  Since $k_n \ne \varnothing$ for all $n$, there exists
  $a_n \in k_n$. Then for any $m \in \N$,
  $\{a_n\}_{n = m}^\infty$ is a sequence
  in $k_m$, which is compact. For $m = 1$,
  there exists a convergent
  subsequence $\{a_{n_k}\}$ such that
  $x = \lim_{k \to \infty} a_{n_k} \in k_1$. For any
  $m \in \N$, there is $k \in \N$ such that for all
  $k \ge k_m$, $n_k \ge m$. So
  \[a_{n_k} \in k_{n_k} \subseteq k_m,\]
  which means that $\{a_{n_k}\}$ is a convergent sequence
  in $k_m$, which is closed. Thus $x \in k_m$ for all
  $m$, which means that $x \in \bigcap_{n = 1}^\infty k_n$.
  So $\bigcap_{n = 1}^\infty k_n$ is nonempty.
\end{proof}

\begin{example}
  For $U_n = (0, 1 / n)$. Then
  $U_1 \supseteq U_2 \supseteq U_3 \supseteq \dots$, but
  their intersection is empty.
\end{example}

However, this is not so surprising, since the measure of
the sets $U_n$ tends to 0.

\begin{example}
  The sequence $V_n = (n, \infty)$ also satisfies
  $V_1 \supseteq V_2 \supseteq V_3 \supseteq \dots$,
  but their intersection is empty, despite each $V_n$ having
  infinite measure.
\end{example}

\section{Another Definition of Compactness}

\begin{theorem}
  A set $A \subseteq \R$ is compact if and only if
  it satisfies the following property:
  \begin{quote}
    (Covering property)\,
    For any family $U_\lambda$, $\lambda \in \Lambda$ of
    open subsets of $\R$ such that
    $A \subseteq \bigcup_{\lambda \in \Lambda} U_\lambda$,
    \footnote{This is called an \textit{open cover} of $A$.}
    there exists $n \in \N$ and
    $\lambda_1, \dots, \lambda_n \in \Lambda$ such that
    $A \subseteq \bigcup_{k = 1}^n U_{\lambda_k}$.
    \footnote{I.e.~there exists a finite subcover.}
  \end{quote}
\end{theorem}

\begin{proof}
  $(\Leftarrow)$\, Assume that the covering property holds for
  $A$. For boundedness, let
  \[A \subseteq \bigcup_{n = 1}^\infty (-n, n) = \R.\]
  By the covering property, there exist
  $n_1 < n_2 < \dots < n_k$ such that
  \[A \subseteq \bigcup_{i = 1}^k (-n_i, n_i) = (-n_k, n_k).\]
  So $A$ is bounded. Now for closedness, suppose otherwise
  that $A$ is not closed. So there exists a limit point
  $x \notin A$. Then
  \[A \subseteq \R \setminus \{x\} = \bigcup_{n = 1}^\infty \left\{y : |y - x| > \frac{1}{n}\right\}.\]
  By the covering property, there exist
  $n_1 < n_2 < \dots < n_k$ such that
  \[A \subseteq \bigcup_{i = 1}^k \left\{y : |y - x| > \frac{1}{n_i}\right\} = \left\{y : |y - x| > \frac{1}{n}\right\}.\]
  But $x$ is a limit point of $A$, so there exists a
  $z \in V_{1 / n_k}(x) \cap A$. This is a contradiction.
  Hence $A$ is also closed, and thus $A$ is compact.

  $(\Rightarrow)$\, Assume $A$ is compact and let
  $U_\lambda$, $\lambda \in \Lambda$ be an open cover
  of $A$. Suppose otherwise that there does not
  exist a finite subcover.
  Since $A$ is compact, $A$ is bounded. So there exists
  an $M$ such that $A \subseteq [-M, M]$. Let $A_1 = A$
  and define a sequence $A_n$ of sets inductively by
  $A_n = A \cap [a_n, b_n] \ne \varnothing$ with
  \[
    |b_n - a_n| = \frac{2M}{2^{n - 1}}
  ,\]
  and $A_n$ can't be covered by a finite subcollection
  of $U_\lambda$. Suppose such $A_n$ is defined.
  Then
  \[A_n = \underbrace{\left(A \cap \left[a_n, \frac{b_n + a_n}{2}\right]\right)}_{= k_1} \cup \underbrace{\left(A \cap \left[\frac{a_n + b_n}{2}, b_n\right]\right)}_{= k_2}.\]
  So either $k_1$ or $k_2$ can't be covered by a finite
  subcollection of $U_\lambda$. Let this one be $A_{n + 1}$.
  Since $A_n$ is closed and bounded, $A_n$ is compact.
  Then we have a sequence of compact sets
  \[A_1 \supseteq A_2 \supseteq A_3 \supseteq \dots,\]
  so $\bigcap_{n = 1}^\infty A_n \ne \varnothing$.
  Since $|a_n - b_n| \to 0$ (prove this as exercise),
  there exists $x \in \R$ such that
  $\cap_{n = 1}^\infty A_n = \{x\}$. But there exists
  $\lambda_0$ such that $x \in U_{\lambda_0}$, which is open.
  So there exists $n$ such that
  $V_{1 / n}(x) \subseteq U_{\lambda_0}$, which means that
  $A_{n + 1} \subseteq U_{\lambda_0}$. This is a
  contradiction.
\end{proof}

\begin{example}
  Let $A = (0, 1)$, $\Lambda = (0, 1)$, and
  $U_\lambda = (0, \lambda)$. We have
  $A \subseteq \bigcup_{\lambda \in (0, 1)} U_\lambda$,
  but there does not exist a finite subcover.
\end{example}
