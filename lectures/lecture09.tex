\chapter{Sept.~19 -- Absolute Convergence}

\section{Absolute Convergence}

\begin{definition}
  Consider a series $\sum_{n=1}^\infty a_n$. If
  \[\sum_{n=1}^\infty |a_n|\]
  converges, then we say
  $\sum_{n=1}^\infty a_n$ \textbf{converges absolutely}.
\end{definition}

\begin{theorem}
  If $\sum_{n=1}^\infty |a_n|$ converges,
  then $\sum_{n = 1}^\infty a_n$ converges.
\end{theorem}

\begin{proof}
  For every $\epsilon > 0$, since $\sum_{n=1}^\infty a_n$
  converges, there is $N$ such that for all
  $m, k \ge N$,
  \[\sum_{n=k + 1}^m |a_n| < \epsilon.\]
  This is by the Cauchy criterion for series.
  Then for all $m, k \ge N$, we have
  \[
    \left\lvert \sum_{n=k+1}^m a_n\right\rvert
    \le \sum_{n=k+1}^m |a_n|
    < \epsilon
  \]
  by the triangle inequality. Apply
  the Cauchy criterion again to conclude that
  $\sum_{n=1}^\infty a_n$ converges.
\end{proof}

\begin{theorem}[Alternating series test]
  If $a_1 \ge a_2 \ge a_3 \ge \dots$ and
  $\lim_{n \to \infty} a_n = 0$, then
  the series
  \[
    \sum_{n = 1}^\infty (-1)^{n + 1} a_n
  \]
  converges.
\end{theorem}

\begin{proof}
  Set $s_m = \sum_{n=1}^m (-1)^{n+1} a_n$. Check
  that
  \[s_m - s_k = \sum_{n=k+1}^m (-1)^{n+1} a_n.\]
  Suppose that $m$ and $k$ are odd, then
  \[
    s_m - s_k = \underbrace{a_m - a_{m - 1}}_{\le 0} + a_{m-2} -
    \dots + a_{k + 2} - a_{k + 1}
  .\]
  So $s_m - s_k \le 0$.
  We can also group the terms as as
  \[
    s_m - s_k = a_m \underbrace{- a_{m - 1} + a_{m - 2}}_{\ge 0} -
    \dots - a_{k + 3} - a_{k + 2} - a_{k + 1}
    \ge a_{m} - a_{k + 1}
  .\]
  So $|s_m - s_k| \le |a_m| + |a_{k + 1}|$ by the
  triangle inequality.
  Since $\lim_{n \to \infty} a_n = 0$, for all
  $\epsilon > 0$, there is $N$ such that $n \ge N$,
  we have $|a_n| < \epsilon$. Then for all $m, k \ge N$,
  \[
    |s_m - s_k| \le |a_m| + |a_{k + 1}| < 2\epsilon
  .\]
  Thus $\{s_k\}$ converges.
  Left as exercise to check the other parities of
  $m$ and $k$ (group differently).
\end{proof}

\begin{example}
  We saw previously that for $a_n = \frac{1}{n}$,
  $\sum_{n=1}^\infty a_n$ diverges. But
  $\sum_{n = 1}^\infty (-1)^{n+1} a_n$ converges.
\end{example}

\section{Rearrangements}

\begin{definition}
  Given a series $\sum_{k = 1}^\infty a_k$,
  we say that a series $\sum_{k = 1}^\infty b_k$ is
  a \textbf{rearrangement} of $\sum_{k = 1}^\infty a_k$
  if there is a bijection $f : \N \to \N$
  such that $b_{f(k)} = a_k$ for all $k \in \N$.
\end{definition}

\begin{example}
  Let
  \begin{align*}
    S &= 1 - \frac{1}{2} + \frac{1}{3} - \frac{1}{4} + \frac{1}{5} - \frac{1}{6} + \frac{1}{7} - \frac{1}{8} + \cdots, \\
    \frac{1}{2}S &= \frac{1}{2} - \frac{1}{4} + \frac{1}{6} - \frac{1}{8} + \frac{1}{10} - \cdots, \\
    S + \frac{1}{2}S &= 1 + \frac{1}{3} - \frac{1}{2} + \frac{1}{5} - \frac{1}{4} + \frac{1}{7} - \cdots.
  \end{align*}
  Notice that $S + \frac{1}{2}S$ is a rearrangment
  of $S$. Supposing that $S + \frac{1}{2}S$ converges
  to the same limit as $S$, we would have
  \[S + \frac{1}{2}S = S,\]
  or $S = 0$. This cannot be the case.
\end{example}

\begin{remark}
  A rearrangement of a series might have
  different convergence properties from the original
  series.
\end{remark}

\begin{theorem}
  If a series converges absolutely to $A$, then
  any rearrangement of the series converges to the
  same limit $A$.
\end{theorem}

\begin{proof}
  Let $\sum_{k = 1}^\infty a_k$ converge absolutely
  to $A$. Let $\sum_{k = 1}^\infty b_k$ be a
  rearrangement of $\sum_{k = 1}^\infty a_k$.
  We set
  \[s_n = \sum_{k=1}^n a_k, \quad t_m = \sum_{k=1}^m b_k.\]
  We want to show that $t_m$ converges to $A$.
  Since $\lim_{n \to \infty} s_n = A$, for every
  $\epsilon > 0$, there is $N_1$ such that
  \[
  |s_n - A| < \frac{\epsilon}{2}
  \]
  for all $n \in \N$. Since $\sum_{k = 1}^\infty a_k$
  converges absolutely, there is $N_2$ such that
  for all $n, m \ge N_2$, we have
  \[
    \sum_{k = m + 1}^n |a_k| < \frac{\epsilon}{2}
  .\]
  Since $\sum_{k = 1}^\infty b_k$ is a rearrangement
  of $\sum_{k = 1}^\infty a_k$, we can write
  $b_{f(k)} = a_k$ for some bijection $f$.
  Set
  \[N = \max\{N_1, N_2\}, \quad M = \max\{f(k) : 1 \le k \le N\}.\]
  Then for all $m \ge M$, $t_m - s_n$ will only
  consist of terms $a_k$ for $k > N$. In particular,
  \[|t_m - s_n| \le \sum_{k=n}^\infty |a_k| < \frac{\epsilon}{2}.\]
  Then we have
  \[|t_m - A| = |t_m - s_n + s_n - A|
  \le |t_m - s_n| + |s_n - A| < \frac{\epsilon}{2} + \frac{\epsilon}{2}
  = \epsilon.\]
  So $\lim_{m \to \infty} t_m = A$.
\end{proof}
