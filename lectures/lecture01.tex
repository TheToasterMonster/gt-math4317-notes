\chapter{Aug.~22 -- The Real Numbers}

\section{Number Systems}
We start with the natural numbers
\footnote{$0 \notin \N$ for this class.}
\[\N = \{1, 2, 3, \dots\}.\]
These are perhaps
the most natural in a way, since they are what we use to
count things. They are closed under addition, but
fail when it comes to subtraction. For example,
$1 - 2 = -1 \notin \N$.
So we must expand our number system to the
integers
\[\Z = \{\dots, -3, -2, -1, 0, 1, 2, 3, \dots\}.\]
We can now add, subtract, and
multiply. But we run into problems when we start to
consider quotients. For example,
$1 \div 2 = \frac{1}{2} \notin \Z$.
So we continue to the rational numbers
\[\Q = \left\{\frac{p}{q} : p, q \in \Z,\, q \ne 0\right\}.\]
We now have summation, subtraction,
multiplication, and quotients. But there is still a
problem.

Consider the diagonal of a square with side length $1$.

\begin{theorem}
  $\sqrt{2}$ is not a rational number. 
  \footnote{
    In some sense, this shows that the notion of
    ``rationals'' is strictly weaker than the notion of
    ``length.''
  }
\end{theorem}

\begin{proof}
  Argue by contradiction. Suppose $\sqrt{2}$ is rational.
  Then we can write
   \[
     \sqrt{2} = \frac{p}{q}
  \]
  for some integers $p, q$. Further assume $p$ and
  $q$ have no common factors. Then
  \[2 = \frac{p^2}{q^2} \implies p^2 = 2q^2.\]
  So $p$ is even and we can write $p = 2r$ for some
  $r \in \Z$. Then
  \[4r^2 = 2q^2 \implies 2r^2 = q^2.\]
  So $q$ is also even, and $p, q$ share a common factor
  of $2$. Contradiction.
\end{proof}

Another weakness of $\Q$ is that we cannot take limits
($\Q$ is not complete). For example, note that
\[(\sqrt{2} - 1)(\sqrt{2} + 1) = 2 - 1 = 1,\]
\[
  \sqrt{2} = 1 + \frac{1}{\sqrt{2} + 1}
  = 1 + \frac{1}{1 + 1 + \frac{1}{\sqrt{2} + 1}}
  = \dots
.\]
So if we define the rational sequence
\[
  a_1 = 1, \quad
  a_2 = 1 + \frac{1}{2}, \quad
  a_3 = 1 + \frac{1}{2 + \frac{1}{2}}, \quad
  a_4 = 1 + \frac{1}{2 + \frac{1}{2 + \frac{1}{2}}},
  \quad \dots
,\]
then as $n \to \infty$, $a_n \to \sqrt{2} \notin \Q$.

\section{Sets}
Sets are any collections of objects. Given a set $A$,
we write $x \in A$ if $x$ is an element of $A$. We write
$x \notin A$ otherwise. The \textbf{union} of two sets is
\[
  A \cup B = \{x : x \in A \text{ or } x \in B\}
,\]
and the \textbf{intersection} of two sets is
\[
  A \cap B = \{x : x \in A \text{ and } x \in B\}
.\]
We use the notation
\[\bigcup_{k = 1}^{\infty} A_k\]
to denote the countable union of a family of sets
indexed by $k$.

\section{Functions}
\begin{definition}
  Given two sets $A$ and $B$, a \textbf{function}
  from $A$ to $B$ is a rule, relation, or mapping that
  takes each element $x \in A$ and associates with
  it a single element in $B$. In this case,
  we write $f : A \to B$.
\end{definition}

We call $A$ the \textbf{domain} of $f$ and $B$ the
\textbf{codomain} of $f$. The element in $B$ associated
with $x \in A$ is $f(x)$, called the \textbf{image}
of $x$. The \textbf{range} of $f$ is
\[\text{range}(f) = \{y \in B : y = f(x) \text{ for some }x \in A\}.\]

We say $f$ is:
\begin{enumerate}
  \item \textbf{onto} or \textbf{surjective} if
    $\text{range}(f) = B$.
  \item \textbf{one-to-one} or \textbf{injective} if
    $x, x' \in A$ and $x \ne x'$, then $f(x) \ne f(x')$.
  \item \textbf{bijective} if it is injective and
    surjective.
\end{enumerate}

\begin{example}
  First Dirichlet function:
  \[
  g(x) =
  \begin{cases}
    1 & \text{if $x \in \Q$} \\
    0 & \text{if $x \notin \Q$}
  \end{cases}
  =
  \lim_{k \to \infty} \left(\lim_{j \to \infty} \left[\cos(k! \pi x)\right]^{2j}\right)
  .\]
\end{example}

\begin{example}
  Second Dirichlet function:
  \[
  f(x) =
  \begin{cases}
    \frac{1}{q} & \text{if $x = \frac{p}{q} \in \Q$ in lowest terms} \\
    0 & \text{if $x \notin \Q$}.
  \end{cases}
  \]
\end{example}

\begin{example}
  Absolute value:
  \[
    |x| =
    \begin{cases}
      x & \text{if $x \ge 0$} \\
      -x & \text{if $x < 0$}.
    \end{cases}
  \]
  Note that we have the following two properties:
  \begin{itemize}
    \item $|xy| = |x||y|$.
    \item $|x + y| \le |x| + |y|$. This is called the
      \textit{triangle inequality}.
  \end{itemize}
\end{example}

\section{Induction}
If we have a set $S \subseteq \N$ and
\begin{enumerate}
  \item $1 \in S$
  \item if $n \in S$, then $n + 1 \in S$
\end{enumerate}
then $S = \N$.
\footnote{We always use induction in conjunction with
$\N$.}
