\chapter{Sept.~7 -- Bolzano-Weierstrass Theorem}

\section{Review of Limits}

\begin{theorem}[Squeeze theorem]
  Let $\{x_n\}, \{y_n\}, \{z_n\}$ be sequences such
  that $x_n \le y_n \le z_n$ for all $n$, and suppose that
  \[\lim_{n \to \infty} x_n = \lim_{n \to \infty} z_n = l.\]
  Then $\lim_{n \to \infty} y_n = l$.
\end{theorem}

\begin{proof}
  Consider $|y_n - l|$.
  If
  \begin{align*}
    y_n - l \ge 0, &\quad \text{then} \quad y_n - l \le z_n - l, \\
    y_n - l < 0, &\quad \text{then} \quad |y_n - l| = l - y_n \le l - x_n.
  \end{align*}
  So we have
  \[
  |y_n - l| \le |z_n - l| + |x_n - l|
  .\]
  For all $\epsilon > 0$, there exist $N_1, N_2$ such that
  for all $n \ge \N_1$,
  \[
  |z_n - l| < \frac{\epsilon}{2},
  \]
  and for all $n \ge N_2$, \[
  |x_n - l| < \frac{\epsilon}{2}
  .\]
  Take $N = \max\{N_1, N_2\}$. If $n \ge N$, then
  \[
  |y_n - l| \le |z_n - l| + |x_n - l| <
  \frac{\epsilon}{2} + \frac{\epsilon}{2} = \epsilon
  .\]
  So $\lim_{n \to \infty} y_n = l$.
\end{proof}

\section{Subsequences and the Bolzano-Weierstrass Theorem}
\begin{definition}
  Let $\{a_n\}$ be a sequence of real numbers.
  Let $n_1 < n_2 < n_3 < \dots$ be an increasing sequence of
  natural numbers. Then $\{a_{n_1}, a_{n_2}, \dots, \}$
  is a \textbf{subsequence} of $\{a_n\}$, and it is
  denoted by $\{a_{n_k}\}$.
\end{definition}

\begin{example}
  Let
  \[\{a_n\} = \left\{1, \frac{1}{2}, \frac{1}{3}, \frac{1}{4}, \dots\right\}.\]
  Then 
  \[\left\{\frac{1}{2}, \frac{1}{4}, \frac{1}{6}, \frac{1}{8}, \dots\right\}\]
  is a subsequence of $\{a_n\}$.
  However, note that
  \[\left\{\frac{1}{10}, \frac{1}{5}, \frac{1}{100}, \frac{1}{500}, \dots\right\}\]
  is \textit{not} a subsequence of $\{a_n\}$ since the
  the $n_k$ are not strictly increasing. Similarly,
  \[\left\{1, \frac{1}{3}, \frac{1}{3}, \frac{1}{5}, \frac{1}{5} \dots\right\}\]
  is also not a subsequence of $\{a_n\}$.
\end{example}

\begin{theorem}
  \label{thm:subseq}
  Subsequences of a convergent sequence converge to the
  same limit as the original sequence.
\end{theorem}

\begin{proof}
  Suppose $\lim_{n \to \infty} a_n = a$. So for every
  $\epsilon > 0$, there exists $N$ such that
  $|a_n - a| < \epsilon$ for all $n \ge N$.
  Consider an arbitrary subsequence $\{a_{n_k}\}$.
  Note that $n_k \ge k$. So when $k \ge N$,
  \[|a_{n_k} - a| < \epsilon.\]
  Therefore $\lim_{k \to \infty} a_{n_k} = a$.
\end{proof}

\begin{example}
  Let $0 < b < 1$. Clearly
  \[1 > b > b^2 > b^3 > b^4 > \dots \ge 0.\]
  The sequence $\{b^n\}$ is decreasing and bounded below,
  so by the monotone convergence theorem,
  $\lim_{n \to \infty} b^n = l \in \R$ exists. Note that
  $\{b^{2n}\}$ is a subsequence of $\{b^n\}$, so
  by Theorem \ref{thm:subseq}, we have
  $\lim_{n \to \infty} b^{2n} = l$.
  Note that $b^{2n} = b^n b^n$. By the algebraic limit
  theorem,
  \[\lim_{n \to \infty} b^{2n} = \left(\lim_{n \to \infty} b^n\right)\left(\lim_{n \to \infty} b^n\right).\]
  Therefore, $l = l^2$, so we have $l = 0$ or $l = 1$. But
  the entire sequence is strictly less than $1$ and
  decreasing, so $l = 0$.
\end{example}

\begin{example}
  Consider the sequence
  \[\{(-1)^n\} = \{-1, 1, -1, 1, \dots\}.\]
  This sequence does not converge. But the
  subsequence
  \[
    \{-1, -1, -1, \dots\}
  \]
  does converge.
\end{example}

\begin{remark}
  This shows that the converse of Theorem \ref{thm:subseq} is
  not true, i.e.~a convergent subsequence does not
  imply that the original sequence converges.
\end{remark}

\begin{example}
  The sequence
  \[
    a_n =
    \begin{cases}
      1 & \text{if $n$ is prime}, \\
      0 & \text{otherwise}
    \end{cases}
  \]
  does not converge.
\end{example}

\begin{exercise}
  Show the limit of the sequence
  \[
    \left\{1, -\frac{1}{2}, \frac{1}{3}, -\frac{1}{4}, \frac{1}{5}, \frac{1}{5}, -\frac{1}{5}, \frac{1}{5}, -\frac{1}{5}, \dots\right\}
  \]
\end{exercise}

\begin{proof}
  The subsequence
  \[\left\{\frac{1}{5}, \frac{1}{5}, \dots\right\}\]
  converges to $\frac{1}{5}$ while the subsequence
  \[\left\{-\frac{1}{5}, -\frac{1}{5}, \dots\right\}\]
  converges to $-\frac{1}{5}$. Thus the original
  sequence diverges.
\end{proof}

\begin{remark}
  If we can find two subsequences that converge to different
  limits, then the original sequence diverges.
  This is the contrapositive of Theorem \ref{thm:subseq}.
\end{remark}

\begin{theorem}[Bolzano-Weierstrass theorem]
  Every bounded sequence has a convergent subsequence.
  \footnote{This demonstrates some kind of
    \textit{compactness} of the real numbers.}
\end{theorem}

\begin{proof}
  Let $\{a_n\}$ be a bounded  be a bounded sequence.
  So there exists $M > 0$ such that $\sup_{n} |a_n| < M$.
  So $a_n$ is contained in $[-M, M]$. Split
  $[-M, M]$ into $[-M, 0]$ and $[0, M]$.
  Pick one that contains infinitely many elements of
  $\{a_n\}$ and call it $I_1$. Then pick
  $a_{n_1} \in \{a_n\}$ such that $a_{n_1} \in I_1$.
  Split $I_1$ again into two closed intervals of
  the same size. Take one of these two that contains
  infinitely many elements of $\{a_n\}$ and call it $I_2$.
  Then take $a_{n_2} \in \{a_n\}$ such that
  $a_{n_2} \in I_2$. Repeat this process to to get
  $I_{k+1} \subseteq I_k$ with
  $|I_{k+1}| = \frac{1}{2}|I_k|$ such that $I_{k+1}$
  contains infinitely many elements of $\{a_n\}$.
  \footnote{Here, by $|I_k|$ we mean the length of the
    interval $I_k$.}
  Also pick $a_{n_{k+1}} \in \{a_n\}$ such that
  $a_{n_{k+1}} \in I_{k+1}$ with $n_{k + 1} > n_k$.
  
  By construction, $\{a_{n_k}\}$ is a subsequence of
  $\{a_n\}$ and $a_{n_k} \in I_k$. We have the
  $I_k$ being closed intervals with
  \[
  I_1 \supseteq I_2 \supseteq I_3 \supseteq \dots
  .\]
  So there exists $x \in \R$ such that
  $x \in \bigcap_{k = 1}^{\infty} I_k$. Note that
  $|I_k| = M\left(\frac{1}{2}\right)^{k-1}$. Then we claim
  $\lim_{k \to \infty} a_{n_k} = x$.

  Let $\epsilon > 0$. Take $N$ such that
  \[2^N > \frac{2M}{\epsilon}.\]
  Then for every $k \ge N$, we have
  \[
    |a_{n_k} - x| \le M \left(\frac{1}{2}\right)^{k - 1}
    < \epsilon
  \]
  since $a_{n_k}, x \in I_k$.
  Thus $\lim_{k \to \infty} a_{n_k} = x$, and
  $\{a_{n_k}\}$ is a convergent subsequence.
\end{proof}
