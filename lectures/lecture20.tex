\chapter{Nov.~2 -- Power Series}

\section{Power Series}

\begin{definition}
  A \textbf{power series} is a series of the form
  \[\sum_{n = 0}^\infty a_n x^n.\]
\end{definition}

\begin{theorem}
  If \[\sum_{n = 0}^\infty a_n x^n\] converges at
  $x_0 \in \R$,
  then it converges absolutely at any $x$ such that
  $|x| < |x_0|$.
\end{theorem}

\begin{proof}
  Since $\sum_{n = 0}^\infty a_n x_0^n$ converges,
  there exists $M > 0$ such that $|a_nx_0^n| < M$
  for all $n$. Then observe
  \[
    |a_nx^n| = |a_nx_0^n| \left|\frac{x^n}{x_0^n}\right|
    = |a_nx_0^n| \left|\frac{x}{x_0}\right|^n.
  \]
  Now let $|x| < |x_0$, so that
  \[
    \sum_{n = 0}^\infty |a_n x^n| = \sum_{n = 0}^\infty |a_n x_0^n|\left|\frac{x}{x_0}\right|^n
    \le M \sum_{n = 0}^\infty \left|\frac{x}{x_0}\right|^n.
  \]
  Note that $\left|\frac{x}{x_0}\right| < 1$, so the
  right hand side converges by geometric series.
  Thus $\sum_{n = 0}^\infty |a_n x^n| < \infty$.
\end{proof}

\begin{remark}
  We need to make this argument since $\sum_{n = 0}^\infty a_n x_0^n$
  need not converge absolutely. So we cannot simply do a
  direct comparison.
\end{remark}

\begin{theorem}
  If \[\sum_{n = 0}^\infty a_n x^n\] converges absolutely
  at $x_0 \in \R$, then it converges uniformly on
  $[-c, c]$ where $c = |x_0|$.
\end{theorem}

\begin{proof}
  By assumption, $\sum_{n = 0}^\infty |a_n x_0^n| < \infty$.
  Then for any $x \in [-c, c]$, note that
  $|a_nx^n| \le |a_nx_0^n|$. Then
  \[
    |a_{m + 1} x^{m + 1} + \dots + a_nx^n|
    \le |a_{m + 1} x^{m + 1}| + \dots + |a_nx^n|
    \le |a_{m + 1} x_0^{m + 1}| + \dots + |a_nx_0^n|
  \]
  for $m < n$. For any $\epsilon$, we can find $N$
  such that for any $n > m \ge N$, we have
  \[|a_{m + 1}x_0^{m + 1}| + \dots + |a_nx_0^n| < \epsilon.\]
  Then by the Cauchy criterion for uniform convergence,
  $\sum_{n = 0}^\infty a_n x^n$ converges uniformly on
  $[-c, c]$.
\end{proof}

\begin{remark}
  This result holds on a closed interval,
  whereas the previous holds on an open interval.
\end{remark}

\begin{example}
  Consider the sum
  \[\sum_{n = 1}^\infty \frac{(-1)^n}{n} x^n,\]
  which converges at $x = 1$ by the alternating series test.
  Then by the previous theorem, this series converges
  absolutely on $(-1, 1)$. However, observe that at $x = -1$,
  we have
  \[\sum_{n = 1}^\infty \frac{(-1)^n}{n} (-1)^n = \sum_{n = 1}^\infty \frac{1}{n},\]
  which diverges. This is an example of why the
  interval is open.
\end{example}

\section{Abel's Theorem}
\begin{lemma}[Abel's lemma]
  Let $\{b_n\}$ be a sequence satisfying
  $b_1 \ge b_2 \ge b_3 \dots \ge 0$ and $\{a_n\}$ satisfy
  \[|s_k| = \left|\sum_{n = 1}^k a_n\right| \le A\]
  for some $A > 0$. Then
  $|a_1b_1 + a_2b_2 + \dots a_nb_n| \le Ab_1$.
\end{lemma}

\begin{proof}
  See Homework 5. The key is summation by parts.
\end{proof}

\begin{theorem}[Abel's theorem]
  If
  \[\sum_{n = 0}^\infty a_nx^n\]
  converges at $x = R > 0$, then $\sum_{n = 0}^\infty a_nx^n$
  converges uniformly on $[0, R]$. The same result holds for
  $R < 0$.
\end{theorem}

\begin{proof}
  Fix $0 \le x \le R$. Observe that
  \[
    \sum_{n = 0}^\infty a_n x^n =
    \sum_{n = 0}^\infty a_n \frac{x^n}{R^n} R^n.
  \]
  Note that $\sum_{n = 0}^\infty a_n R^n$
  converges by assumption. By the Cauchy criterion, for
  every $\epsilon > 0$, there exists $N$ such that
  for all $n > m \ge N$, we have
  \[|a_{m + 1}R^{m + 1} + \dots + a_nR^n| < \frac{\epsilon}{2}. \tag{$+$}\]
  Now consider
  \[
    |a_{m + 1}x^{m + 1} + \dots + a_nx^n|
    = \left|a_{m + 1} R^{m + 1} \left(\frac{x}{R}\right)^{m + 1}
    + \dots + a_n R^n \left(\frac{x}{R}\right)^n\right|.
    \tag{$*$}
  \]
  Then pick
  \[b_1 = \left(\frac{x}{R}\right), \quad \dots, \quad b_{n - m} = \left(\frac{x}{R}\right)^n,\]
  so that $b_1 \ge b_2 \ge \dots \ge b_{n - m} \ge 0$
  since $1 \ge \frac{x}{R} \ge 0$. Also pick
  \[\alpha_1 = a_{m + 1} R^{m + 1}, \quad \dots, \quad \alpha_{n - m} = a_nR^n.\]
  Then we have 
  \[(*) = |\alpha_1 b_1 + \alpha_2 b_2 + \dots + \alpha_{n - m}b_{n - m}|.\]
  Note that $\left|\sum_{j = 1}^k \alpha_j\right| \le \frac{\epsilon}{2}$
  for $k \le n - m$ by $(+)$. Then by Abel's lemma,
  \[(*) = |\alpha_1 b_1 + \alpha_2 b_2 + \dots + \alpha_{n - m}b_{n - m}| \le \frac{\epsilon}{2} \left(\frac{x}{R}\right)^{m + 1} < \epsilon.\]
  Therefore by the Cauchy criterion,
  $\sum_{n = 0} a_n x^n$ converges uniformly on $[0, R]$.
\end{proof}

\begin{remark}
  Observe that for $R < 0$ and $x \in [R, 0]$, we still have
  $0 \le \frac{x}{R} \le 1$, so the proof is the same.
\end{remark}

\begin{theorem}
  If
  \[\sum_{n = 0}^\infty a_n x^n\]
  converges pointwise
  on a set $A \subseteq \R$, then it converges
  uniformly on any compact set $K \subseteq A$.
\end{theorem}

\begin{proof}
  Since $K$ is compact, it is closed and bounded.
  In particular, $x_0 = \inf K$ exists and $x_0 \in K$.
  Similarly, $x_1 = \sup K$ exists and $x_1 \in K$.
  Note that $K \subseteq [x_0, x_1]$. By assumption,
  $\sum_{n = 0}^\infty a_n x^n$ converges at $x_0$ and $x_1$.
  Then by Abel's theorem,
  $\sum_{n = 0}^\infty a_n x^n$
  converges uniformly on $[x_0, x_1] \supseteq K$.
  \footnote{To be more precise, we need some case work
  here depending on the signs of $x_0$ and $x_1$.}
\end{proof}
