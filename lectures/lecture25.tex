\chapter{Nov.~30 -- Review for Final, Part 1}

\section{Discussion of Exam 2}
\begin{exercise}
  Suppose $f : [0, \pi / 2] \to \R$ is continuous and
  $f(t) \in [0, 1]$ for every $t \in [0, \pi / 2]$. Prove
  that there exists $t \in [0, \pi / 2]$ with
  $f(t) = \sin (t)$. (You may use without proof that
  $\sin$ is continuous.)
\end{exercise}

\begin{proof}[Solution]
  Define $g(t) = f(t) - \sin t$. Note that
  \[g(0) = f(0) - \sin 0 = f(0).\]
  So if $f(0) = 0$, then we are done. Also note that
  \[
    g(\pi / 2) = f(\pi / 2) - \sin (\pi / 2) = f(\pi / 2) - 1,
  \]
  so if $f(\pi / 2) = 1$, then we are also done. So
  assume $f(0) > 0$ and $f(\pi / 2) < 1$. Then $g(0) > 0$
  and $g(1) < 0$, after which the result follows from the
  intermediate value theorem.
\end{proof}

\section{More on Metric Spaces}
\begin{definition}
  Let $E$ be a subset of $(X, d)$. The \textbf{closure}
  of $E$, denoted $\overline{E}$, is the union of $E$
  and its limit points. The \textbf{interior} $E^\circ$ is defined to be
  \[
    E^\circ = \{x \in E : \text{there exists $V_\epsilon(x) \subseteq E$ for some $\epsilon > 0$}\}.
  \]
\end{definition}

\begin{definition}
  A set $A \subseteq X$ is \textbf{dense} in $(X, d)$ if
  $\overline{A} = X$.
\end{definition}

\begin{definition}
  A subset $E$ is \textbf{nowhere-dense} in $(X, d)$ if
  ${\overline{E}}^\circ = \varnothing$
  (interior of the closure is empty).
\end{definition}

\begin{theorem}[Baire category theorem]
  A complete metric space is not the union of a countable
  collection of nowhere dense sets.
\end{theorem}

\section{Review for Final}

\subsection{Cesaro Means}
\begin{definition}
  Given a sequence $\{x_n\}_{n = 1}^\infty$, the
  \textbf{Cesaro means} of $\{x_n\}_{n = 1}^\infty$ are
  $\{y_n\}_{n = 1}^\infty$ given by
  \[
    y_n = \frac{x_1 + \dots + x_n}{n}.
  \]
\end{definition}

\begin{prop}
  We have the following properties about a sequence
  $\{x_n\}$ and
  its Cesaro means $\{y_n\}$:
  \begin{enumerate}[(a)]
    \item Suppose $\lim_{n \to \infty} x_n = x$. Then
      $\lim_{n \to \infty} y_n = x$.
    \item $\lim_{n \to \infty} y_n$ exists does not
      imply that $\lim_{n \to \infty} x_n$ exists.
  \end{enumerate}
\end{prop}

\begin{proof}
  $(a)$\, For all $\epsilon > 0$, we can find $N_1$ such that
  for all $n \ge N_1$, we have
  $|x_n - x| < \epsilon$. Then
  \[
    y_n - x = \frac{x_1 + \dots + x_n - nx}{n}
    = \frac{(x_1 - x) + \dots + (x_n - x)}{n}
    = \frac{\sum_{k = 1}^{N_1 - 1} (x_k - x)
    + \sum_{j = N_1}^{n} (x_j - x)}{n}.
  \]
  Now note that
  \[
    \left|\sum_{k = 1}^{N_1 - 1} (x_k - x)\right|
    \le N_1 \max_{k = 1, \dots, N_1 - 1} |x_k - x|.
  \]
  For the other sum, we have
  \[
    \left|\sum_{j = N_1}^n (x_j - x)\right|
    \le (n - N_1 + 1)\epsilon.
  \]
  Then
  \[
  |y_n - x| \le \frac{N_1 \max_{k = 1, \dots, N_1 - 1} |x_k - x|}{n} + \frac{(n - N_1 + 1)\epsilon}{n}.
  \]
  So take $N_2$ large enough so that
  \[
    \frac{N_1 \max_{k = 1, \dots, N_1 - 1} |x_k - x|}{N_2} < \epsilon.
  \]
  Then for $n \ge \max\{N_1, N_2\}$, we have
  \[
    |y_n - x| < \epsilon + \epsilon = 2\epsilon.
  \]
  Thus we have $\lim_{n \to \infty} y_n = x$, as desired.

  $(b)$\, Simply take $x_n = (-1)^n$. Then
  \[
    y_n = \frac{-1 + 1 - 1 + \dots + (-1)^n}{n}.
  \]
  So $|y| \le 1 / n$ and
  $\lim_{n \to \infty} y_n = 0$, but $\{x_n\}$ diverges.
\end{proof}

\subsection{Limit Superior}
\begin{exercise}
  Let $\{a_n\}$ be a bounded sequence.
  \begin{enumerate}[(a)]
    \item Let
    \[
      y_n = \sup\{a_k : k \ge n\}.
    \]
    Then $\lim_{n \to \infty} y_n$ exists, and we define
    $\limsup_{n \to \infty} a_n = \lim_{n \to \infty} y_n$.
  \item Let
    \[z_n = \inf\{a_k : k \ge n\}.\]
    Then $\lim_{n \to \infty} z_n$ exists, and we define
    $\liminf_{n \to \infty} a_n = \lim_{n \to \infty} z_n$.
  \item We have $\lim_{n \to \infty} a_n = L$ exists if and
    only if
    \[\limsup_{n \to \infty} a_n = \liminf_{n \to \infty} a_n = L.\]
  \end{enumerate}
\end{exercise}

\begin{proof}
  $(a)$\, Note that
  \[\{a_k : k \ge n + 1\} \subseteq \{a_k : k \ge n\},\]
  so $y_{n + 1} \le y_n$. Then $\{y_n\}$ is a
  bounded, nonincreasing sequence, so
  $\lim_{n \to \infty} y_n$
  exists by the monotone convergence theorem.

  $(b)$\, We have $z_n \le z_{n + 1}$ by a similar argument.
  So $\{z_n\}$ is bounded and monotone, so
  $\lim_{n \to \infty} z_n$ exists by the monotone convergence
  theorem.

  $(c)$\, $(\Rightarrow)$\,
  Suppose $\lim_{n \to \infty} a_n = L$.
  Then for all $\epsilon > 0$, there exists $N$ such that
  for all $n \ge N$, we have
  \[
    L - \epsilon \le a_n \le L + \epsilon.
  \]
  Then when $n \ge N$, we have
  \[
    L - \epsilon \le y_n \le L + \epsilon \quad \text{and}
    \quad L - \epsilon \le z_n \le L + \epsilon.
  \]
  Then it follows that
  $\lim_{n \to \infty} y_n = \lim_{n \to \infty} z_n = L$.

  $(c)$\, $(\Leftarrow)$\, Note that
  $z_n \le a_n \le y_n$ by construction. Then if
  $\lim_{n \to \infty} y_n = \lim_{n \to \infty} z_n = L$,
  by the squeeze theorem we also have
  $\lim_{n \to \infty} a_n = L$.
\end{proof}
