\chapter{Oct.~31 -- Sequences of Functions}

\section{Pointwise Convergence}
\begin{definition}
  For all $n \in \N$, let $f_n$ be a function defined
  on $A \subseteq \R$. The sequence of functions $\{f_n\}$
  \textbf{converges pointwise} on $A$ to a function $f$ if
  for all $x \in A$, we have
  $\lim_{n \to \infty} f_n(x) = f(x)$.
\end{definition}

\begin{example}
  Let $f_n : \R \to \R$ be defined by
  \[f_n(x) = \frac{x + nx}{n}.\]
  For any fixed $x \in \R$, we have
  $f_n(x) = x + \frac{x}{n} \to x$ as $n \to \infty$.
  So $\{f_n\}$ converges pointwise on $\R$ to $f(x) = x$.
\end{example}

\begin{example}
  Let $g_n : [0, 1] \to \R$ be defined by $g_n(x) = x^n$.
  First note that for $x = 1$,
  \[g_n(1) = 1^n = 1,\]
  so $\lim_{n \to \infty} g_n(1) = 1$. For any
  fixed $0 \le x < 1$,
  \[\lim_{n \to \infty} x^n = 0.\]
  So $\{g_n\}$ converges pointwise on $[0, 1]$ to
  \[g(x) = \begin{cases}
    1 & \text{if }x = 1 \\
    0 & \text{if } 0 \le x < 1.
  \end{cases}\]
\end{example}

\begin{example}
  Let $h_n : [-1, 1] \to \R$ be defined by
  \[
    h_n(x) = x^{1 + \frac{1}{2n - 1}}
  .\]
  Note that $2n - 1$ is odd for every $n$, so $h_n(x)$
  is well-defined even for $x < 0$. Observe that
  \[h_n(x) = x \cdot x^{\frac{1}{2n - 1}}.\]
  For $x = 0$, we have $h_n(0) = 0$, so
  $\lim_{n \to \infty} h_n(0) = 0$. For $0 < x \le 1$,
  we have
  \[\lim_{n \to \infty} x^{\frac{1}{2n - 1}} = 1.\]
  Finally, if $-1 \le x < 0$, then we have
  \[\lim_{n \to \infty} x^{\frac{1}{2n - 1}} = \lim_{n \to \infty} (-1)^{\frac{1}{}}(-x)^{2n - 1} = -1 \lim_{n \to \infty} (-x)^{\frac{1}{2n - 1}} = -1\]
  since $2n - 1$ is odd and $0 < -x \le 1$. Then by the
  algebraic properties of limits, we have
  \[
    \lim_{n \to \infty} h_n(x) = \begin{cases}
      0 & \text{if } x = 0 \\
      x & \text{if } 0 < x \le 1 \\
      -x & \text{if } {-1 \le x < 0}.
    \end{cases}.
  \]
  So $\{h_n\}$ converges pointwise on $[-1, 1]$ to
  $h(x) = |x|$.
\end{example}

\begin{remark}
  The pointwise limit of a sequence of
  continuous functions may fail to be continuous, and
  the pointwise limit of a sequence of
  differentiable functions may fail to be differentiable.
\end{remark}

\section{Uniform Convergence}
\begin{definition}
  Let $f_n$ be functions on $A \subseteq \R$. We say that
  \textbf{$f_n \to f$ uniformly} on $A$ if for all
  $\epsilon > 0$, there exists $N \in \N$ such that
  for all $x \in A$ and $n \ge N$, one has
  $|f_n(x) - f(x)| < \epsilon$.
\end{definition}

\begin{example}
  Let $g_n : \R \to \R$ be defined by
  \[g_n = \frac{1}{n(1 + x)^2}.\]
  For any fixed $x \in \R$, we have
  \[g_n(x) = \frac{1}{n(1 + x)^2} \to 0,\]
  so $g_n(x) \to 0$ pointwise on $\R$. Further observe that
  \[
    |g_n(x) - 0| = \frac{1}{n(1 + x)^2} \le \frac{1}{n}.
  \]
  So for any $\epsilon > 0$, pick $N \in \N$
  such that $N > \frac{1}{\epsilon}$. Then for all
  $x \in \R$ and $n \ge N$, we have
  \[|g_n(x) - 0) \le \frac{1}{n} < \epsilon.\]
  So in fact $g_n(x) \to 0$ uniformly.
\end{example}

\begin{example}
  Let
  \[g_n(x) = \frac{x^2 + nx}{n} = \frac{x^2}{n} + x.\]
  For any fixed $x \in \R$, we have $g_n(x) \to x$,
  so $g_n(x)$ converges pointwise to $x$ on $\R$.
  Then we can note that
  \[|g_n(x) - x| \le \frac{x^2}{n},\]
  so we need to take $N > \frac{x^2}{\epsilon}$ to satisfy
  $|g_n(x) - x| < \epsilon$ for $n \ge N$. So
  this convergence is not uniform.
\end{example}

\begin{theorem}[Cauchy criterion for uniform convergence]
  Let $\{f_n\}$ be a sequence of functions defined
  on $A \subseteq \R$. Then $f_n$ converges uniformly
  on $A$ if and only if for all $\epsilon > 0$,
  there exists $N \in \N$ such that for all $x \in A$
  and $m, n \ge N$, we have
  $|f_n(x) - f_m(x)| < \epsilon$.
\end{theorem}

\begin{proof}
  Just apply the Cauchy criterion for sequences at every
  point.
\end{proof}

\begin{theorem}
  Let $\{f_n\}$ be a sequence of functions defined
  on $A$ and suppose that $f_n \to f$ uniformly on $A$.
  If each $f_n$ is continuous at $c \in A$, then
  $f$ is also continuous at $c$.
\end{theorem}

\begin{proof}
  For all $\epsilon > 0$, since $f_n \to f$ uniformly,
  we can find $N \in \N$ such that whenever $n \ge N$,
  we have
  \[|f_n(x) - f(x)| < \frac{\epsilon}{3}\]
  for all $x \in A$. Take $n = N$. Since $f_N$ is continuous
  at $c$, there exists $\delta$ such that for all
  $|x - c| < \delta$, one has
  \[|f_N(x) - f_N(c)| < \frac{\epsilon}{3}.\]
  Then we can note that
  \[|f(x) - f(c)| = |f(x) - f_N(x) + f_N(x) - f_N(c) + f_N(c) - f(c)|.\]
  By the triangle inequality, we have
  \[|f(x) - f(c)| \le |f(x) - f_N(x)| + |f_N(x) - f_N(c)| + |f_N(c) - f(c)|.\]
  So for all $|x - c| < \delta$, we have
  \[|f(x) - f(c)| \le \frac{\epsilon}{3} + \frac{\epsilon}{3} + \frac{\epsilon}{3} = \epsilon,\]
  where the first and last terms are from uniform
  convergence and the middle term is from continuity.
  So for every $\epsilon > 0$, we can find $\delta > 0$
  such that for all $|x - c| < \delta$, we have
  $|f(x) - f(c)| < \epsilon$. Thus we can conclude that
  $f$ is continuous at $c$.
\end{proof}

\begin{remark}
  We can also pass differentiability and integrability
  through uniform limits.
\end{remark}

\section{Series of Functions}

\begin{definition}
  Let $f$ and $f_n$ be functions defined on $A \subseteq \R$
  for each $n \in \N$. We say that
  $\sum_{n = 1}^\infty f_n$ \textbf{converges pointwise}
  to $f(x)$ if the partial sums
  \[s_k(x) = \sum_{n = 1}^k f_n(x)\]
  converge pointwise to $f(x)$ for every $x \in A$.
  Similarly, we say that $\sum_{n = 1}^\infty f_n$
  \textbf{converges uniformly} to $f(x)$ if the partial sums
  converge uniformly to $f(x)$ for every $x \in A$.
\end{definition}

\begin{theorem}
  Let $f_n(x)$ be a continuous function defined on a set
  $A \subseteq \R$ and assume that
  $\sum_{n = 1}^\infty f_n(x)$ converges uniformly to
  $f$. Then $f$ is continuous on $A$.
\end{theorem}

\begin{proof}
  If $f_n$ is continuous, then
  $s_k(x) = \sum_{n = 1}^k f_n(x)$ is continuous.
  Then we can pass continuity through the limit since
  the convergence of $s_k(x)$ is uniform.
\end{proof}

\begin{theorem}
  A series $\sum_{n = 1}^\infty f_n$ converges uniformly
  on $A$ if and only if for all $\epsilon > 0$, there
  exists $N \in \N$ such that for all $n > m \ge N$ and
  $x \in A$, one has
  $|f_{m + 1}(x) + \cdots + f_n(x)| < \epsilon$.
\end{theorem}

\begin{proof}
  This is just the Cauchy criterion for series.
\end{proof}

\begin{corollary}
  For every $n \in \N$, let $f_n$ be a function
  defined on $A \subseteq \R$. Assume
  \[|f_n(x)| \le M_n\]
  for all $x \in A$. If $\sum_{n = 1}^\infty M_n$
  converges, then $\sum_{n = 1}^\infty f_n$ converges
  uniformly on $A$.
\end{corollary}

\begin{proof}
  For every $\epsilon > 0$, since $\sum_{i = 1}^\infty M_n$
  converges,
  there exists $N$ such that
  for all $m, n \ge N$, we have
  \[M_{m + 1} + \dots + M_n < \epsilon.\]
  Note here that $0 \le |f_n(x)| \le M_n$, so we can
  drop the absolute values. Then for the same choice
  of $N$,
  \[
    |f_{m + 1}(x) + \dots + f_n(x)|
    \le |f_{m + 1}(x)| + \dots + |f_n(x)|
    \le M_{m + 1} + \dots + M_n < \epsilon.
  \]
  Apply the Cauchy criterion again to finish.
\end{proof}
