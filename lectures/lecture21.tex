\chapter{Nov.~7 -- Weierstrass Approximation Theorem}

\section{Bernstein Polynomials}
\begin{definition}
  A \textbf{Bernstein polynomial} is a polynomial
  of the form
  \[
    B_{n, k}(x) = \binom{n}{k} x^k (1 - x)^{n - k}
  \]
  for $k = 0, \dots, n$.
  \footnote{Here $\binom{n}{k} = \frac{n!}{(n - k)!k!}$ is
  the binomial coefficient.}
\end{definition}

Clearly we have $B_{n, k}(x) \ge 0$ for all $x \in [0, 1]$.
We also have
\[
  B_{1, 0} = 1 - x, \quad B_{1, 1} = x, \quad
  B_{2, 0} = (1 - x)^2, \quad B_{2, 1} = 2x(1 - x), \quad
  B_{2, 2} = x^2, \quad \dots
.\]

\begin{lemma}
  We have the following statements about
  Bernstein polynomials:
  \begin{enumerate}
    \item $\sum_{k = 0}^n B_{n, k}(x) = 1$,
    \item $\sum_{k = 0}^n k B_{n, k}(x) = nx$,
    \item and $\sum_{k = 0}^n k(k - 1) B_{n, k}(x) = n(n - 1)x^2$.
  \end{enumerate}
\end{lemma}

\begin{proof}
  For $(1)$, we note that
  \[
    1 = (x + 1 - x)^n = \sum_{k = 0}^n \binom{n}{k}
    x^k (1 - x)^{n - k} = \sum_{k = 0}^n B_{n, k}
  \]
  by the binomial theorem. For $(2)$, we can compute
  \begin{align*}
    \sum_{k = 0}^n k B_{n, k}(x)
    &= \sum_{k = 0}^n \frac{n!}{(n - k)!k!} x^k (1 - x)^{n - k} k
    = nx \sum_{k = 1}^n \frac{(n - 1)!}{(n - k)!(k - 1)!}
    x^{k - 1}(1 - x)^{n - k} \\
    &= nx \sum_{j = 0}^{n - 1} \frac{(n - 1)!}{(n - 1 - j)!j!} x^j (1 - x)^{n - 1 - j}
    = nx\sum_{j = 0}^{n - 1} B_{n - 1, j}(x)
    = nx
  \end{align*}
  by reindexing the sum and then using $(1)$.
  Also we dropped $k = 0$ in step two.
  Finally, for $(3)$, we have
  \begin{align*}
    \sum_{k = 0}^n k(k - 1)B_{n, k}(x)
    &= \sum_{k = 0}^n k(k - 1) \frac{n!}{(n - k)!k!} x^k (1 - x)^{n - k}
    = n(n - 1)x^2 \sum_{k = 2}^n \frac{(n - 2)!}{(n - k)!(k - 2)!} x^{k - 2} (1 - x)^{n - k} \\
    &= n(n - 1)x^2 \sum_{j = 0}^{n - 2} \frac{(n - 2)!}{(n - 2 - j)!j!}x^j (1 - x)^{n - 2 - j}
    = n(n - 1)x^2 \sum_{j = 0}^{n - 2} B_{n - 2, j}(x)
    = n(n - 1)x^2,
  \end{align*}
  where we reindexed and used $(1)$ again.
\end{proof}

\begin{remark}
  By dividing by $n$, we can get $(\widetilde{2})$:
  \[
    \sum_{k = 0}^n \frac{k}{n} B_{n, k}(x) = x
  .\]
  If we view $\frac{k}{n}$ as an event and
  then $B_{n, k}(x)$ as a probability, then this tells us
  $x$ is the expectation. We can perform a similar
  process to get an analogous $(\widetilde{3})$:
  \[
    \sum_{k = 0}^n \left(\frac{k}{n} - x\right)^2 B_{n, k}(x) = \frac{x(1 - x)}{n}
  .\]
  Using the same interpretation, we can view this
  as a computation of the variance.
\end{remark}

\section{Weierstrass Approximation Theorem}

\begin{theorem}[Weierstrass approximation theorem]
  Let $f : [a, b] \to \R$ be continuous. Given
  $\epsilon > 0$, there exists a polynomial $p(x)$ such
  that
  \[
    |f(x) - p(x)| < \epsilon
  \]
  for all $x \in [a, b]$.
\end{theorem}

\begin{proof}
  First we can normalize the interval so that
  $f : [0, 1] \to \R$. Then define
  \[
    B_n(f)(x) = \sum_{k = 0}^n f\left(\frac{k}{n}\right) B_{n, k}(x)
  \]
  It suffices to show that for all $\epsilon > 0$,
  there exists $N$ such that for all $n \ge N$ and
  $x \in [0, 1]$, we have $|B_n(x) - f(x)| < \epsilon$.
  Note that this somewhat resembles a Riemann sum.
  \footnote{While the usual Riemann sum gives the integral of $f$ as $n \to \infty$, this construction recovers $f$ itself instead.}

  Now we proceed. Since $f$ is continuous on the compact
  set $[0, 1]$, it is uniformly continuous. In other
  words, for all $\epsilon > 0$, there exists $\delta > 0$
  such that
  \[
  |f(x) - f(y)| < \frac{\epsilon}{2}
  \]
  for all $|x - y| < \delta$. Then we can bound the
  error by
  \begin{align*}
    B_n(x) - f(x)
    &= \sum_{k = 0}^n f\left(\frac{k}{n}\right) B_{n, k}(x) - \sum_{k = 0}^n f(x) B_{n, k}(x)
    = \sum_{k = 0}^n \left[f\left(\frac{k}{n}\right) - f(x)\right] B_{n, k}(x) \\
    &= \sum_{\left|\frac{k}{n} - x\right| < \delta}
    \left[f\left(\frac{k}{n}\right) - f(x)\right] B_{n, k}(x)
    + \sum_{\left|\frac{k}{n} - x\right| \ge \delta}
    \left[f\left(\frac{k}{n}\right) - f(x)\right] B_{n, k}(x).
  \end{align*}
  Here we have used the result that
  $\sum_{k = 0}^n B_{n, k}(x) = 1$. Then we continue with
  \[
    |B_n(x) - f(x)| \le
    \underbrace{\sum_{\left|\frac{k}{n} - x\right| < \delta}
    \left|f\left(\frac{k}{n}\right) - f(x)\right| B_{n, k}(x)}_{(\star)}
    + \underbrace{\sum_{\left|\frac{k}{n} - x\right| \ge \delta}
    \left|f\left(\frac{k}{n}\right) - f(x)\right| B_{n, k}(x)}_{(\star \star)}
  .\]
  For $(\star)$, by uniform continuity we can bound
  \[
    (\star) < \sum_{\left|\frac{k}{n} - x\right| < \delta}
    \frac{\epsilon}{2} B_{n, k}(x)
    = \frac{\epsilon}{2} \sum_{\left|\frac{k}{n} - x\right| < \delta} B_{n, k}(x)
    \le \frac{\epsilon}{2} \sum_{k = 0}^n B_{n, k}(x)
    = \frac{\epsilon}{2}
  .\]
  Then for $(\star \star)$,
  \[
    (\star \star) \le 2 \max_{x \in [0, 1]} |f|
    \sum_{\left|\frac{k}{n} - x\right| \ge \delta} B_{n, k}(x)
  \]
  since $[0, 1]$ is compact. Now by $(\widetilde{3})$,
  \begin{align*}
    \sum_{\left|\frac{k}{n} - x\right| \ge \delta} B_{n, k}(x) &= \sum_{\left|\frac{k}{n} - x\right| \ge \delta}
    \frac{\left(\frac{k}{n} - x\right)^2}{\left(\frac{k}{n} - x\right)^2} B_{n, k}(x)
    \le \frac{1}{\delta^2} \sum_{\left|\frac{k}{n} - x\right| \ge \delta} \left(\frac{k}{n} - x\right)^2 B_{n, k}(x) \\
                                                               &\le \frac{1}{\delta^2} \sum_{k = 0}^n \left(\frac{k}{n} - x\right)^2 B_{n, k}(x)
    = \frac{1}{\delta^2} \frac{x(1 - x)}{n}.
  \end{align*}
  Plugging this back in yields
  \[
    (\star \star) \le 2 \max_{x \in [0, 1]} |f| \frac{1}{\delta^2n}
  \]
  since $x(1 - x) \le 1$. Now
  take
  \[N > \frac{4\max_{x \in [0, 1]}|f|}{\epsilon \delta^2}.\]
  So when $n \ge N$, we have $(\star \star) < \epsilon / 2$.
  Then 
  \[|B_n(x) - f(x)| \le (\star) + (\star \star) < \frac{\epsilon}{2} + \frac{\epsilon}{2} = \epsilon,\]
  which finishes the proof.
\end{proof}
