\chapter{Aug.~31 -- Cantor's Theorem, Sequences}


\section{Cantor's Theorem}

\begin{definition}
  The \textbf{power set} of $A$, denoted $\mathcal{P}(A)$,
  is the collection of all
  subsets of $A$.
\end{definition}

\begin{theorem}[Cantor's theorem]
  Given any set $A$, there does not exist a function
  $f : A \to \mathcal{P}(A)$ which is surjective.
  \footnote{Note that if $\#(A) = n < \infty$, this is
    true as $\#(\mathcal{P}(A)) = 2^n \ne \#(A)$.}
\end{theorem}

\begin{proof}
  Argue by contradiction. Suppose
  $f : A \to \mathcal{P}(A)$ is onto.
  Then for any $a \in A$, $f(a)$ is a subset of $A$.
  Since $f$ is onto, for any subset $B$ of $A$, we
  can find $a \in A$ such that $f(a) = B$. Define
  \[B = \{a \in A : a \notin f(a)\} \subseteq A.\]
  We can find $a' \in A$ such that $f(a') = B$.
  If  $a' \in B$, then $a' \notin f(a') = B$,
  which is a contradition.. If $a' \notin B$,
  this is a contradiction with the definition of
  $B$. Thus such $f$ cannot exist.
\end{proof}

\begin{remark}
  This means that the cardinality of $\mathcal{P}(A)$
  is strictly larger than that of $A$.
\end{remark}

\section{Sequences}
\begin{definition}
  A \textbf{sequence} is a function whose domain
  is $\N$.
\end{definition}

We usually write $\{a_n\}$, $\{x_n\}$
or $(a_n)$, $(x_n)$ to denote sequences.

\begin{example}
  The following
  \[\left\{\frac{1+n}{n}\right\}_{n=1}^\infty = \left\{2, \frac{3}{2}, \frac{4}{3}, \frac{5}{4}, \dots\right\}\]
  is a sequence.
\end{example}

\begin{example}
  $\{a_n\}$, where $a_n = 2^n$ for $n \in \N$ is a sequence.
\end{example}

\begin{example}
  We can also define $\{x_n\}$ recursively by $x_1 = 2$
  and
  \[
    x_{n+1} = \frac{x_n + 1}{2}
  .\]
\end{example}

\begin{remark}
  Sometimes a sequence is also labeled starting from
  $n = 0$.
\end{remark}

\begin{definition}
  A sequence $\{a_n\}$ \textbf{converges} to a real number
  $a$ if for every $\epsilon > 0$, we can find $N \in \N$
  such that for all $n \ge N$, one has
  $|a_n - a| < \epsilon$.
  We write $\lim_{n \to \infty} a_n = a$.
\end{definition}

\begin{remark}
  In analysis, $\epsilon$ is always taken to be a
  positive number.
\end{remark}

\begin{example}
  The sequence $\{1/n\}_{n = 1}^\infty$ converges with
  \[\lim_{n \to \infty} \frac{1}{n} = 0.\]
\end{example}

\begin{definition}
  For $\epsilon > 0$, the
  \textbf{$\epsilon$-neighborhood} of $a$
  is defined to be
  \[V_\epsilon(a) = \{x \in \R : |x - a| < \epsilon\}.\]
\end{definition}

\begin{definition}
  We say that $a$ is the \textbf{limit} of a sequence
  $\{a_n\}$ if for every $\epsilon > 0$,
  $V_\epsilon(a)$ contains all but finitely
  many elements of $\{a_n\}$.
  \footnote{This is the \textit{topological} definition
    of the limit.}
\end{definition}

\begin{remark}
  This definition of the limit is equivalent to the
  definition of convergence.
\end{remark}

\begin{definition}
  A sequence $\{a_n\}$ that does not converge is said
  to be \textbf{divergent}.
\end{definition}

\begin{theorem}
  The limit of a sequence, when it exists, must be
  unique.
\end{theorem}

\begin{exercise}
  Show
  \[\lim_{n \to \infty} \frac{n+1}{n}\]
  exists and
  \[\lim_{n \to \infty} \frac{n+1}{n} = 1.\]
\end{exercise}

\begin{proof}
  We show
  \[\lim_{n \to \infty} \frac{n+1}{n} = 1.\]
  For every $\epsilon > 0$, take $N \in \N$ such that
  $N > \frac{1}{\epsilon}$. We have for all $n \ge N$,
  \[
  \left|\frac{n+1}{n} - 1\right| = \left|\frac{1}{n}\right| \le \frac{1}{N} < \epsilon
  .\]
  Therefore,
  \[
    \lim_{n \to \infty} \frac{n+1}{n} = 1
  .\]
\end{proof}

\subsection{Tips for Showing Limits}
To show the limit of a sequence, take the following steps:
\begin{enumerate}
  \item Identify the limit $a$. This is always given
    by the problem or observation.
  \item $\forall \epsilon > 0$.
  \item Find $N = N(\epsilon)$. Do this in sketch paper
    (need computations and manipulations).
  \item Set $N$ as what is found in (3).
  \item Check that $N$ works.
\end{enumerate}
