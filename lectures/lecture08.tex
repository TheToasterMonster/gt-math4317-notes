\chapter{Sept.~14 -- Series}

\begin{definition}
  Let $\{b_n\}$ be a sequence. An infinite \textbf{series}
  is formally given by
  \[
    \sum_{n = 1}^\infty b_n = b_1 + b_2 + \dots
  .\]
\end{definition}

\begin{definition}
  We define the \textbf{partial sum} of a series by
  \[s_m = \sum_{n = 1}^m b_n.\]
\end{definition}

\section{Convergence of Series}

\begin{definition}
  The series $\sum_{n=1}^\infty{b_n}$ \textbf{converges}
  to $B$ if $\lim_{m \to \infty} s_m = B$. Otherwise
  we say that the series \textbf{diverges}.
\end{definition}

\begin{example}
  Consider the series
  \[\sum_{n = 1}^{\infty} \frac{1}{n^2} = 1 + \frac{1}{2^2} + \frac{1}{3}^2 + \dots.\]
  We look at the partial sums for $m > 1$:
  \begin{align*}
    s_m &= \sum_{n = 1}^{m} \frac{1}{n^2} = 1 + \frac{1}{2^2} + \frac{1}{3^2} + \frac{1}{4^2} + \dots + \frac{1}{m^2}
    \le 1 + \frac{1}{2(1)} + \frac{1}{3(2)} + \frac{1}{4(3)} + \dots + \frac{1}{m(m-1)} \\
        &= 1 + 1 - \frac{1}{2} + \frac{1}{2} - \frac{1}{3} + \frac{1}{3} - \frac{1}{4} + \dots + \frac{1}{m-1} - \frac{1}{m}
        \le 2 - \frac{1}{m}.
  \end{align*}
  Note that $\{s_m\}$ is a monotone sequence and it is
  bounded above by $2$. Thus by the monotone convergence
  theorem, $\{s_m\}$ converges and there is some
  $B \in \R$ such that
  $\lim_{m \to \infty} s_m = B$.
\end{example}

\begin{remark}
  Using some complex analysis, we can find $B$ by way of
  residue calculations.
\end{remark}

\begin{example}
  Consider the harmonic series
  \[\sum_{n = 1}^\infty \frac{1}{n} = 1 + \frac{1}{2} + \frac{1}{3} + \dots.\]
  We look at the partial sums
  \[
  s_m = 1 + \frac{1}{2} + \dots + \frac{1}{m}
  .\]
  Note specifically that
  \begin{align*}
    s_4 &= 1 + \frac{1}{2} + \frac{1}{3} + \frac{1}{4}
    > 1 + \frac{1}{2} + \frac{1}{4} + \frac{1}{4}
    = 1 + \frac{1}{2} + 2\left(\frac{1}{4}\right)
    = 1 + \frac{1}{2} + \frac{1}{2}
    = 1 + 2\left(\frac{1}{2}\right)\\
    s_8 &= 1 + \frac{1}{2} + \frac{1}{3} + \frac{1}{4}
    + \frac{1}{5} + \frac{1}{6} + \frac{1}{7} + \frac{1}{8}
    > 1 + \frac{1}{2} + 2\left(\frac{1}{4}\right) + 4\left(\frac{1}{8}\right)
    = 1 + 3\left(\frac{1}{2}\right)\\
        &\ \ \vdots \\
    s_{2^k} &= 1 + \frac{1}{2} + \frac{1}{3} + \frac{1}{4}
    + \dots + \frac{1}{2^{k - 1} + 1} +
    \frac{1}{2^{k - 1} + 2} + \dots + \frac{1}{2^k}
    > 1 + \frac{k}{2}
  \end{align*}
  Thus $\{s_{2^k}\}$ diverges, so $\{s_m\}$ also
  diverges.
\end{example}

\begin{remark}
  This type of trick (analyzing $2^k$ terms) is called
  \textit{dyadic analysis},
  and it shows up frequently in analysis, particularly
  harmonic analysis.
\end{remark}

\begin{theorem}[Cauchy condensation test]
  Suppose $\{b_n\}$ is decreasing and $b_n \ge 0$ for
  all $n$. Then
  \[\sum_{n = 1}^\infty b_n = b_1 + b_2 + b_3 + \dots\]
  converges if
  and only if
  \[\sum_{n=0}^{\infty} 2^n b_{2^n} = b_1 + 2b_2 + 4b_4 + \dots\]
  converges.
\end{theorem}

\begin{proof}
  First we show the backwards direction.
  Assume $\sum_{n = 0}^\infty 2^n b_{2^n}$ converges.
  Define
  \[t_k = b_1 + \dots + 2^k b_{2^k}.\]
  By assumption, $\{t_k\}$ converges.
  Note that $t_k \ge 0$ and $\sup_k t_k \le M$ since
  convergent series are bounded. Set
  \[s_m = \sum_{n=1}^m b_n.\]
  Fix $m$ and take $k$ large such that $m \le 2^{k + 1} - 1$.
  Then $s_m \le s_{2^{k + 1} - 1}$ since $b_n \ge 0$.
  Observe that
  \begin{align*}
    s_{2^{k+1} - 1} &= b_1 + (b_2 + b_3) +
    (b_4 + b_5 + b_6 + b_7) + \dots
    + (b_{2^k} + \dots + b_{2^{k+1} - 1}) \\
    &\le b_1 + 2b_2 + 4b_4 + \dots + 2^k b_{2^k}.
  \end{align*}
  So $s_m \le s_{2^{k + 1} - 1} \le t_k \le M$.
  Thus $\{s_m\}$ is increasing and bounded, so
  by the monotone convergence theorem,
  $\lim_{m \to \infty} s_m = B \in \R$ exists.

  Now we show the forwards direction. Argue
  by contraposition.
  Suppose $\sum_{n = 0}^\infty 2^n b_{2^n}$
  diverges, then we show that $\sum_{n = 1}^\infty b_n$ also
  diverges. Just need to check that
  $s_{2^k} \ge \frac{1}{2} + k$ (left as an exercise).
\end{proof}

\begin{corollary}
  The series
  \[\sum_{n = 1}^\infty \frac{1}{n^p}\]
  converges if and only if $p > 1$.
\end{corollary}

\begin{proof}
  Let $b_n = \frac{1}{n^p}$ and $b_{2^n} = \frac{1}{2^{np}}$.
  Then we have
  \[\sum_{n=0}^\infty 2^n b_{2^n} = \sum_{n=0}^\infty 2^{(1 - p)n}.\]
  The RHS is a geometric series, which converges if and
  only if $p > 1$. To see this, denote $2^{1 - p} = a$.
  Then we have
  \[\sum_{n = 0}^\infty 2^{(1 - p)n} = \sum_{n = 0}^\infty a^n.\]
  We can observe that the partial sums
  \[t_k = \sum_{n=0}^k a^n = \frac{a^{k+1} - 1}{a - 1}\]
  converges if and only if $a^{k + 1}$ converges.
  This happens if and only if $a < 1$, which happens
  if and only if $p > 1$.
\end{proof}

\section{Properties of Series}
\begin{theorem}[Algebraic limit theorem for series]
  Let
  \[\sum_{n=1}^\infty a_n = A, \quad \sum_{n=1}^\infty b_n = B.\]
  Then for all $c \in \R$, we have
  \[
    \sum_{n=1}^\infty ca_n = cA, \quad
    \sum_{n=1}^\infty (a_n + b_n) = A + B
  .\]
\end{theorem}

\begin{proof}
  Let $\sum_{n=1}^\infty a_n = A$. So
  $s_m = \sum_{n=1}^m a_n$ converges. Set
  $\lim_{n \to \infty} s_m = A$. Define
  \[
    t_m = \sum_{n=1}^m ca_n = c\sum_{n=1}^m a_n = cs_m
  .\]
  Then by the algebraic limit theorem, we have
  $\lim_{m \to \infty} t_m = c\lim_{m \to \infty} s_m = cA$.
\end{proof}

\begin{theorem}[Cauchy criterion for series]
  The series $\sum_{n=1}^\infty a_n$ converges if and only if
  for all $\epsilon > 0$, there exist $N$ such that
  whenever $m, n \ge N$, we have
$|a_{m+1} + \dots + a_n| < \epsilon$.
\end{theorem}

\begin{proof}
  The series $\sum_{k = 1}^\infty a_k$ converges
  if and only if $s_m = \sum_{k = 1}^m a_k$ converges.
  We show that $\{s_m\}$ is a Cauchy sequence. For
  all $\epsilon > 0$, there exists $N$ such
  that for all $m, n \ge N$
  \[|s_n - s_m| = |a_n + \dots + a_{m+1}| < \epsilon.\]
  The converse is the same inequality.
\end{proof}

\begin{corollary}
  If $\sum_{n=1}^\infty a_n$ converges, then
  $\lim_{n \to \infty} a_n = 0$.
\end{corollary}

\begin{proof}
  Take $m = n - 1$.
\end{proof}

\begin{theorem}
  Assume $\{a_n\}$ and $\{b_n\}$ are sequences such that
  $0 \le a_n \le b_n$ for all $n$. Then
  \begin{enumerate}
    \item $\sum_{n=1}^\infty b_n$ converges implies
      $\sum_{n=1}^\infty a_n$ converges,
    \item and $\sum_{n=1}^\infty a_n$ diverges implies
      $\sum_{n=1}^\infty b_n$ diverges.
  \end{enumerate}
\end{theorem}

\begin{proof}
  For all $m, n$, we have
  \[
    |a_{m + 1} + \dots + a_n| \le |b_{m + 1} + \dots + b_n|
  .\]
  Then apply the Cauchy criterion.
\end{proof}

\begin{definition}
  A series is called \textbf{geometric} if it is of the
  form
  \[
    \sum_{k=0}^\infty ar^k = a + ar + ar^2 + \dots
  .\]
\end{definition}

Note that the geometric series diverges when $r = 1$ and
$a \ne 0$. When $r \ne 1$, the partial sums
\[
  s_m = \sum_{k=0}^m ar^k = a\frac{1 - r^{m + 1}}{1 - r}
\]
converge if $|r| < 1$. In this case, as $m \to \infty$, we
have
\[
  s_m \to \frac{a}{1 - r}
.\]
