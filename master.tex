\documentclass[12pt, letterpaper, oneside]{book}
\usepackage[margin=1in]{geometry}
\usepackage{microtype}
\usepackage{kpfonts}
\usepackage{amsmath, amssymb, amsthm}
\usepackage{hyperref}
\usepackage{parskip}
\usepackage[many]{tcolorbox}
\usepackage{footnote}
\usepackage{cancel}
\usepackage{titlesec}
\usepackage{pgffor}
\usepackage[shortlabels]{enumitem}

\renewcommand{\chaptername}{Lecture}
\newtheorem{axiom}{Axiom}[chapter]
\newtheorem{theorem}{Theorem}[chapter]
\newtheorem{corollary}{Corollary}[theorem]
\newtheorem{lemma}{Lemma}[chapter]
\theoremstyle{definition}
\newtheorem{definition}{Definition}[chapter]
\newtheorem{exercise}{Exercise}[chapter]
\newtheorem{example}{Example}[definition]
\newtheorem*{remark}{Remark}

\definecolor{nordblue}{RGB}{135, 160, 190}
\definecolor{nordyellow}{RGB}{230, 204, 147}
\definecolor{darkyellow}{RGB}{207, 184, 132}
\definecolor{nordred}{RGB}{179, 102, 108}
\definecolor{nordgreen}{RGB}{168, 189, 145}
\definecolor{nordpurple}{RGB}{174, 143, 171}
\definecolor{nordteal}{RGB}{148, 190, 206}

\tcbset{sharp corners, breakable, enhanced, parbox=false}
\newtcolorbox{mybox}[3][]
{
  colframe = #2!150,
  colback  = #2!5,
  coltitle = #2!0!white,  
  title    = {#3},
  #1,
}

\titleformat{\chapter}[display]
    {\normalfont\huge\bfseries}{\chaptertitlename\ \thechapter}{20pt}{\Huge}
\titlespacing*{\chapter}{0pt}{0pt}{40pt}

\newcommand{\R}{\mathbb{R}}
\newcommand{\N}{\mathbb{N}}
\newcommand{\Z}{\mathbb{Z}}
\newcommand{\C}{\mathbb{C}}
\newcommand{\Q}{\mathbb{Q}}
\newcommand{\F}{\mathbb{F}}

\title{MATH 4317: Analysis I}
\author{Frank Qiang}
\date{Georgia Institute of Technology, Fall 2023}

\begin{document}
  \maketitle

  \begingroup
  \let\cleardoublepage\clearpage
  \tableofcontents
  \endgroup

  % \foreach \i in {00, 01, 02, 03, 04, ..., 50} {%
  %   \edef\FileName{lectures/lecture\i.tex}%     The % here are necessary to eliminate any
  %   \IfFileExists{\FileName}{%  spurious spaces that may get inserted
  %      \input{\FileName}%       at these points
  %   }
  % }
  \chapter{Aug.~22 -- The Real Numbers}

\section{Number Systems}
We start with the natural numbers
\footnote{$0 \notin \N$ for this class.}
\[\N = \{1, 2, 3, \dots\}.\]
These are perhaps
the most natural in a way, since they are what we use to
count things. They are closed under addition, but
fail when it comes to subtraction. For example,
$1 - 2 = -1 \notin \N$.
So we must expand our number system to the
integers
\[\Z = \{\dots, -3, -2, -1, 0, 1, 2, 3, \dots\}.\]
We can now add, subtract, and
multiply. But we run into problems when we start to
consider quotients. For example,
$1 \div 2 = \frac{1}{2} \notin \Z$.
So we continue to the rational numbers
\[\Q = \left\{\frac{p}{q} : p, q \in \Z,\, q \ne 0\right\}.\]
We now have summation, subtraction,
multiplication, and quotients. But there is still a
problem.

Consider the diagonal of a square with side length $1$.

\begin{theorem}
  $\sqrt{2}$ is not a rational number. 
  \footnote{
    In some sense, this shows that the notion of
    ``rationals'' is strictly weaker than the notion of
    ``length.''
  }
\end{theorem}

\begin{proof}
  Argue by contradiction. Suppose $\sqrt{2}$ is rational.
  Then we can write
   \[
     \sqrt{2} = \frac{p}{q}
  \]
  for some integers $p, q$. Further assume $p$ and
  $q$ have no common factors. Then
  \[2 = \frac{p^2}{q^2} \implies p^2 = 2q^2.\]
  So $p$ is even and we can write $p = 2r$ for some
  $r \in \Z$. Then
  \[4r^2 = 2q^2 \implies 2r^2 = q^2.\]
  So $q$ is also even, and $p, q$ share a common factor
  of $2$. Contradiction.
\end{proof}

Another weakness of $\Q$ is that we cannot take limits
($\Q$ is not complete). For example, note that
\[(\sqrt{2} - 1)(\sqrt{2} + 1) = 2 - 1 = 1,\]
\[
  \sqrt{2} = 1 + \frac{1}{\sqrt{2} + 1}
  = 1 + \frac{1}{1 + 1 + \frac{1}{\sqrt{2} + 1}}
  = \dots
.\]
So if we define the rational sequence
\[
  a_1 = 1, \quad
  a_2 = 1 + \frac{1}{2}, \quad
  a_3 = 1 + \frac{1}{2 + \frac{1}{2}}, \quad
  a_4 = 1 + \frac{1}{2 + \frac{1}{2 + \frac{1}{2}}},
  \quad \dots
,\]
then as $n \to \infty$, $a_n \to \sqrt{2} \notin \Q$.

\section{Sets}
Sets are any collections of objects. Given a set $A$,
we write $x \in A$ if $x$ is an element of $A$. We write
$x \notin A$ otherwise. The \textbf{union} of two sets is
\[
  A \cup B = \{x : x \in A \text{ or } x \in B\}
,\]
and the \textbf{intersection} of two sets is
\[
  A \cap B = \{x : x \in A \text{ and } x \in B\}
.\]
We use the notation
\[\bigcup_{k = 1}^{\infty} A_k\]
to denote the countable union of a family of sets
indexed by $k$.

\section{Functions}
\begin{definition}
  Given two sets $A$ and $B$, a \textbf{function}
  from $A$ to $B$ is a rule, relation, or mapping that
  takes each element $x \in A$ and associates with
  it a single element in $B$. In this case,
  we write $f : A \to B$.
\end{definition}

We call $A$ the \textbf{domain} of $f$ and $B$ the
\textbf{codomain} of $f$. The element in $B$ associated
with $x \in A$ is $f(x)$, called the \textbf{image}
of $x$. The \textbf{range} of $f$ is
\[\text{range}(f) = \{y \in B : y = f(x) \text{ for some }x \in A\}.\]

We say $f$ is:
\begin{enumerate}
  \item \textbf{onto} or \textbf{surjective} if
    $\text{range}(f) = B$.
  \item \textbf{one-to-one} or \textbf{injective} if
    $x, x' \in A$ and $x \ne x'$, then $f(x) \ne f(x')$.
  \item \textbf{bijective} if it is injective and
    surjective.
\end{enumerate}

\begin{tcolorbox}
  First Dirichlet function:
  \[
  g(x) =
  \begin{cases}
    1 & \text{if $x \in \Q$} \\
    0 & \text{if $x \notin \Q$}
  \end{cases}
  =
  \lim_{k \to \infty} \left(\lim_{j \to \infty} \left[\cos(k! \pi x)\right]^{2j}\right)
  .\]
\end{tcolorbox}

\begin{tcolorbox}
  Second Dirichlet function:
  \[
  f(x) =
  \begin{cases}
    \frac{1}{q} & \text{if $x = \frac{p}{q} \in \Q$ in lowest terms} \\
    0 & \text{if $x \notin \Q$}.
  \end{cases}
  \]
\end{tcolorbox}

\begin{tcolorbox}
  Absolute value:
  \[
    |x| =
    \begin{cases}
      x & \text{if $x \ge 0$} \\
      -x & \text{if $x < 0$}.
    \end{cases}
  \]
  Note that we have the following two properties:
  \begin{itemize}
    \item $|xy| = |x||y|$.
    \item $|x + y| \le |x| + |y|$. This is called the
      \textit{triangle inequality}.
  \end{itemize}
\end{tcolorbox}

\section{Induction}
If we have a set $S \subseteq \N$ and
\begin{enumerate}
  \item $1 \in S$
  \item if $n \in S$, then $n + 1 \in S$
\end{enumerate}
then $S = \N$.
\footnote{We always use induction in conjunction with
$\N$.}

  \chapter{Aug.~24 -- The Axiom of Completeness}

  \chapter{Aug.~29 -- Completeness and Countability}

\section{Consequences of Completeness}

\subsection{3rd Consequence: Density of $\Q$ in $\R$}
\begin{theorem}[Density of $\Q$ in $\R$]
For all $a, b \in \R$, $a < b$, there exists
$r \in \Q$ such that $a < r < b$.
\end{theorem}

\begin{proof}
  We want to find $m \in \Z$, $n \in \N$ such that
  \[
  a < \frac{m}{n} < b
  .\]
  By (ii) of the Archimedean property, we can find
  $n \in \N$ such that
  \[
  \frac{1}{n} < b - a
  .\]
  Fix such an $n$. Then let $m$ be the smallest integer
  such that $m - 1 \le na < m$. By construction,
  \[
  \frac{m}{n} - \frac{1}{n} \le a < \frac{m}{n},
  \]
  \[\frac{m}{n} \le a + \frac{1}{n} < b.\]
  Therefore, $a < \frac{m}{n} < b$.
\end{proof}

\begin{corollary}
  For all $a, b \in \Q$, $a < b$, there exists
   $t \in \R \setminus \Q$ such that  $a < t < b$.
\end{corollary}

\subsection{4th Consequence: Existence of $\sqrt{2}$}
\begin{theorem}[Existence of $\sqrt{2}$]
  There exists $s \in \R$, $s > 0$ such that $s^2 = 2$.
\end{theorem}

\begin{proof}
  Define
  \[S = \{x > 0 : x^2 < 2\} \subseteq \R.\]
  $x = 1 \in S$, so $S \ne \emptyset$. $2$ is an upper
  bound for $S$, so $S$ is bounded above. Then by the
  axiom of completeness, $s = \sup S$ exists.
  We claim that $s^2 = 2$. 

  Suppose otherwise that $s^2 < 2$. Then we can find
  $\epsilon > 0$ such that $s + \epsilon \in S$.
  Define $\delta = 2 - s^2 > 0$. Note that
  \[(s + \epsilon)^2 - 2 = s^2 + 2s\epsilon + \epsilon^2 - 2 = -\delta + 2s\epsilon + \epsilon^2.\]
  We know $s \le 2$ since $2$ is an upper bound. 
  Pick
  \[\epsilon = \frac{\delta}{100000000000},\]
  \[
  2s\epsilon + \epsilon \le 4\epsilon + \epsilon^2 < \frac{\delta}{2}
  .\]
  Then
  \[
  (s + \epsilon)^2 - 2 < -\delta + \frac{\delta}{2}
  = -\frac{\delta}{2} < 0
  .\]
  So $s + \epsilon \in S$, which contradicts with
  $s = \sup S$.

  $s^2 > 2$ also leads to a contradiction
  (left as an exercise). Thus we must have $s^2 = 2$.
\end{proof}

\section{Countability}
\begin{definition}
  We say two sets $A$ and $B$ have the same
  \textbf{cardinality}
  if there is a bijection $f : A \to B$.
  We write $A \sim B$.
\end{definition}

\begin{definition}
  We say that a set $A$  is \textbf{finite} if
  $A \sim \{1, 2, \dots, n\}$ for some integer $n$.
  We say that a set $A$ is \textbf{countable}
  (or countably infinite) if
  $A \sim \N$.
  If a set $A$ is not countable, then we say it is
  \textbf{uncountable}.
\end{definition}

\begin{tcolorbox}
  $E = \{2, 4, 6, 8, \dots\}$.

  $E$ is not finite but it is countable: $E \sim \N$.
  We can define $f : \N \to E$ by $f(n) = 2n$.
\end{tcolorbox}

\begin{tcolorbox}
  $\N \sim \Z$. 

  The bijection $f : \N \to \Z$ is given by
  \[
  f(n) =
  \begin{cases}
    \frac{n-1}{2} & \text{$n$ is odd} \\
    -\frac{n}{2} & \text{$n$ is even}.
  \end{cases}
  \]
\end{tcolorbox}

\begin{tcolorbox}
  $(-1, 1) \sim \R$.

  The bijection  $f : (-1, 1) \to \R$ is given by
  \[x \mapsto \frac{x}{x^2 - 1}.\]
\end{tcolorbox}

\begin{theorem}\leavevmode
  \begin{enumerate}
    \item $\Q$ is countable.
    \item  $\R$ is uncountable.
  \end{enumerate}
\end{theorem}

\begin{proof}[Proof of (1)]
  Set $A_1 = \{0\}$ and for $n \ge 2$,
  \[A_n = \left\{\pm \frac{p}{q} : p, q \in \N,\, \text{$p, q$ in lowest terms},\, p + q = n\right\}.\]
  So the first few $A_n$ are:
  \[A_2 = \left\{\frac{1}{1}, \frac{-1}{1}\right\},\]
  \[A_3 = \left\{\frac{1}{2}, \frac{2}{1}, \frac{-1}{2}, \frac{-2}{1}\right\},\]
  etc. Note that $A_n$ is finite and
  for all
  $x \in \Q$, there is an $n \in \N$ such that
  $x \in A_n$.
  We can list elements in $A_1, \dots, A_n$ and label
  them with
  integers in $\N$. Any element of $A_n$ will be
  listed eventually. Then this pairing gives
  a bijection since the $A_n$ are disjoint.
  So $\Q \sim \N$.
\end{proof}

\begin{proof}[Proof of (2)]
  Argue by contradiction. Suppose $f$ is one-to-one from
  $\N \to \R$.
  Set $x_1 = f(1)$, $x_2 = f(2)$, etc. We can write
  \[\R = \{x_1, x_2, \dots\}.\]
  Let $I_1$ be a closed interval such that
  $x_1 \notin I_1$. Pick $I_2 \subseteq I_1$ such that
  $x_2 \notin I_2$. Continue this process such that
  $I_{n+1} \subseteq I_n$ is a closed interval where
  $x_{n+1} \notin I_{n+1}$.
  By construction,
  \[I_1 \supseteq I_2 \supseteq \dots \supseteq I_n \supseteq \dots.\]
  We know that
  \[\bigcap_{n = 1}^\infty I_n \ne \emptyset.\]
  So we can find $n_0$ such that
  \[x_{n_0} \in \bigcap_{n = 1}^\infty I_n.\]
  This is a contradiction with $x_{n_0} \notin I_{n_0}$.
  Thus such an $f$ cannot exist and $\R$ is
  uncountable.
\end{proof}

\begin{theorem}\leavevmode
  \begin{enumerate}
    \item Let $A \subseteq B$. If $B$ is countable, then
      $A$ is either finite or countable.
    \item If $A_n$ is a countable set, then
      \[\bigcup_{n=1}^\infty A_n\]
      is also countable.
  \end{enumerate}
\end{theorem}

\begin{theorem}[Cantor's theorem]
  The open interval
  \[(0, 1) = \{x \in \R : 0 < x < 1\}\]
  is uncountable.
\end{theorem}

\begin{proof}
  Argue by contradiction. Assume $f : \N \to (0, 1)$
  is one-to-one and onto. Then for $m \in \N$,
  we can write (decimal expansion)
  \[f(m) = 0.a_{m1}a_{m2}a_{m3}\ldots \in (0, 1).\]
  For every $m, n \in \N$, $a_{mn} \in \{0, \dots, 9\}$
  is the $n$th digit in the decimal expansion of
  $f(m)$. We can write in a table
  \[1 \quad f(1) \quad a_{11} \quad a_{12} \quad a_{13} \quad \dots\]
  \[2 \quad f(2) \quad a_{21} \quad a_{22} \quad a_{23} \quad \dots\]
  \[3 \quad f(3) \quad a_{31} \quad a_{32} \quad a_{33} \quad \dots\]
  \[\vdots\]
  Take $x = 0.b_1b_2b_3\ldots$ where
  \[
    b_n =
    \begin{cases}
      2 & \text{if $a_{nn} \ne 2$} \\
      3 & \text{if $a_{nn} = 2$}.
    \end{cases}
  \]
  Then $x \ne f(m)$ for any $m \in \N$
  (since $b_m \ne a_{mm}$). This is a
  contradiction.
\end{proof}

  \chapter{Aug.~31 -- Cantor's Theorem, Sequences}


\section{Cantor's Theorem}

\begin{definition}
  The \textbf{power set} of $A$, denoted $\mathcal{P}(A)$,
  is the collection of all
  subsets of $A$.
\end{definition}

\begin{theorem}[Cantor's theorem]
  Given any set $A$, there does not exist a function
  $f : A \to \mathcal{P}(A)$ which is surjective.
  \footnote{Note that if $\#(A) = n < \infty$, this is
    true as $\#(\mathcal{P}(A)) = 2^n \ne \#(A)$.}
\end{theorem}

\begin{proof}
  Argue by contradiction. Suppose
  $f : A \to \mathcal{P}(A)$ is onto.
  Then for any $a \in A$, $f(a)$ is a subset of $A$.
  Since $f$ is onto, for any subset $B$ of $A$, we
  can find $a \in A$ such that $f(a) = B$. Define
  \[B = \{a \in A : a \notin f(a)\} \subseteq A.\]
  We can find $a' \in A$ such that $f(a') = B$.
  If  $a' \in B$, then $a' \notin f(a') = B$,
  which is a contradition.. If $a' \notin B$,
  this is a contradiction with the definition of
  $B$. Thus such $f$ cannot exist.
\end{proof}

\begin{remark}
  This means that the cardinality of $\mathcal{P}(A)$
  is strictly larger than that of $A$.
\end{remark}

\section{Sequences}
\begin{definition}
  A \textbf{sequence} is a function whose domain
  is $\N$.
\end{definition}

We usually write $\{a_n\}$, $\{x_n\}$
or $(a_n)$, $(x_n)$ to denote sequences.

\begin{example}
  The following
  \[\left\{\frac{1+n}{n}\right\}_{n=1}^\infty = \left\{2, \frac{3}{2}, \frac{4}{3}, \frac{5}{4}, \dots\right\}\]
  is a sequence.
\end{example}

\begin{example}
  $\{a_n\}$, where $a_n = 2^n$ for $n \in \N$, is a sequence.
\end{example}

\begin{example}
  We can also define $\{x_n\}$ recursively by $x_1 = 2$
  and
  \[
    x_{n+1} = \frac{x_n + 1}{2}
  .\]
\end{example}

\begin{remark}
  Sometimes a sequence is also labeled starting from
  $n = 0$.
\end{remark}

\subsection{Limits}
\begin{definition}
  A sequence $\{a_n\}$ \textbf{converges} to a real number
  $a$ if for every $\epsilon > 0$, we can find $N \in \N$
  such that for all $n \ge N$, one has
  $|a_n - a| < \epsilon$.
  We write $\lim_{n \to \infty} a_n = a$.
\end{definition}

\begin{remark}
  In analysis, $\epsilon$ is always taken to be a
  positive number.
\end{remark}

\begin{example}
  The sequence $\{1/n\}_{n = 1}^\infty$ converges with
  \[\lim_{n \to \infty} \frac{1}{n} = 0.\]
\end{example}

\begin{definition}
  For $\epsilon > 0$, the
  \textbf{$\epsilon$-neighborhood} of $a$
  is defined to be
  \[V_\epsilon(a) = \{x \in \R : |x - a| < \epsilon\}.\]
\end{definition}

\begin{definition}
  We say that $a$ is the \textbf{limit} of a sequence
  $\{a_n\}$ if for every $\epsilon > 0$,
  $V_\epsilon(a)$ contains all but finitely
  many elements of $\{a_n\}$.
  \footnote{This is the \textit{topological} definition
    of the limit.}
\end{definition}

\begin{remark}
  This definition of the limit is equivalent to the
  definition of convergence.
\end{remark}

\begin{definition}
  A sequence $\{a_n\}$ that does not converge is said
  to be \textbf{divergent}.
\end{definition}

\begin{theorem}
  The limit of a sequence, when it exists, must be
  unique.
\end{theorem}

\begin{proof}
  Homework problem.
\end{proof}

\begin{exercise}
  Show
  \[\lim_{n \to \infty} \frac{n+1}{n}\]
  exists and
  \[\lim_{n \to \infty} \frac{n+1}{n} = 1.\]
\end{exercise}

\begin{proof}
  We show
  \[\lim_{n \to \infty} \frac{n+1}{n} = 1.\]
  For every $\epsilon > 0$, take $N \in \N$ such that
  $N > \frac{1}{\epsilon}$. We have for all $n \ge N$,
  \[
  \left|\frac{n+1}{n} - 1\right| = \left|\frac{1}{n}\right| \le \frac{1}{N} < \epsilon
  .\]
  Therefore,
  \[
    \lim_{n \to \infty} \frac{n+1}{n} = 1
  \]
  as desired.
\end{proof}

\subsection{Tips for Showing Limits}
To show the limit of a sequence, take the following steps:
\begin{enumerate}
  \item Identify the limit $a$. This is always given
    by the problem or observation.
  \item $\forall \epsilon > 0$.
  \item Find $N = N(\epsilon)$. Do this in sketch paper
    (need computations and manipulations).
  \item Set $N$ as what is found in (3).
  \item Check that $N$ works.
\end{enumerate}

  \chapter{Sept.~5 -- Limits and Limit Theorems}

\section{Review of Limits}

\begin{example}
  Find
  \[
    \lim_{n \to \infty} \frac{1 + \sqrt{n}}{\sqrt{n}}
  .\]
\end{example}

\begin{proof}
  We want to show that
  \[
    \lim_{n \to \infty} \frac{1 + \sqrt{n}}{\sqrt{n}} = 1
  .\]
  Fix $\epsilon > 0$ and take $N \in \N$ such that
  $N > \frac{1}{\epsilon^2}$. Then for any $n > N$,
  \[
    \left\lvert \frac{1 + \sqrt{n}}{\sqrt{n}} - 1 \right\rvert \le
    \left\lvert \frac{1}{\sqrt{n}} \right\rvert
    \le \frac{1}{\sqrt{N}} < \epsilon
  ,\]
  as desired.
\end{proof}

How can we understand this using the topological definition?
For all $\epsilon > 0$, take $V_\epsilon(1)$. Pick
$N > \frac{1}{\epsilon^2}$. Then we claim that
$V_\epsilon(1)$ contains all but at most $N$ elements of
$\left\{\frac{\sqrt{n} + 1}{\sqrt{n}}\right\}$. When
$n \ge N$, we have
\[
  \left\lvert \frac{\sqrt{n} + 1}{\sqrt{n}} - 1 \right\rvert
  < \epsilon
,\]
i.e.~$\frac{\sqrt{n} + 1}{\sqrt{n}} \in V_\epsilon(1)$. So
at most $N$ elements might not be in $V_\epsilon(1)$.

\section{Limit Theorems}
\subsection{Algebraic Facts About Limits}
\begin{definition}
  A sequence $\{x_n\}$ is said to be \textbf{bounded} if there
  exists $M$ such that $|x_n| \le M$ for all $n$.
  Alternatively, $\sup_n |x_n| \le M$.
\end{definition}

\begin{theorem}
  Every convergent sequence is bounded.
\end{theorem}

\begin{proof}
  Suppose
  \[
    \lim_{n \to \infty} x_n = l
  .\]
  Take $\epsilon = 1$, we can find $N$ such that for all
  $n \ge N$, $|x_n - l| < 1$. By the triangle inequality,
  $|x_n| < |l| + 1$ for $n \ge N$. Take
  \[M = \max\{|x_1|, |x_2|, \dots, |x_{N - 1}|, |l| + 1\}.\]
  Then $|x_n| \le M$ for all $n \in \N$.
\end{proof}

\begin{theorem}[Algebraic limit theorem]
  If
  \[
    \lim_{n \to \infty} a_n = a \quad \text{and} \quad
    \lim_{n \to \infty} b_n = b
  ,\]
  then for all $c \in \R$,
  \[
    (1)\, \lim_{n \to \infty} ca_n = ca, \quad
    (2)\, \lim_{n \to \infty} (a_n + b_n) = a + b, \quad \text{and} \quad
    (3)\, \lim_{n \to \infty} a_nb_n = ab
  .\]
  Furthermore, if $b \ne 0$, then
  \[
    \lim_{n \to \infty} \frac{a_n}{b_n} = \frac{a}{b} \tag{4}
  .\]
\end{theorem}

\begin{proof}
  (1)\, When $c = 0$, the result is trivial. When $c \ne 0$, for
  all $\epsilon > 0$, we set
  $\epsilon' = \frac{\epsilon}{|c|}$.
  Since $\lim_{n \to \infty} a_n = a$, we can find
  $N_{\epsilon'}$ such that for all $n \ge N_{\epsilon'}$,
  $|a_n - a| < \epsilon'$.
  When $n > N_{\epsilon'}$, we have
  \[
  |ca_n - ca| = |c||a_n - a| < |c|e' = |c| \frac{\epsilon}{|c|} = \epsilon
  .\]
  So $\lim_{n \to \infty} ca_n = ca$.

  (2)\, For all $\epsilon > 0$, since $a_n \to a$ and $b_n \to b$,
  we can find $N_1$ and $N_2$ such that when
  \begin{align*}
    n \ge N_1, &\quad |a_n - a| < \frac{\epsilon}{2}, \\
    n \ge N_2, &\quad |b_n - b| < \frac{\epsilon}{2}.
  \end{align*}
  Take $N = \max\{N_1, N_2\}$. Then for all $n \ge N$,
   \[
  |a_n + b_n - (a + b)| = |a_n - a + b_n - b| \le
  |a_n - a| + |b_n - b| < \frac{\epsilon}{2} + \frac{\epsilon}{2}
  = \epsilon
  .\]
  Therefore $\lim_{n \to \infty} (a_n + b_n) = a + b$.
\end{proof}

\subsection{Order Limit Theorem}
\begin{theorem}[Order limit theorem]
  Let $\{a_n\}$ and $\{b_n\}$ be sequences such that
  \[
    \lim_{n \to \infty} a_n = a \quad \text{and} \quad
    \lim_{n \to \infty} b_n = b
  .\]
  (5)\, If $a_n \ge 0$ for every $n$, then $a \ge 0$.
  (6)\, If $a_n \le b_n$, then $a \le b$.
  (7)\, If $a_n \ge c$, then $a \ge c$.
\end{theorem}

\begin{proof}
  (5)\, Argue by contradiction. Suppose $a < 0$. Take
  $\epsilon = \frac{|a|}{2}$. Since
  $\lim_{n \to \infty} a_n = a$, we can find $N$ such
  that when $n \ge N$, $|a_n - a| < \epsilon$. Note that
  this means
  \[-\epsilon < a_n - a < \epsilon \]
  Then we have
  \[a_n < \epsilon + a = \frac{-a}{2} + a = \frac{a}{2} < 0.\]
  Contradiction.
\end{proof}

\subsection{Monotone Convergence Theorem}

\begin{definition}
  A sequence $\{a_n\}$ is \textbf{increasing} if
  $a_n \le a_{n + 1}$ for every $n$ and \textbf{decreasing}
  if $a_n \ge a_{n + 1}$ for every $n$. A sequence is
  \textbf{monotone} if it is either increasing or decreasing.
\end{definition}

\begin{theorem}[Monotone convergence theorem]
  If a sequence is monotone and bounded, then it converges.
\end{theorem}

\begin{proof}
  Let $\{a_n\}$ be increasing and bounded. Set
  $A = \{a_n : n \in \N\}$.
  Note that $A \ne \varnothing$ and $A$ is bounded. Therefore, by the
  axiom of completeness, $s = \sup A \in \R$ exists.
  Then we claim that $\lim_{n \to \infty} a_n = s$.
  For
  every $\epsilon > 0$, $s - \epsilon$ is not an upper bound
  for $A$, so we can find $N$ such that
  $s - \epsilon < a_{N} \le s$. 
  Since $\{a_n\}$ is increasing, for all $n \ge N$, we know
  $s - \epsilon < a_N \le a_n \le s$,
  i.e.~$|a_n - s| < \epsilon$. Therefore
  $\lim_{n \to \infty} a_n = s$.

  For $\{a_n\}$ decreasing and bounded, simply apply the
  previous result to $\{-a_n\}$.
\end{proof}

\end{document}
