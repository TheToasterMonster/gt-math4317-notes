\documentclass[12pt, letterpaper, oneside]{book}
\usepackage[margin={0.6in, 0.75in}]{geometry}
\usepackage{microtype}
\usepackage{kpfonts}
\usepackage{amsmath, amssymb, amsthm}
\usepackage{hyperref}
\usepackage{parskip}
\usepackage[many]{tcolorbox}
\usepackage{footnote}
\usepackage{cancel}
\usepackage{titlesec}
\usepackage{pgffor}
\usepackage[shortlabels]{enumitem}

\renewcommand{\chaptername}{Lecture}
\newtheorem{axiom}{Axiom}[chapter]
\newtheorem{theorem}{Theorem}[chapter]
\newtheorem{corollary}{Corollary}[theorem]
\newtheorem{lemma}{Lemma}[chapter]
\theoremstyle{definition}
\newtheorem{definition}{Definition}[chapter]
\newtheorem{exercise}{Exercise}[chapter]
\newtheorem{example}{Example}[definition]
\newtheorem*{remark}{Remark}

\tcbset{sharp corners, breakable, enhanced, parbox=false}
\newtcolorbox{mybox}[3][]
{
  colframe = #2!150,
  colback  = #2!5,
  coltitle = #2!0!white,  
  title    = {#3},
  #1,
}

\titleformat{\chapter}[display]
    {\normalfont\huge\bfseries}{\chaptertitlename\ \thechapter}{20pt}{\Huge}
\titlespacing*{\chapter}{0pt}{0pt}{40pt}

\newcommand{\R}{\mathbb{R}}
\newcommand{\N}{\mathbb{N}}
\newcommand{\Z}{\mathbb{Z}}
\newcommand{\C}{\mathbb{C}}
\newcommand{\Q}{\mathbb{Q}}
\newcommand{\F}{\mathbb{F}}

\title{MATH 4317: Analysis I}
\author{Frank Qiang}
\date{Georgia Institute of Technology, Fall 2023}

\begin{document}
  \maketitle

  \begingroup
  \let\cleardoublepage\clearpage
  \tableofcontents
  \endgroup

  % \foreach \i in {00, 01, 02, 03, 04, ..., 50} {%
  %   \edef\FileName{lectures/lecture\i.tex}%     The % here are necessary to eliminate any
  %   \IfFileExists{\FileName}{%  spurious spaces that may get inserted
  %      \input{\FileName}%       at these points
  %   }
  % }
  \chapter{Aug.~22 -- The Real Numbers}

\section{Number Systems}
We start with the natural numbers
\footnote{$0 \notin \N$ for this class.}
\[\N = \{1, 2, 3, \dots\}.\]
These are perhaps
the most natural in a way, since they are what we use to
count things. They are closed under addition, but
fail when it comes to subtraction. For example,
$1 - 2 = -1 \notin \N$.
So we must expand our number system to the
integers
\[\Z = \{\dots, -3, -2, -1, 0, 1, 2, 3, \dots\}.\]
We can now add, subtract, and
multiply. But we run into problems when we start to
consider quotients. For example,
$1 \div 2 = \frac{1}{2} \notin \Z$.
So we continue to the rational numbers
\[\Q = \left\{\frac{p}{q} : p, q \in \Z,\, q \ne 0\right\}.\]
We now have summation, subtraction,
multiplication, and quotients. But there is still a
problem.

Consider the diagonal of a square with side length $1$.

\begin{theorem}
  $\sqrt{2}$ is not a rational number. 
  \footnote{
    In some sense, this shows that the notion of
    ``rationals'' is strictly weaker than the notion of
    ``length.''
  }
\end{theorem}

\begin{proof}
  Argue by contradiction. Suppose $\sqrt{2}$ is rational.
  Then we can write
   \[
     \sqrt{2} = \frac{p}{q}
  \]
  for some integers $p, q$. Further assume $p$ and
  $q$ have no common factors. Then
  \[2 = \frac{p^2}{q^2} \implies p^2 = 2q^2.\]
  So $p$ is even and we can write $p = 2r$ for some
  $r \in \Z$. Then
  \[4r^2 = 2q^2 \implies 2r^2 = q^2.\]
  So $q$ is also even, and $p, q$ share a common factor
  of $2$. Contradiction.
\end{proof}

Another weakness of $\Q$ is that we cannot take limits
($\Q$ is not complete). For example, note that
\[(\sqrt{2} - 1)(\sqrt{2} + 1) = 2 - 1 = 1,\]
\[
  \sqrt{2} = 1 + \frac{1}{\sqrt{2} + 1}
  = 1 + \frac{1}{1 + 1 + \frac{1}{\sqrt{2} + 1}}
  = \dots
.\]
So if we define the rational sequence
\[
  a_1 = 1, \quad
  a_2 = 1 + \frac{1}{2}, \quad
  a_3 = 1 + \frac{1}{2 + \frac{1}{2}}, \quad
  a_4 = 1 + \frac{1}{2 + \frac{1}{2 + \frac{1}{2}}},
  \quad \dots
,\]
then as $n \to \infty$, $a_n \to \sqrt{2} \notin \Q$.

\section{Sets}
Sets are any collections of objects. Given a set $A$,
we write $x \in A$ if $x$ is an element of $A$. We write
$x \notin A$ otherwise. The \textbf{union} of two sets is
\[
  A \cup B = \{x : x \in A \text{ or } x \in B\}
,\]
and the \textbf{intersection} of two sets is
\[
  A \cap B = \{x : x \in A \text{ and } x \in B\}
.\]
We use the notation
\[\bigcup_{k = 1}^{\infty} A_k\]
to denote the countable union of a family of sets
indexed by $k$.

\section{Functions}
\begin{definition}
  Given two sets $A$ and $B$, a \textbf{function}
  from $A$ to $B$ is a rule, relation, or mapping that
  takes each element $x \in A$ and associates with
  it a single element in $B$. In this case,
  we write $f : A \to B$.
\end{definition}

We call $A$ the \textbf{domain} of $f$ and $B$ the
\textbf{codomain} of $f$. The element in $B$ associated
with $x \in A$ is $f(x)$, called the \textbf{image}
of $x$. The \textbf{range} of $f$ is
\[\text{range}(f) = \{y \in B : y = f(x) \text{ for some }x \in A\}.\]

We say $f$ is:
\begin{enumerate}
  \item \textbf{onto} or \textbf{surjective} if
    $\text{range}(f) = B$.
  \item \textbf{one-to-one} or \textbf{injective} if
    $x, x' \in A$ and $x \ne x'$, then $f(x) \ne f(x')$.
  \item \textbf{bijective} if it is injective and
    surjective.
\end{enumerate}

\begin{tcolorbox}
  First Dirichlet function:
  \[
  g(x) =
  \begin{cases}
    1 & \text{if $x \in \Q$} \\
    0 & \text{if $x \notin \Q$}
  \end{cases}
  =
  \lim_{k \to \infty} \left(\lim_{j \to \infty} \left[\cos(k! \pi x)\right]^{2j}\right)
  .\]
\end{tcolorbox}

\begin{tcolorbox}
  Second Dirichlet function:
  \[
  f(x) =
  \begin{cases}
    \frac{1}{q} & \text{if $x = \frac{p}{q} \in \Q$ in lowest terms} \\
    0 & \text{if $x \notin \Q$}.
  \end{cases}
  \]
\end{tcolorbox}

\begin{tcolorbox}
  Absolute value:
  \[
    |x| =
    \begin{cases}
      x & \text{if $x \ge 0$} \\
      -x & \text{if $x < 0$}.
    \end{cases}
  \]
  Note that we have the following two properties:
  \begin{itemize}
    \item $|xy| = |x||y|$.
    \item $|x + y| \le |x| + |y|$. This is called the
      \textit{triangle inequality}.
  \end{itemize}
\end{tcolorbox}

\section{Induction}
If we have a set $S \subseteq \N$ and
\begin{enumerate}
  \item $1 \in S$
  \item if $n \in S$, then $n + 1 \in S$
\end{enumerate}
then $S = \N$.
\footnote{We always use induction in conjunction with
$\N$.}

  \chapter{Aug.~24 -- The Axiom of Completeness}

  \chapter{Aug.~29 -- Completeness and Countability}

\section{Consequences of Completeness}

\subsection{3rd Consequence: Density of $\Q$ in $\R$}
\begin{theorem}[Density of $\Q$ in $\R$]
For all $a, b \in \R$, $a < b$, there exists
$r \in \Q$ such that $a < r < b$.
\end{theorem}

\begin{proof}
  We want to find $m \in \Z$, $n \in \N$ such that
  \[
  a < \frac{m}{n} < b
  .\]
  By (ii) of the Archimedean property, we can find
  $n \in \N$ such that
  \[
  \frac{1}{n} < b - a
  .\]
  Fix such an $n$. Then let $m$ be the smallest integer
  such that $m - 1 \le na < m$. By construction,
  \[
  \frac{m}{n} - \frac{1}{n} \le a < \frac{m}{n},
  \]
  \[\frac{m}{n} \le a + \frac{1}{n} < b.\]
  Therefore, $a < \frac{m}{n} < b$.
\end{proof}

\begin{corollary}
  For all $a, b \in \Q$, $a < b$, there exists
   $t \in \R \setminus \Q$ such that  $a < t < b$.
\end{corollary}

\subsection{4th Consequence: Existence of $\sqrt{2}$}
\begin{theorem}[Existence of $\sqrt{2}$]
  There exists $s \in \R$, $s > 0$ such that $s^2 = 2$.
\end{theorem}

\begin{proof}
  Define
  \[S = \{x > 0 : x^2 < 2\} \subseteq \R.\]
  $x = 1 \in S$, so $S \ne \emptyset$. $2$ is an upper
  bound for $S$, so $S$ is bounded above. Then by the
  axiom of completeness, $s = \sup S$ exists.
  We claim that $s^2 = 2$. 

  Suppose otherwise that $s^2 < 2$. Then we can find
  $\epsilon > 0$ such that $s + \epsilon \in S$.
  Define $\delta = 2 - s^2 > 0$. Note that
  \[(s + \epsilon)^2 - 2 = s^2 + 2s\epsilon + \epsilon^2 - 2 = -\delta + 2s\epsilon + \epsilon^2.\]
  We know $s \le 2$ since $2$ is an upper bound. 
  Pick
  \[\epsilon = \frac{\delta}{100000000000},\]
  \[
  2s\epsilon + \epsilon \le 4\epsilon + \epsilon^2 < \frac{\delta}{2}
  .\]
  Then
  \[
  (s + \epsilon)^2 - 2 < -\delta + \frac{\delta}{2}
  = -\frac{\delta}{2} < 0
  .\]
  So $s + \epsilon \in S$, which contradicts with
  $s = \sup S$.

  $s^2 > 2$ also leads to a contradiction
  (left as an exercise). Thus we must have $s^2 = 2$.
\end{proof}

\section{Countability}
\begin{definition}
  We say two sets $A$ and $B$ have the same
  \textbf{cardinality}
  if there is a bijection $f : A \to B$.
  We write $A \sim B$.
\end{definition}

\begin{definition}
  We say that a set $A$  is \textbf{finite} if
  $A \sim \{1, 2, \dots, n\}$ for some integer $n$.
  We say that a set $A$ is \textbf{countable}
  (or countably infinite) if
  $A \sim \N$.
  If a set $A$ is not countable, then we say it is
  \textbf{uncountable}.
\end{definition}

\begin{tcolorbox}
  $E = \{2, 4, 6, 8, \dots\}$.

  $E$ is not finite but it is countable: $E \sim \N$.
  We can define $f : \N \to E$ by $f(n) = 2n$.
\end{tcolorbox}

\begin{tcolorbox}
  $\N \sim \Z$. 

  The bijection $f : \N \to \Z$ is given by
  \[
  f(n) =
  \begin{cases}
    \frac{n-1}{2} & \text{$n$ is odd} \\
    -\frac{n}{2} & \text{$n$ is even}.
  \end{cases}
  \]
\end{tcolorbox}

\begin{tcolorbox}
  $(-1, 1) \sim \R$.

  The bijection  $f : (-1, 1) \to \R$ is given by
  \[x \mapsto \frac{x}{x^2 - 1}.\]
\end{tcolorbox}

\begin{theorem}\leavevmode
  \begin{enumerate}
    \item $\Q$ is countable.
    \item  $\R$ is uncountable.
  \end{enumerate}
\end{theorem}

\begin{proof}[Proof of (1)]
  Set $A_1 = \{0\}$ and for $n \ge 2$,
  \[A_n = \left\{\pm \frac{p}{q} : p, q \in \N,\, \text{$p, q$ in lowest terms},\, p + q = n\right\}.\]
  So the first few $A_n$ are:
  \[A_2 = \left\{\frac{1}{1}, \frac{-1}{1}\right\},\]
  \[A_3 = \left\{\frac{1}{2}, \frac{2}{1}, \frac{-1}{2}, \frac{-2}{1}\right\},\]
  etc. Note that $A_n$ is finite and
  for all
  $x \in \Q$, there is an $n \in \N$ such that
  $x \in A_n$.
  We can list elements in $A_1, \dots, A_n$ and label
  them with
  integers in $\N$. Any element of $A_n$ will be
  listed eventually. Then this pairing gives
  a bijection since the $A_n$ are disjoint.
  So $\Q \sim \N$.
\end{proof}

\begin{proof}[Proof of (2)]
  Argue by contradiction. Suppose $f$ is one-to-one from
  $\N \to \R$.
  Set $x_1 = f(1)$, $x_2 = f(2)$, etc. We can write
  \[\R = \{x_1, x_2, \dots\}.\]
  Let $I_1$ be a closed interval such that
  $x_1 \notin I_1$. Pick $I_2 \subseteq I_1$ such that
  $x_2 \notin I_2$. Continue this process such that
  $I_{n+1} \subseteq I_n$ is a closed interval where
  $x_{n+1} \notin I_{n+1}$.
  By construction,
  \[I_1 \supseteq I_2 \supseteq \dots \supseteq I_n \supseteq \dots.\]
  We know that
  \[\bigcap_{n = 1}^\infty I_n \ne \emptyset.\]
  So we can find $n_0$ such that
  \[x_{n_0} \in \bigcap_{n = 1}^\infty I_n.\]
  This is a contradiction with $x_{n_0} \notin I_{n_0}$.
  Thus such an $f$ cannot exist and $\R$ is
  uncountable.
\end{proof}

\begin{theorem}\leavevmode
  \begin{enumerate}
    \item Let $A \subseteq B$. If $B$ is countable, then
      $A$ is either finite or countable.
    \item If $A_n$ is a countable set, then
      \[\bigcup_{n=1}^\infty A_n\]
      is also countable.
  \end{enumerate}
\end{theorem}

\begin{theorem}[Cantor's theorem]
  The open interval
  \[(0, 1) = \{x \in \R : 0 < x < 1\}\]
  is uncountable.
\end{theorem}

\begin{proof}
  Argue by contradiction. Assume $f : \N \to (0, 1)$
  is one-to-one and onto. Then for $m \in \N$,
  we can write (decimal expansion)
  \[f(m) = 0.a_{m1}a_{m2}a_{m3}\ldots \in (0, 1).\]
  For every $m, n \in \N$, $a_{mn} \in \{0, \dots, 9\}$
  is the $n$th digit in the decimal expansion of
  $f(m)$. We can write in a table
  \[1 \quad f(1) \quad a_{11} \quad a_{12} \quad a_{13} \quad \dots\]
  \[2 \quad f(2) \quad a_{21} \quad a_{22} \quad a_{23} \quad \dots\]
  \[3 \quad f(3) \quad a_{31} \quad a_{32} \quad a_{33} \quad \dots\]
  \[\vdots\]
  Take $x = 0.b_1b_2b_3\ldots$ where
  \[
    b_n =
    \begin{cases}
      2 & \text{if $a_{nn} \ne 2$} \\
      3 & \text{if $a_{nn} = 2$}.
    \end{cases}
  \]
  Then $x \ne f(m)$ for any $m \in \N$
  (since $b_m \ne a_{mm}$). This is a
  contradiction.
\end{proof}

  \chapter{Aug.~31 -- Cantor's Theorem, Sequences}


\section{Cantor's Theorem}

\begin{definition}
  The \textbf{power set} of $A$, denoted $\mathcal{P}(A)$,
  is the collection of all
  subsets of $A$.
\end{definition}

\begin{theorem}[Cantor's theorem]
  Given any set $A$, there does not exist a function
  $f : A \to \mathcal{P}(A)$ which is surjective.
  \footnote{Note that if $\#(A) = n < \infty$, this is
    true as $\#(\mathcal{P}(A)) = 2^n \ne \#(A)$.}
\end{theorem}

\begin{proof}
  Argue by contradiction. Suppose
  $f : A \to \mathcal{P}(A)$ is onto.
  Then for any $a \in A$, $f(a)$ is a subset of $A$.
  Since $f$ is onto, for any subset $B$ of $A$, we
  can find $a \in A$ such that $f(a) = B$. Define
  \[B = \{a \in A : a \notin f(a)\} \subseteq A.\]
  We can find $a' \in A$ such that $f(a') = B$.
  If  $a' \in B$, then $a' \notin f(a') = B$,
  which is a contradition.. If $a' \notin B$,
  this is a contradiction with the definition of
  $B$. Thus such $f$ cannot exist.
\end{proof}

\begin{remark}
  This means that the cardinality of $\mathcal{P}(A)$
  is strictly larger than that of $A$.
\end{remark}

\section{Sequences}
\begin{definition}
  A \textbf{sequence} is a function whose domain
  is $\N$.
\end{definition}

We usually write $\{a_n\}$, $\{x_n\}$
or $(a_n)$, $(x_n)$ to denote sequences.

\begin{example}
  The following
  \[\left\{\frac{1+n}{n}\right\}_{n=1}^\infty = \left\{2, \frac{3}{2}, \frac{4}{3}, \frac{5}{4}, \dots\right\}\]
  is a sequence.
\end{example}

\begin{example}
  $\{a_n\}$, where $a_n = 2^n$ for $n \in \N$, is a sequence.
\end{example}

\begin{example}
  We can also define $\{x_n\}$ recursively by $x_1 = 2$
  and
  \[
    x_{n+1} = \frac{x_n + 1}{2}
  .\]
\end{example}

\begin{remark}
  Sometimes a sequence is also labeled starting from
  $n = 0$.
\end{remark}

\subsection{Limits}
\begin{definition}
  A sequence $\{a_n\}$ \textbf{converges} to a real number
  $a$ if for every $\epsilon > 0$, we can find $N \in \N$
  such that for all $n \ge N$, one has
  $|a_n - a| < \epsilon$.
  We write $\lim_{n \to \infty} a_n = a$.
\end{definition}

\begin{remark}
  In analysis, $\epsilon$ is always taken to be a
  positive number.
\end{remark}

\begin{example}
  The sequence $\{1/n\}_{n = 1}^\infty$ converges with
  \[\lim_{n \to \infty} \frac{1}{n} = 0.\]
\end{example}

\begin{definition}
  For $\epsilon > 0$, the
  \textbf{$\epsilon$-neighborhood} of $a$
  is defined to be
  \[V_\epsilon(a) = \{x \in \R : |x - a| < \epsilon\}.\]
\end{definition}

\begin{definition}
  We say that $a$ is the \textbf{limit} of a sequence
  $\{a_n\}$ if for every $\epsilon > 0$,
  $V_\epsilon(a)$ contains all but finitely
  many elements of $\{a_n\}$.
  \footnote{This is the \textit{topological} definition
    of the limit.}
\end{definition}

\begin{remark}
  This definition of the limit is equivalent to the
  definition of convergence.
\end{remark}

\begin{definition}
  A sequence $\{a_n\}$ that does not converge is said
  to be \textbf{divergent}.
\end{definition}

\begin{theorem}
  The limit of a sequence, when it exists, must be
  unique.
\end{theorem}

\begin{proof}
  Homework problem.
\end{proof}

\begin{exercise}
  Show
  \[\lim_{n \to \infty} \frac{n+1}{n}\]
  exists and
  \[\lim_{n \to \infty} \frac{n+1}{n} = 1.\]
\end{exercise}

\begin{proof}
  We show
  \[\lim_{n \to \infty} \frac{n+1}{n} = 1.\]
  For every $\epsilon > 0$, take $N \in \N$ such that
  $N > \frac{1}{\epsilon}$. We have for all $n \ge N$,
  \[
  \left|\frac{n+1}{n} - 1\right| = \left|\frac{1}{n}\right| \le \frac{1}{N} < \epsilon
  .\]
  Therefore,
  \[
    \lim_{n \to \infty} \frac{n+1}{n} = 1
  \]
  as desired.
\end{proof}

\subsection{Tips for Showing Limits}
To show the limit of a sequence, take the following steps:
\begin{enumerate}
  \item Identify the limit $a$. This is always given
    by the problem or observation.
  \item $\forall \epsilon > 0$.
  \item Find $N = N(\epsilon)$. Do this in sketch paper
    (need computations and manipulations).
  \item Set $N$ as what is found in (3).
  \item Check that $N$ works.
\end{enumerate}

  \chapter{Sept.~5 -- Limits and Limit Theorems}

\section{Review of Limits}

\begin{example}
  Find
  \[
    \lim_{n \to \infty} \frac{1 + \sqrt{n}}{\sqrt{n}}
  .\]
\end{example}

\begin{proof}
  We want to show that
  \[
    \lim_{n \to \infty} \frac{1 + \sqrt{n}}{\sqrt{n}} = 1
  .\]
  Fix $\epsilon > 0$ and take $N \in \N$ such that
  $N > \frac{1}{\epsilon^2}$. Then for any $n > N$,
  \[
    \left\lvert \frac{1 + \sqrt{n}}{\sqrt{n}} - 1 \right\rvert \le
    \left\lvert \frac{1}{\sqrt{n}} \right\rvert
    \le \frac{1}{\sqrt{N}} < \epsilon
  ,\]
  as desired.
\end{proof}

How can we understand this using the topological definition?
For all $\epsilon > 0$, take $V_\epsilon(1)$. Pick
$N > \frac{1}{\epsilon^2}$. Then we claim that
$V_\epsilon(1)$ contains all but at most $N$ elements of
$\left\{\frac{\sqrt{n} + 1}{\sqrt{n}}\right\}$. When
$n \ge N$, we have
\[
  \left\lvert \frac{\sqrt{n} + 1}{\sqrt{n}} - 1 \right\rvert
  < \epsilon
,\]
i.e.~$\frac{\sqrt{n} + 1}{\sqrt{n}} \in V_\epsilon(1)$. So
at most $N$ elements might not be in $V_\epsilon(1)$.

\section{Limit Theorems}
\subsection{Algebraic Facts About Limits}
\begin{definition}
  A sequence $\{x_n\}$ is said to be \textbf{bounded} if there
  exists $M$ such that $|x_n| \le M$ for all $n$.
  Alternatively, $\sup_n |x_n| \le M$.
\end{definition}

\begin{theorem}
  Every convergent sequence is bounded.
\end{theorem}

\begin{proof}
  Suppose
  \[
    \lim_{n \to \infty} x_n = l
  .\]
  Take $\epsilon = 1$, we can find $N$ such that for all
  $n \ge N$, $|x_n - l| < 1$. By the triangle inequality,
  $|x_n| < |l| + 1$ for $n \ge N$. Take
  \[M = \max\{|x_1|, |x_2|, \dots, |x_{N - 1}|, |l| + 1\}.\]
  Then $|x_n| \le M$ for all $n \in \N$.
\end{proof}

\begin{theorem}[Algebraic limit theorem]
  If
  \[
    \lim_{n \to \infty} a_n = a \quad \text{and} \quad
    \lim_{n \to \infty} b_n = b
  ,\]
  then for all $c \in \R$,
  \[
    (1)\, \lim_{n \to \infty} ca_n = ca, \quad
    (2)\, \lim_{n \to \infty} (a_n + b_n) = a + b, \quad \text{and} \quad
    (3)\, \lim_{n \to \infty} a_nb_n = ab
  .\]
  Furthermore, if $b \ne 0$, then
  \[
    \lim_{n \to \infty} \frac{a_n}{b_n} = \frac{a}{b} \tag{4}
  .\]
\end{theorem}

\begin{proof}
  (1)\, When $c = 0$, the result is trivial. When $c \ne 0$, for
  all $\epsilon > 0$, we set
  $\epsilon' = \frac{\epsilon}{|c|}$.
  Since $\lim_{n \to \infty} a_n = a$, we can find
  $N_{\epsilon'}$ such that for all $n \ge N_{\epsilon'}$,
  $|a_n - a| < \epsilon'$.
  When $n > N_{\epsilon'}$, we have
  \[
  |ca_n - ca| = |c||a_n - a| < |c|e' = |c| \frac{\epsilon}{|c|} = \epsilon
  .\]
  So $\lim_{n \to \infty} ca_n = ca$.

  (2)\, For all $\epsilon > 0$, since $a_n \to a$ and $b_n \to b$,
  we can find $N_1$ and $N_2$ such that when
  \begin{align*}
    n \ge N_1, &\quad |a_n - a| < \frac{\epsilon}{2}, \\
    n \ge N_2, &\quad |b_n - b| < \frac{\epsilon}{2}.
  \end{align*}
  Take $N = \max\{N_1, N_2\}$. Then for all $n \ge N$,
   \[
  |a_n + b_n - (a + b)| = |a_n - a + b_n - b| \le
  |a_n - a| + |b_n - b| < \frac{\epsilon}{2} + \frac{\epsilon}{2}
  = \epsilon
  .\]
  Therefore $\lim_{n \to \infty} (a_n + b_n) = a + b$.
\end{proof}

\subsection{Order Limit Theorem}
\begin{theorem}[Order limit theorem]
  Let $\{a_n\}$ and $\{b_n\}$ be sequences such that
  \[
    \lim_{n \to \infty} a_n = a \quad \text{and} \quad
    \lim_{n \to \infty} b_n = b
  .\]
  (5)\, If $a_n \ge 0$ for every $n$, then $a \ge 0$.
  (6)\, If $a_n \le b_n$, then $a \le b$.
  (7)\, If $a_n \ge c$, then $a \ge c$.
\end{theorem}

\begin{proof}
  (5)\, Argue by contradiction. Suppose $a < 0$. Take
  $\epsilon = \frac{|a|}{2}$. Since
  $\lim_{n \to \infty} a_n = a$, we can find $N$ such
  that when $n \ge N$, $|a_n - a| < \epsilon$. Note that
  this means
  \[-\epsilon < a_n - a < \epsilon \]
  Then we have
  \[a_n < \epsilon + a = \frac{-a}{2} + a = \frac{a}{2} < 0.\]
  Contradiction.
\end{proof}

\subsection{Monotone Convergence Theorem}

\begin{definition}
  A sequence $\{a_n\}$ is \textbf{increasing} if
  $a_n \le a_{n + 1}$ for every $n$ and \textbf{decreasing}
  if $a_n \ge a_{n + 1}$ for every $n$. A sequence is
  \textbf{monotone} if it is either increasing or decreasing.
\end{definition}

\begin{theorem}[Monotone convergence theorem]
  If a sequence is monotone and bounded, then it converges.
\end{theorem}

\begin{proof}
  Let $\{a_n\}$ be increasing and bounded. Set
  $A = \{a_n : n \in \N\}$.
  Note that $A \ne \varnothing$ and $A$ is bounded. Therefore, by the
  axiom of completeness, $s = \sup A \in \R$ exists.
  Then we claim that $\lim_{n \to \infty} a_n = s$.
  For
  every $\epsilon > 0$, $s - \epsilon$ is not an upper bound
  for $A$, so we can find $N$ such that
  $s - \epsilon < a_{N} \le s$. 
  Since $\{a_n\}$ is increasing, for all $n \ge N$, we know
  $s - \epsilon < a_N \le a_n \le s$,
  i.e.~$|a_n - s| < \epsilon$. Therefore
  $\lim_{n \to \infty} a_n = s$.

  For $\{a_n\}$ decreasing and bounded, simply apply the
  previous result to $\{-a_n\}$.
\end{proof}

  \chapter{Sept.~7 -- Bolzano-Weierstrass Theorem}

\section{Review of Limits}

\begin{theorem}[Squeeze theorem]
  Let $\{x_n\}, \{y_n\}, \{z_n\}$ be sequences such
  that $x_n \le y_n \le z_n$ for all $n$, and suppose that
  \[\lim_{n \to \infty} x_n = \lim_{n \to \infty} z_n = l.\]
  Then $\lim_{n \to \infty} y_n = l$.
\end{theorem}

\begin{proof}
  Consider $|y_n - l|$.
  If
  \begin{align*}
    y_n - l \ge 0, &\quad \text{then} \quad y_n - l \le z_n - l, \\
    y_n - l < 0, &\quad \text{then} \quad |y_n - l| = l - y_n \le l - x_n.
  \end{align*}
  So we have
  \[
  |y_n - l| \le |z_n - l| + |x_n - l|
  .\]
  For all $\epsilon > 0$, there exist $N_1, N_2$ such that
  for all $n \ge \N_1$,
  \[
  |z_n - l| < \frac{\epsilon}{2},
  \]
  and for all $n \ge N_2$, \[
  |x_n - l| < \frac{\epsilon}{2}
  .\]
  Take $N = \max\{N_1, N_2\}$. If $n \ge N$, then
  \[
  |y_n - l| \le |z_n - l| + |x_n - l| <
  \frac{\epsilon}{2} + \frac{\epsilon}{2} = \epsilon
  .\]
  So $\lim_{n \to \infty} y_n = l$.
\end{proof}

\section{Subsequences and the Bolzano-Weierstrass Theorem}
\begin{definition}
  Let $\{a_n\}$ be a sequence of real numbers.
  Let $n_1 < n_2 < n_3 < \dots$ be an increasing sequence of
  natural numbers. Then $\{a_{n_1}, a_{n_2}, \dots, \}$
  is a \textbf{subsequence} of $\{a_n\}$, and it is
  denoted by $\{a_{n_k}\}$.
\end{definition}

\begin{example}
  Let
  \[\{a_n\} = \left\{1, \frac{1}{2}, \frac{1}{3}, \frac{1}{4}, \dots\right\}.\]
  Then 
  \[\left\{\frac{1}{2}, \frac{1}{4}, \frac{1}{6}, \frac{1}{8}, \dots\right\}\]
  is a subsequence of $\{a_n\}$.
  However, note that
  \[\left\{\frac{1}{10}, \frac{1}{5}, \frac{1}{100}, \frac{1}{500}, \dots\right\}\]
  is \textit{not} a subsequence of $\{a_n\}$ since the
  the $n_k$ are not strictly increasing. Similarly,
  \[\left\{1, \frac{1}{3}, \frac{1}{3}, \frac{1}{5}, \frac{1}{5} \dots\right\}\]
  is also not a subsequence of $\{a_n\}$.
\end{example}

\begin{theorem}
  \label{thm:subseq}
  Subsequences of a convergent sequence converge to the
  same limit as the original sequence.
\end{theorem}

\begin{proof}
  Suppose $\lim_{n \to \infty} a_n = a$. So for every
  $\epsilon > 0$, there exists $N$ such that
  $|a_n - a| < \epsilon$ for all $n \ge N$.
  Consider an arbitrary subsequence $\{a_{n_k}\}$.
  Note that $n_k \ge k$. So when $k \ge N$,
  \[|a_{n_k} - a| < \epsilon.\]
  Therefore $\lim_{k \to \infty} a_{n_k} = a$.
\end{proof}

\begin{example}
  Let $0 < b < 1$. Clearly
  \[1 > b > b^2 > b^3 > b^4 > \dots \ge 0.\]
  The sequence $\{b^n\}$ is decreasing and bounded below,
  so by the monotone convergence theorem,
  $\lim_{n \to \infty} b^n = l \in \R$ exists. Note that
  $\{b^{2n}\}$ is a subsequence of $\{b^n\}$, so
  by Theorem \ref{thm:subseq}, we have
  $\lim_{n \to \infty} b^{2n} = l$.
  Note that $b^{2n} = b^n b^n$. By the algebraic limit
  theorem,
  \[\lim_{n \to \infty} b^{2n} = \left(\lim_{n \to \infty} b^n\right)\left(\lim_{n \to \infty} b^n\right).\]
  Therefore, $l = l^2$, so we have $l = 0$ or $l = 1$. But
  the entire sequence is strictly less than $1$ and
  decreasing, so $l = 0$.
\end{example}

\begin{example}
  Consider the sequence
  \[\{(-1)^n\} = \{-1, 1, -1, 1, \dots\}.\]
  This sequence does not converge. But the
  subsequence
  \[
    \{-1, -1, -1, \dots\}
  \]
  does converge.
\end{example}

\begin{remark}
  This shows that the converse of Theorem \ref{thm:subseq} is
  not true, i.e.~a convergent subsequence does not
  imply that the original sequence converges.
\end{remark}

\begin{example}
  The sequence
  \[
    a_n =
    \begin{cases}
      1 & \text{if $n$ is prime}, \\
      0 & \text{otherwise}
    \end{cases}
  \]
  does not converge.
\end{example}

\begin{exercise}
  Show the limit of the sequence
  \[
    \left\{1, -\frac{1}{2}, \frac{1}{3}, -\frac{1}{4}, \frac{1}{5}, \frac{1}{5}, -\frac{1}{5}, \frac{1}{5}, -\frac{1}{5}, \dots\right\}
  \]
\end{exercise}

\begin{proof}
  The subsequence
  \[\left\{\frac{1}{5}, \frac{1}{5}, \dots\right\}\]
  converges to $\frac{1}{5}$ while the subsequence
  \[\left\{-\frac{1}{5}, -\frac{1}{5}, \dots\right\}\]
  converges to $-\frac{1}{5}$. Thus the original
  sequence diverges.
\end{proof}

\begin{remark}
  If we can find two subsequences that converge to different
  limits, then the original sequence diverges.
  This is the contrapositive of Theorem \ref{thm:subseq}.
\end{remark}

\begin{theorem}[Bolzano-Weierstrass theorem]
  Every bounded sequence has a convergent subsequence.
  \footnote{This demonstrates some kind of
    \textit{compactness} of the real numbers.}
\end{theorem}

\begin{proof}
  Let $\{a_n\}$ be a bounded  be a bounded sequence.
  So there exists $M > 0$ such that $\sup_{n} |a_n| < M$.
  So $a_n$ is contained in $[-M, M]$. Split
  $[-M, M]$ into $[-M, 0]$ and $[0, M]$.
  Pick one that contains infinitely many elements of
  $\{a_n\}$ and call it $I_1$. Then pick
  $a_{n_1} \in \{a_n\}$ such that $a_{n_1} \in I_1$.
  Split $I_1$ again into two closed intervals of
  the same size. Take one of these two that contains
  infinitely many elements of $\{a_n\}$ and call it $I_2$.
  Then take $a_{n_2} \in \{a_n\}$ such that
  $a_{n_2} \in I_2$. Repeat this process to to get
  $I_{k+1} \subseteq I_k$ with
  $|I_{k+1}| = \frac{1}{2}|I_k|$ such that $I_{k+1}$
  contains infinitely many elements of $\{a_n\}$.
  \footnote{Here, by $|I_k|$ we mean the length of the
    interval $I_k$.}
  Also pick $a_{n_{k+1}} \in \{a_n\}$ such that
  $a_{n_{k+1}} \in I_{k+1}$ with $n_{k + 1} > n_k$.
  
  By construction, $\{a_{n_k}\}$ is a subsequence of
  $\{a_n\}$ and $a_{n_k} \in I_k$. We have the
  $I_k$ being closed intervals with
  \[
  I_1 \supseteq I_2 \supseteq I_3 \supseteq \dots
  .\]
  So there exists $x \in \R$ such that
  $x \in \bigcap_{k = 1}^{\infty} I_k$. Note that
  $|I_k| = M\left(\frac{1}{2}\right)^{k-1}$. Then we claim
  that $\lim_{k \to \infty} a_{n_k} = x$.

  Let $\epsilon > 0$. Take $N$ such that
  \[2^N > \frac{2M}{\epsilon}.\]
  Then for every $k \ge N$, we have
  \[
    |a_{n_k} - x| \le M \left(\frac{1}{2}\right)^{k - 1}
    < \epsilon
  \]
  since $a_{n_k}, x \in I_k$.
  Thus $\lim_{k \to \infty} a_{n_k} = x$, and
  $\{a_{n_k}\}$ is a convergent subsequence.
\end{proof}

  \chapter{Sept.~12 -- The Cauchy Criterion}

\section{Cauchy Sequences}
\begin{definition}
  A sequence $\{a_n\}$ is called a \textbf{Cauchy}
  sequence if for every $\epsilon > 0$, there exists
  $N \in \N$ such that for every $m, n \ge N$,
  one has $|a_m - a_n| < \epsilon$.
  \footnote{The Cauchy condition controls the
    \textit{oscillation} of the \textit{tail} of a sequence.}
\end{definition}

\begin{theorem}
  \label{thm:cauchy-convergent}
  Every convergent sequence is a Cauchy sequence.
\end{theorem}

\begin{proof}
  Assume $\lim_{n \to \infty} a_n = a$. Then for every
  $\epsilon > 0$, we can find $N$ such that for
  every $n \ge N$, we have $|a_n - a| < \epsilon/2$.
  Then for every $m, n \ge N$, we have
  \[
  |a_m - a_n| = |a_m - a + a - a_n| \le
  |a_m - a| + |a - a_n| <
  \frac{\epsilon}{2} + \frac{\epsilon}{2} = \epsilon
  \]
  by the triangle inequality.
\end{proof}

\begin{lemma}
  Every Cauchy sequence is bounded.
\end{lemma}

\begin{proof}
  Suppose $\{x_n\}$ is a Cauchy sequence.
  Pick $\epsilon = 1$. Then there exists $N$ such that
  for all $m, n \ge N$, we have $|x_m - x_n| < 1$.
  Fixing $m = N$, we know that for all $n \ge N$,
  $|x_N - x_n| < 1$. So $|x_n| \le |x_N| + 1$ for all
  $n \ge N$. Set
  \[
    M = \max\{|x_1|, |x_2|, \dots, |x_{N-1}|, |x_N| + 1\}
  .\]
  Then $\sup |x_n| \le M$ by construction.
\end{proof}

\begin{theorem}[Cauchy criterion]
  A sequence converges if and only if it is a Cauchy
  sequence.
\end{theorem}

\begin{proof}
  $(\Rightarrow)$\, This is Theorem \ref{thm:cauchy-convergent}.

  $(\Leftarrow)$\, Suppose $\{a_n\}$ is a Cauchy sequence.
  Since $\{a_n\}$ is Cauchy, we know $\sup |a_n| \le M$
  for some $M \in \R$. Then by the Bolzano-Weierstrass
  theorem, we can find a convergent subsequence
  $\{a_{n_k}\}$ such that
  $\lim_{k \to \infty} a_{n_k} = a$. We show that
  we also have $\lim_{n \to \infty} a_n = a$.

  For every $\epsilon > 0$, we can find $N_1$ such that
  for all $m, n \ge N_1$, we have
  $|a_m - a_n| < \epsilon/2$. Since
  $\lim_{k \to \infty} a_{n_k} = a$, there is some
  $K$ such that for all $k \ge K$, we have
  $|a_{n_k} - a| < \epsilon/2$. Take
  \[N \ge \max\{N_1, n_K\}.\]
  We can find $K_0$ such that $n_{K_0} \ge N$.
  Then for every $n \ge N$,
  \[
    |a_n - a| = |a_n - a_{n_{K_0}} + a_{n_{K_0}} - a|
    \le |a_n - a_{n_{K_0}}| + |a_{n_{K_0}} - a|
    < \frac{\epsilon}{2} + \frac{\epsilon}{2} = \epsilon
  \]
  by the triangle inequality and the Cauchy condition.
\end{proof}

\begin{remark}
  The Cauchy condition allows us to show that a sequence
  converges without explicitly providing its limit.
\end{remark}

\section{Revisiting Completeness}
This is the way we have discussed completeness
(ordered by implication):
\begin{itemize}
  \item Axiom of Completeness
    \begin{itemize}
      \item Nested intervals property
        \begin{itemize}
          \item Bolzano-Weierstrass theorem
            \begin{itemize}
              \item Cauchy criterion
            \end{itemize}
        \end{itemize}
      \item Monotone convergence theorem.
    \end{itemize}
\end{itemize}

But this is not the only way to do so: We have several
ways of choosing axioms to define completeness. For example,
we can also prove the nested intervals property using
the monotone convergence theorem.

\begin{exercise}
  The monotone convergence theorem implies the nested
  intervals property.
\end{exercise}

\begin{proof}
  Let $I_n = [a_n, b_n]$ with $I_{n+1} \subseteq I_n$.
  In particular, $\{a_n\}$ is increasing and
  bounded ($b_1$ is an upper bound). So by
  the monotone convergence theorem,
  $\lim_{n \to \infty} a_n = a$ exists.

  Left as an exercise to show that $a \in I_n$ for
  all $n$.
\end{proof}

\begin{exercise}
  Given the Archimedean property,
  the nested intervals property implies the Axiom of
  Completeness.
\end{exercise}

\begin{proof}
  Note that $\frac{1}{2^n} \to 0$ as $n \to \infty$.
  This is because for
  every $\epsilon > 0$, we can find $N$ such that
  $\frac{1}{N} < \epsilon$ by the Archimidean property.
  Then
  \[
    \frac{1}{2^N} < \frac{1}{N}
  \]
  for all $N \in \N$. So
  $\lim_{n \to \infty} \frac{1}{2^n} = 0$.

  Now let $S$ be a nonempty set which is bounded above.
  Let $U$ be an upper bound for $S$. Take $s \in S$.
  Set $a_1 = s$, $b_1 = U$. Consider
  \[
    \frac{s + U}{2}
  .\]
  If $\frac{s + U}{2}$ is an upper bound for $S$, then
  we set $a_2 = a_1 = s$, $b_2 = \frac{s + U}{2}$. If
  $\frac{s + U}{2}$ is not an upper bound for $S$, then
  we set $a_2 = \frac{s + U}{2}$, $b_2 = b_1 = U$.
  Note that $[a_2, b_2] \subseteq [a_1, b_1]$.
  Repeat the same process for $a_n$ and $b_n$ to
  obtain the closed intervals
  \[[a_1, b_1] \supseteq [a_2, a_2] \supseteq [a_3, b_3] \supseteq \dots.\]
  By the nested interval properties, the intersection
  $\bigcap_{n = 1}^\infty [a_n, b_n]$ is nonempty.
  Note that
  \begin{align*}
    \left|[a_1, b_1]\right| &= |b_1 - a_1| = |U - s| \\
    \left|[a_2, b_2]\right| &= |b_2 - a_2| = \left|\frac{U - s}{2}\right| \\
                            &\ \ \vdots \\
    \left|[a_n, b_n]\right| &= |b_n - a_n| = \frac{2}{2^n}|U - s|
  \end{align*}
  So there is only one $x \in \R$ such that
  $x \in \bigcap_{n = 1}^\infty [a_n, b_n]$. We claim
  that $\sup S = x$.

  Note that $x \in [a_n, b_n]$ for all $n$. So
  $a_n$ is not an upper bound and $b_n$ is an upper bound.
  Suppose for contradiction that $x$ is not an upper bound.
  Then there exists $s_0 \in S$ such that $s_0 > x$.
  Since $|[a_n, b_n]| \to 0$, there exists an $N$ such
  that whenever $n \ge N$,
  \[
    |[a_n, b_n]| < \frac{1}{2}|s_0 - x|
  .\]
  Since $x \in [a_n, b_n]$, this implies that $s_0 > b_n$,
  which is a contradiction with $b_n$ being an upper bound.

  Use a similar idea to show that $x$ is the \textit{least}
  upper bound.
\end{proof}

\begin{remark}
  These are all different ways to understand the
  same idea of completeness.
\end{remark}

  \chapter{Sept.~14 -- Series}

\begin{definition}
  Let $\{b_n\}$ be a sequence. An infinite \textbf{series}
  is formally given by
  \[
    \sum_{n = 1}^\infty b_n = b_1 + b_2 + \dots
  .\]
\end{definition}

\begin{definition}
  We define the \textbf{partial sum} of a series by
  \[s_m = \sum_{n = 1}^m b_n.\]
\end{definition}

\section{Convergence of Series}

\begin{definition}
  The series $\sum_{n=1}^\infty{b_n}$ \textbf{converges}
  to $B$ if $\lim_{m \to \infty} s_m = B$. Otherwise
  we say that the series \textbf{diverges}.
\end{definition}

\begin{example}
  Consider the series
  \[\sum_{n = 1}^{\infty} \frac{1}{n^2} = 1 + \frac{1}{2^2} + \frac{1}{3}^2 + \dots.\]
  We look at the partial sums for $m > 1$:
  \begin{align*}
    s_m &= \sum_{n = 1}^{m} \frac{1}{n^2} = 1 + \frac{1}{2^2} + \frac{1}{3^2} + \frac{1}{4^2} + \dots + \frac{1}{m^2}
    \le 1 + \frac{1}{2(1)} + \frac{1}{3(2)} + \frac{1}{4(3)} + \dots + \frac{1}{m(m-1)} \\
        &= 1 + 1 - \frac{1}{2} + \frac{1}{2} - \frac{1}{3} + \frac{1}{3} - \frac{1}{4} + \dots + \frac{1}{m-1} - \frac{1}{m}
        \le 2 - \frac{1}{m}.
  \end{align*}
  Note that $\{s_m\}$ is a monotone sequence and it is
  bounded above by $2$. Thus by the monotone convergence
  theorem, $\{s_m\}$ converges and there is some
  $B \in \R$ such that
  $\lim_{m \to \infty} s_m = B$.
\end{example}

\begin{remark}
  Using some complex analysis, we can find $B$ by way of
  residue calculations.
\end{remark}

\begin{example}
  Consider the harmonic series
  \[\sum_{n = 1}^\infty \frac{1}{n} = 1 + \frac{1}{2} + \frac{1}{3} + \dots.\]
  We look at the partial sums
  \[
  s_m = 1 + \frac{1}{2} + \dots + \frac{1}{m}
  .\]
  Note specifically that
  \begin{align*}
    s_4 &= 1 + \frac{1}{2} + \frac{1}{3} + \frac{1}{4}
    > 1 + \frac{1}{2} + \frac{1}{4} + \frac{1}{4}
    = 1 + \frac{1}{2} + 2\left(\frac{1}{4}\right)
    = 1 + \frac{1}{2} + \frac{1}{2}
    = 1 + 2\left(\frac{1}{2}\right)\\
    s_8 &= 1 + \frac{1}{2} + \frac{1}{3} + \frac{1}{4}
    + \frac{1}{5} + \frac{1}{6} + \frac{1}{7} + \frac{1}{8}
    > 1 + \frac{1}{2} + 2\left(\frac{1}{4}\right) + 4\left(\frac{1}{8}\right)
    = 1 + 3\left(\frac{1}{2}\right)\\
        &\ \ \vdots \\
    s_{2^k} &= 1 + \frac{1}{2} + \frac{1}{3} + \frac{1}{4}
    + \dots + \frac{1}{2^{k - 1} + 1} +
    \frac{1}{2^{k - 1} + 2} + \dots + \frac{1}{2^k}
    > 1 + \frac{k}{2}
  \end{align*}
  Thus $\{s_{2^k}\}$ diverges, so $\{s_m\}$ also
  diverges.
\end{example}

\begin{remark}
  This type of trick (analyzing $2^k$ terms) is called
  \textit{dyadic analysis},
  and it shows up frequently in analysis, particularly
  harmonic analysis.
\end{remark}

\begin{theorem}[Cauchy condensation test]
  Suppose $\{b_n\}$ is decreasing and $b_n \ge 0$ for
  all $n$. Then
  \[\sum_{n = 1}^\infty b_n = b_1 + b_2 + b_3 + \dots\]
  converges if
  and only if
  \[\sum_{n=0}^{\infty} 2^n b_{2^n} = b_1 + 2b_2 + 4b_4 + \dots\]
  converges.
\end{theorem}

\begin{proof}
  First we show the backwards direction.
  Assume $\sum_{n = 0}^\infty 2^n b_{2^n}$ converges.
  Define
  \[t_k = b_1 + \dots + 2^k b_{2^k}.\]
  By assumption, $\{t_k\}$ converges.
  Note that $t_k \ge 0$ and $\sup_k t_k \le M$ since
  convergent series are bounded. Set
  \[s_m = \sum_{n=1}^m b_n.\]
  Fix $m$ and take $k$ large such that $m \le 2^{k + 1} - 1$.
  Then $s_m \le s_{2^{k + 1} - 1}$ since $b_n \ge 0$.
  Observe that
  \begin{align*}
    s_{2^{k+1} - 1} &= b_1 + (b_2 + b_3) +
    (b_4 + b_5 + b_6 + b_7) + \dots
    + (b_{2^k} + \dots + b_{2^{k+1} - 1}) \\
    &\le b_1 + 2b_2 + 4b_4 + \dots + 2^k b_{2^k}.
  \end{align*}
  So $s_m \le s_{2^{k + 1} - 1} \le t_k \le M$.
  Thus $\{s_m\}$ is increasing and bounded, so
  by the monotone convergence theorem,
  $\lim_{m \to \infty} s_m = B \in \R$ exists.

  Now we show the forwards direction. Argue
  by contraposition.
  Suppose $\sum_{n = 0}^\infty 2^n b_{2^n}$
  diverges, then we show that $\sum_{n = 1}^\infty b_n$ also
  diverges. Just need to check that
  $s_{2^k} \ge \frac{1}{2} + k$ (left as an exercise).
\end{proof}

\begin{corollary}
  The series
  \[\sum_{n = 1}^\infty \frac{1}{n^p}\]
  converges if and only if $p > 1$.
\end{corollary}

\begin{proof}
  Let $b_n = \frac{1}{n^p}$ and $b_{2^n} = \frac{1}{2^{np}}$.
  Then we have
  \[\sum_{n=0}^\infty 2^n b_{2^n} = \sum_{n=0}^\infty 2^{(1 - p)n}.\]
  The RHS is a geometric series, which converges if and
  only if $p > 1$. To see this, denote $2^{1 - p} = a$.
  Then we have
  \[\sum_{n = 0}^\infty 2^{(1 - p)n} = \sum_{n = 0}^\infty a^n.\]
  We can observe that the partial sums
  \[t_k = \sum_{n=0}^k a^n = \frac{a^{k+1} - 1}{a - 1}\]
  converges if and only if $a^{k + 1}$ converges.
  This happens if and only if $a < 1$, which happens
  if and only if $p > 1$.
\end{proof}

\section{Properties of Series}
\begin{theorem}[Algebraic limit theorem for series]
  Let
  \[\sum_{n=1}^\infty a_n = A, \quad \sum_{n=1}^\infty b_n = B.\]
  Then for all $c \in \R$, we have
  \[
    \sum_{n=1}^\infty ca_n = cA, \quad
    \sum_{n=1}^\infty (a_n + b_n) = A + B
  .\]
\end{theorem}

\begin{proof}
  Let $\sum_{n=1}^\infty a_n = A$. So
  $s_m = \sum_{n=1}^m a_n$ converges. Set
  $\lim_{n \to \infty} s_m = A$. Define
  \[
    t_m = \sum_{n=1}^m ca_n = c\sum_{n=1}^m a_n = cs_m
  .\]
  Then by the algebraic limit theorem, we have
  $\lim_{m \to \infty} t_m = c\lim_{m \to \infty} s_m = cA$.
\end{proof}

\begin{theorem}[Cauchy criterion for series]
  The series $\sum_{n=1}^\infty a_n$ converges if and only if
  for all $\epsilon > 0$, there exist $N$ such that
  whenever $m, n \ge N$, we have
$|a_{m+1} + \dots + a_n| < \epsilon$.
\end{theorem}

\begin{proof}
  The series $\sum_{k = 1}^\infty a_k$ converges
  if and only if $s_m = \sum_{k = 1}^m a_k$ converges.
  We show that $\{s_m\}$ is a Cauchy sequence. For
  all $\epsilon > 0$, there exists $N$ such
  that for all $m, n \ge N$
  \[|s_n - s_m| = |a_n + \dots + a_{m+1}| < \epsilon.\]
  The converse is the same inequality.
\end{proof}

\begin{corollary}
  If $\sum_{n=1}^\infty a_n$ converges, then
  $\lim_{n \to \infty} a_n = 0$.
\end{corollary}

\begin{proof}
  Take $m = n - 1$.
\end{proof}

\begin{theorem}
  Assume $\{a_n\}$ and $\{b_n\}$ are sequences such that
  $0 \le a_n \le b_n$ for all $n$. Then
  \begin{enumerate}
    \item $\sum_{n=1}^\infty b_n$ converges implies
      $\sum_{n=1}^\infty a_n$ converges,
    \item and $\sum_{n=1}^\infty a_n$ diverges implies
      $\sum_{n=1}^\infty b_n$ diverges.
  \end{enumerate}
\end{theorem}

\begin{proof}
  For all $m, n$, we have
  \[
    |a_{m + 1} + \dots + a_n| \le |b_{m + 1} + \dots + b_n|
  .\]
  Then apply the Cauchy criterion.
\end{proof}

\begin{definition}
  A series is called \textbf{geometric} if it is of the
  form
  \[
    \sum_{k=0}^\infty ar^k = a + ar + ar^2 + \dots
  .\]
\end{definition}

Note that the geometric series diverges when $r = 1$ and
$a \ne 0$. When $r \ne 1$, the partial sums
\[
  s_m = \sum_{k=0}^m ar^k = a\frac{1 - r^{m + 1}}{1 - r}
\]
converge if $|r| < 1$. In this case, as $m \to \infty$, we
have
\[
  s_m \to \frac{a}{1 - r}
.\]

  \chapter{Sept.~19 -- Absolute Convergence}

\section{Absolute Convergence}

\begin{definition}
  Consider a series $\sum_{n=1}^\infty a_n$. If
  \[\sum_{n=1}^\infty |a_n|\]
  converges, then we say
  $\sum_{n=1}^\infty a_n$ \textbf{converges absolutely}.
\end{definition}

\begin{theorem}
  If $\sum_{n=1}^\infty |a_n|$ converges,
  then $\sum_{n = 1}^\infty a_n$ converges.
\end{theorem}

\begin{proof}
  For every $\epsilon > 0$, since $\sum_{n=1}^\infty a_n$
  converges, there is $N$ such that for all
  $m, k \ge N$,
  \[\sum_{n=k + 1}^m |a_n| < \epsilon.\]
  This is by the Cauchy criterion for series.
  Then for all $m, k \ge N$, we have
  \[
    \left\lvert \sum_{n=k+1}^m a_n\right\rvert
    \le \sum_{n=k+1}^m |a_n|
    < \epsilon
  \]
  by the triangle inequality. Apply
  the Cauchy criterion again to conclude that
  $\sum_{n=1}^\infty a_n$ converges.
\end{proof}

\begin{theorem}[Alternating series test]
  If $a_1 \ge a_2 \ge a_3 \ge \dots$ and
  $\lim_{n \to \infty} a_n = 0$, then
  the series
  \[
    \sum_{n = 1}^\infty (-1)^{n + 1} a_n
  \]
  converges.
\end{theorem}

\begin{proof}
  Set $s_m = \sum_{n=1}^m (-1)^{n+1} a_n$. Check
  that
  \[s_m - s_k = \sum_{n=k+1}^m (-1)^{n+1} a_n.\]
  Suppose that $m$ and $k$ are odd, then
  \[
    s_m - s_k = \underbrace{a_m - a_{m - 1}}_{\le 0} + a_{m-2} -
    \dots + a_{k + 2} - a_{k + 1}
  .\]
  So $s_m - s_k \le 0$.
  We can also group the terms as as
  \[
    s_m - s_k = a_m \underbrace{- a_{m - 1} + a_{m - 2}}_{\ge 0} -
    \dots - a_{k + 3} - a_{k + 2} - a_{k + 1}
    \ge a_{m} - a_{k + 1}
  .\]
  So $|s_m - s_k| \le |a_m| + |a_{k + 1}|$ by the
  triangle inequality.
  Since $\lim_{n \to \infty} a_n = 0$, for all
  $\epsilon > 0$, there is $N$ such that $n \ge N$,
  we have $|a_n| < \epsilon$. Then for all $m, k \ge N$,
  \[
    |s_m - s_k| \le |a_m| + |a_{k + 1}| < 2\epsilon
  .\]
  Thus $\{s_k\}$ converges.
  Left as exercise to check the other parities of
  $m$ and $k$ (group differently).
\end{proof}

\begin{example}
  We saw previously that for $a_n = \frac{1}{n}$,
  $\sum_{n=1}^\infty a_n$ diverges. But
  $\sum_{n = 1}^\infty (-1)^{n+1} a_n$ converges.
\end{example}

\section{Rearrangements}

\begin{definition}
  Given a series $\sum_{k = 1}^\infty a_k$,
  we say that a series $\sum_{k = 1}^\infty b_k$ is
  a \textbf{rearrangement} of $\sum_{k = 1}^\infty a_k$
  if there is a bijection $f : \N \to \N$
  such that $b_{f(k)} = a_k$ for all $k \in \N$.
\end{definition}

\begin{example}
  Let
  \begin{align*}
    S &= 1 - \frac{1}{2} + \frac{1}{3} - \frac{1}{4} + \frac{1}{5} - \frac{1}{6} + \frac{1}{7} - \frac{1}{8} + \cdots, \\
    \frac{1}{2}S &= \frac{1}{2} - \frac{1}{4} + \frac{1}{6} - \frac{1}{8} + \frac{1}{10} - \cdots, \\
    S + \frac{1}{2}S &= 1 + \frac{1}{3} - \frac{1}{2} + \frac{1}{5} - \frac{1}{4} + \frac{1}{7} - \cdots.
  \end{align*}
  Notice that $S + \frac{1}{2}S$ is a rearrangment
  of $S$. Supposing that $S + \frac{1}{2}S$ converges
  to the same limit as $S$, we would have
  \[S + \frac{1}{2}S = S,\]
  or $S = 0$. This cannot be the case.
\end{example}

\begin{remark}
  A rearrangement of a series might have
  different convergence properties from the original
  series.
\end{remark}

\begin{theorem}
  If a series converges absolutely to $A$, then
  any rearrangement of the series converges to the
  same limit $A$.
\end{theorem}

\begin{proof}
  Let $\sum_{k = 1}^\infty a_k$ converge absolutely
  to $A$. Let $\sum_{k = 1}^\infty b_k$ be a
  rearrangement of $\sum_{k = 1}^\infty a_k$.
  We set
  \[s_n = \sum_{k=1}^n a_k, \quad t_m = \sum_{k=1}^m b_k.\]
  We want to show that $t_m$ converges to $A$.
  Since $\lim_{n \to \infty} s_n = A$, for every
  $\epsilon > 0$, there is $N_1$ such that
  \[
  |s_n - A| < \frac{\epsilon}{2}
  \]
  for all $n \in \N$. Since $\sum_{k = 1}^\infty a_k$
  converges absolutely, there is $N_2$ such that
  for all $n, m \ge N_2$, we have
  \[
    \sum_{k = m + 1}^n |a_k| < \frac{\epsilon}{2}
  .\]
  Since $\sum_{k = 1}^\infty b_k$ is a rearrangement
  of $\sum_{k = 1}^\infty a_k$, we can write
  $b_{f(k)} = a_k$ for some bijection $f$.
  Set
  \[N = \max\{N_1, N_2\}, \quad M = \max\{f(k) : 1 \le k \le N\}.\]
  Then for all $m \ge M$, $t_m - s_n$ will only
  consist of terms $a_k$ for $k > N$. In particular,
  \[|t_m - s_n| \le \sum_{k=n}^\infty |a_k| < \frac{\epsilon}{2}.\]
  Then we have
  \[|t_m - A| = |t_m - s_n + s_n - A|
  \le |t_m - s_n| + |s_n - A| < \frac{\epsilon}{2} + \frac{\epsilon}{2}
  = \epsilon.\]
  So $\lim_{m \to \infty} t_m = A$.
\end{proof}

  \chapter{Sept.~21 -- Iterated Sums and Topology}

\section{Double Sums}
Given a set of doubly indexed real numbers
$\{a_{ij} : i, j \in \N\}$, consider the
sums
\[
  \sum_{j = 1}^\infty \sum_{i = 1}^\infty a_{ij}, \quad
  \sum_{i = 1}^\infty \sum_{j = 1}^\infty a_{ij}.
\]
Are these sums equal? The answer is no, in general.

\begin{example}
  Define $a_{ij}$ by
  \[
    a_{ij} = \begin{cases}
      \left(\frac{1}{2}\right)^{j - i} & \text{if $j > i$}, \\
      -1 & \text{if $i = j$}, \\
      0 & \text{if $j < i$}.
    \end{cases}
  \]
  This looks like
  \[
    \begin{array}{c|ccccc}
      & a_{11} & a_{12} & a_{13} & a_{14} & \cdots \\
      \hline
      a_{11} & -1 & \frac{1}{2} & \frac{1}{4} & \left(\frac{1}{2}\right)^3 & \vdots \\
      a_{21} & 0 & -1 & \frac{1}{2} & \left(\frac{1}{2}\right)^2 & \vdots \\
      a_{31} & 0 & 0 & -1 & \frac{1}{2} & \vdots \\
      \vdots & \vdots & \vdots & \vdots & \vdots & \ddots
    \end{array}
  \]
  Then the first sum (over the columns) is
  \[
    \sum_{j = 1}^\infty \sum_{i = 1}^\infty a_{ij}
    = \sum_{j = 1}^\infty -\left(\frac{1}{2}\right)^{j - 1}
    = -\frac{1}{1 - \frac{1}{2}} = -2
  .\]
  Meanwhile, the second sum (over the rows) is
  \[
    \sum_{i = 1}^\infty \sum_{j = 1}^\infty a_{ij}
    = \sum_{i = 1}^\infty \left(-1 + \sum_{k = 1}^\infty \left(\frac{1}{2}\right)^{k}\right)
    = \sum_{i = 1}^\infty \left(-1 + 1\right)
    = \sum_{i = 1}^\infty 0 = 0
  .\]
  Notice that these two sums are not the same.
\end{example}

\begin{remark}
  We cannot always exchange the order of a double
  sum.
  \footnote{\textit{Fubini's theorem} gives conditions under which we can do this for iterated \textit{integrals} (when the integrand is \textit{absolutely integrable}).}
\end{remark}

\subsection{Convergence of Double Sums}
\begin{definition}
  We say that $\sum_{i = 1} \sum_{j = 1} |a_{ij}|$
  \textbf{converges}
  if for all $i \in \N$,
  $\sum_{j = 1}^\infty |a_{ij}|$ converges to some real
  number $b_i$ and $\sum_{i = 1}^\infty b_i$
  converges.
\end{definition}

\begin{theorem}
  Consider $\{a_{ij} : i, j \in \N\}$. If
  \[
    \sum_{i = 1}^\infty \sum_{j = 1}^\infty |a_{ij}|
  \]
  converges, then both
  \[
    \sum_{i = 1}^\infty \sum_{j = 1}^\infty a_{ij} \quad
    \text{and} \quad
    \sum_{j = 1}^\infty \sum_{i = 1}^\infty a_{ij}
  \]
  converge to the same limit, i.e.
  \[
    \sum_{i = 1}^\infty \sum_{j = 1}^\infty a_{ij}
    = \sum_{j = 1}^\infty \sum_{i = 1}^\infty a_{ij}
    = \lim_{n \to \infty} S_{nn}
  \]
  where $S_{nn} = \sum_{i = 1}^n \sum_{j = 1}^n a_{ij}$.
\end{theorem}

\begin{proof}
  Go look this up.
\end{proof}

\section{Basic Topology in \texorpdfstring{$\R$}{R}}
\subsection{Open Sets}
\begin{definition}
  For all $a \in \R$, $\epsilon > 0$, we define
  \[V_\epsilon(a) = \{x \in \R : |x - a| < \epsilon\}\]
  to be the \textbf{$\epsilon$-neighborhood} of $a$.
\end{definition}

\begin{definition}
  A set $U \subseteq \R$ is \textbf{open} if for all
  $a \in U$, there exists $\epsilon > 0$ such that
  $V_\epsilon(a) \subseteq U$.
\end{definition}

\begin{example}
  The set $\R$ is open: Simply take $\epsilon = 1$ for
  any choice of $a \in \R$.
\end{example}

\begin{example}
  The open interval $(c, d)$ is open: For any
  $x \in (c, d)$, take $\epsilon = \min\{x - c, d - x\}$.
\end{example}

\begin{theorem}\leavevmode
  \begin{enumerate}
    \item The union of an arbitrary collection of open sets
      is open.
    \item The intersection of a finite collection of
      open sets is open.
  \end{enumerate}
\end{theorem}

\begin{proof}
  (1)\, Let $\{U_\lambda : \lambda \in \Lambda\}$ be a
  collection of open sets and consider
  $\bigcup_{\lambda \in \Lambda} U_\lambda$. For
  every $a \in \bigcup_{\lambda \in \Lambda} U_\lambda$,
  there is $\lambda'$ such that $a \in U_{\lambda'}$.
  Since $U_{\lambda'}$ is open, there exists
  $\epsilon > 0$ such that
  \[V_\epsilon(a) \subseteq U_{\lambda'} \subseteq \bigcup_{\lambda \in \Lambda} U_\lambda.\]
  So $\bigcup_{\lambda \in \Lambda} U_\lambda$ is open.

  (2)\, Let $U_1, \ldots, U_n$ be a collection of
  open sets and consider $\bigcap_{j = 1}^n U_j$.
  For every $a \in \bigcap_{j = 1}^n U_j$, note that
  $a \in U_j$ for all $j = 1, \ldots, n$. Since
  $U_j$ is open, there exists $\epsilon_j > 0$ such that
  $V_{\epsilon_j}(a) \subseteq U_j$. Then take
  \[\epsilon = \min_{i \le j \le n}\{\epsilon_j\},\]
  which exists since the collection is finite.
  Since $\epsilon_j > 0$, we have $\epsilon > 0$ as well.
  By construction,
  \[V_\epsilon(a) \subseteq V_{\epsilon_j}(a) \subseteq U_j\]
  for all $j$. So
  $V_\epsilon(a) \subseteq \bigcap_{j = 1}^n U_j$, and
  thus $\bigcap_{j = 1}^n U_j$ is open.
\end{proof}

\begin{example}
  Consider the family of open sets
  $U_n = (-\frac{1}{n}, \frac{1}{n})$ for $n \in \N$. Notice
  that their intersection
  \[\bigcap_{n = 1}^\infty U_n = \{0\}\]
  is not open.
\end{example}

\subsection{Limit Points}
\begin{definition}
  A point $x$ is a \textbf{limit point} of a set
  $A$ if for all $\epsilon > 0$, 
  we have
  \[(V_\epsilon(x) \cap A) \setminus \{x\} \ne \emptyset,\]
  i.e.~there is some other point in the
  $\epsilon$-neighborhood of $x$ that is also in $A$.
\end{definition}

\begin{theorem}
  \label{thm:limit-point}
  A point $x$ is a limit point of $A$ if and only
  if $x = \lim_{n \to \infty} a_n$ for some
  sequence $\{a_n\}$ with $a_n \ne x$ and $a_n \in A$.
\end{theorem}

\begin{proof}
  ($\Rightarrow$)\, Suppose $x$ is a limit point of $A$.
  Take $\epsilon = 1 / n$ and pick
  $a_n \in (V_{1 / n}(x) \cap A)$ such that $a_n \ne x$.
  For such a sequence $\{a_n\}$, for all $\epsilon > 0$,
  if $N \ge 1 / \epsilon$, then for all $n \ge N$,
  we have
  \[|a_n - x| \le \frac{1}{N} < \epsilon.\]
  So $\lim_{n \to \infty} a_n = x$.

  ($\Leftarrow$)\, Assume such a sequence $\{a_n\}$ exists.
  Then for any $\epsilon > 0$, there exists $N$
  such that $|a_n - x| < \epsilon$ for all $n \ge N$.
  Note that $a_N \in V_\epsilon(x)$, and also
  $a_N \in A$ and $a_N \ne x$. So
  $a_N \in (V_\epsilon(x) \cap A) \setminus \{x\}$,
  i.e.~this set is not empty. Thus $x$ is a limit point
  of $A$.
\end{proof}

  \chapter{Sept.~26 -- Closed Sets}

\section{Closed Sets}

\begin{definition}
  Let $A \subseteq \R$. An element $x \in A$ is
  an \textbf{isolated point} of $A$ if it is not
  a limit point of $A$, i.e.~there exists $\epsilon > 0$
  such that $V_\epsilon(x) \cap A = \{x\}$.
\end{definition}

\begin{definition}
  A set $A \subseteq \R$ is \textbf{closed}
  if it contains all of its limit points.
\end{definition}

\begin{example}
  The empty set and $\R$ are closed. Moreover, any
  set without limit points is closed.
\end{example}

\begin{theorem}
  A set $A \subseteq \R$ is closed if and only if
  every Cauchy sequence in $A$ converges to a
  limit in $A$.
\end{theorem}

\begin{proof}
  $(\Rightarrow)$\, Suppose $A$ is closed and
  $\{a_n\}$ is Cauchy with $a_n \in A$ for all $n$.
  Since Cauchy sequences are convergent, let
  $x = \lim_{n \to \infty} a_n$. Now consider two cases.
  If there exists an $n$ such that $x = a_n$, then we're
  done since $x = a_n \in A$. Otherwise, we have
  $a_n \ne x$ for all $n$. By Theorem \ref{thm:limit-point},
  $x$ is a limit point of $A$. So $x \in A$ as
  $A$ is closed.

  $(\Leftarrow)$\, Let $x$ be a limit point of $A$.
  Then there exists a sequence $\{a_n\}$ with $a_n \in A$
  and $a_n \ne x$ for all $n$ such that
  $\lim_{n \to \infty} a_n = x$. This means that
  $\{a_n\}$ is Cauchy, so by assumption,
  $x = \lim_{n \to \infty} a_n \in A$. Thus every limit
  point of $A$ is in $A$, so $A$ is closed.
\end{proof}

\begin{example}
  Consider the set
  \[
    A = \left\{\frac{1}{n} : n \in \N\right\}
  .\]
  First $x \ne 0$, we look at the following cases:
  \begin{enumerate}
    \item If $x < 0$, let $\epsilon = |x|$. Then
    $V_\epsilon(x) = (2x, 0)$, and
    $V_\epsilon(x) \cap A = \emptyset$ since
    $A \subseteq \R^+$. So $x < 0$ is not
    is not a limit point of $A$.
  \item If $x > 1$, let $\epsilon = x - 1$. Then
    $V_\epsilon(x) = (1, 2x - 1)$, so
    $V_\epsilon(x) \cap A = \emptyset$ as all $y \in A$
    satisfies $y \in (0, 1]$.
  \item If $x \in (0, 1]$, then there exists $n \in N$
    such that $n > 1 / x$. Let
    $n_0 = \min\{n \in \N : n > 1 / x\}$, which exists
    by the well-ordering principle. Noting that
    $n_0 \ge 2$, we have
    \[
    \frac{1}{n_0} < x \le \frac{1}{n_0 - 1}
    .\]
    Now we look at two more cases:
    \begin{enumerate}
      \item If $x = \frac{1}{n_0 - 1}$, let
        \[\epsilon = x - \frac{1}{n_0} = \frac{1}{n_0 - 1} - \frac{1}{n_0}.\]
        Then we have
        \[
        V_\epsilon(x) = \left(\frac{1}{n_0}, \frac{2}{n_0 - 1} - \frac{1}{n_0}\right)
        .\]
        Note that
        $(V_\epsilon \cap A) \setminus \{x\} = \emptyset$ if $n_0 = 2$.
        Otherwise, $n_0 > 2$ and we have
        \[\frac{2}{n_0 - 1} - \frac{1}{n_0} - \frac{1}{n_0 - 2} = \dots = \frac{-2}{n_0(n_0 - 2)(n_0 - 1)} < 0.\]
        So $V_\epsilon(x) \subseteq \left(\frac{1}{n_0}, \frac{1}{n_0} - 2\right)$, which means that
        $V_\epsilon(x) \cap A = \{x\}$.
      \item Otherwise, $x \in \left(\frac{1}{n_0}, \frac{1}{n_0 - 1}\right)$
        and let
        \[\epsilon = \min\left\{x - \frac{1}{n_0}, \frac{1}{n_0 - 1} - x\right\}.\]
        Then (left as exercise)
        $V_\epsilon(x) \subseteq \left(\frac{1}{n_0}, \frac{1}{n_0 - 1}\right)$,
        which implies $V_\epsilon(x) \cap A = \emptyset$.
    \end{enumerate}
  \end{enumerate}
  So $x \ne 0$ is not a limit point of $A$. However,
  $x = 0 = \lim_{n \to \infty} \frac{1}{n}$, so $0$ is
  a limit point of $A$. But $0 \notin A$, so $A$ is
  not closed.
\end{example}

\begin{example}
  Let $A = [a, b]$. For any Cauchy sequence
  $\{x_n\} \subseteq A$, let
  $x = \lim_{n \to \infty} x_n$. Since
  $x_n \ge a$, we have $x = \lim_{n \to \infty} x_n \ge a$.
  Similarly, $x_n \le b$ implies that $x \le b$. So
  $x \in [a, b] = A$, and thus $A$ is closed.
\end{example}

\begin{example}
  Consider $\Q$. For any $x \in \R$, for all $n \in N$
  there exists $a_n$ such that $a_n \in \Q$ with
  \[
  \frac{1}{2n} < |a_n - x| < \frac{1}{n}
  .\]
  Thus $a_n \ne x$ and $a_n \in \Q$ for all $n$, so
  $\lim_{n \to \infty} a_n = x$ is a limit point of
  $\Q$. So $\Q$ is not closed.
\end{example}

\begin{remark}
  We can also define the real numbers as equivalence
  classes of Cauchy sequences.
  \footnote{Using \textit{Dedekind cuts} is another such way.}
  Note that Cauchy sequences
  do not require an ordering (only a metric), so we can
  easily extend this definition to higher dimensions.
\end{remark}

\section{The Closure of a Set}

\begin{definition}
  Let $A \subseteq \R$. Define the \textbf{closure}
  of $A$ as
  \[
    \overline{A} = \{x : x \in A \text{ or $x$ is a limit point of $A$}\}
  .\]
\end{definition}

\begin{theorem}
  The closure of a set $A$ is closed. Furthermore,
  if $B$ is closed and $A \subseteq B$, then
  $\overline{A} \subseteq B$.
  \footnote{So $\overline{A}$ is the smallest closed set
    containing $A$.}
\end{theorem}

\begin{proof}
  Let $x$ be a limit point of $\overline{A}$.
  We want to show that $x$ is also a limit point of $A$
  (so we will have $x \in \overline{A}$). If $x \in A$,
  then we're done. Otherwise, $x \notin A$, so for
  all $\epsilon > 0$, we have
  $(V_{\epsilon / 2}(x) \cap \overline{A}) \setminus \{x\} \ne \emptyset$,
  so let $y \in (V_{\epsilon / 2}(x) \cap \overline{A}) \setminus \{x\}$.
  If $y \in A$, then $(V_{\epsilon}(x) \cap A) \setminus \{x\}
  = \emptyset$ and we're done. Otherwise, $y \notin A$,
  so $y \in \overline{A}$ implies that $y$ is a limit
  point of $A$. So there exists
  $z \in (V_{\epsilon / 2} \cap A) \setminus \{y\}$.
  Then
  \[|x - z| \le |x - y| + |y - z| < \frac{\epsilon}{2} + \frac{\epsilon}{2} = \epsilon.\]
  So $z \in (V_\epsilon(x) \cap A) \setminus \{x\}$.
  Since this is true for all $\epsilon > 0$,
  $x$ is a limit point of $A$.
  Thus $x \in \overline{A}$, so $\overline{A}$ is closed.

  For the second part, let $x \in \overline{A}$. If
  $x \in A$, then $A \subseteq B$ implies that $x \in B$.
  If $x \notin A$, then $x$ is a limit point of $A$.
  So there exists a sequence $\{a_n\}$ with $a_n \in A$
  for all $n$ such that $a_n \to x$.
  Since $a_n \in A \subseteq B$, we have $a_n \in B$ for
  all $n$. Since $a_n \to x$, we must have
  $x \in B$ since $B$ is closed. Thus
  $A \subseteq B$.
\end{proof}

\begin{corollary}
  If $A \subseteq B$, then $\overline{A} \subseteq \overline{B}$.
\end{corollary}

\begin{proof}
  Note that
  Cauchy sequences in $A$ are also Cauchy sequences in $B$.
\end{proof}

  \chapter{Sept.~28 -- Compact Sets}

\section{Another Characterization of Closed Sets}
\begin{definition}
  Given a set $A \subseteq \R$, its \textbf{complement} is
  $A^c = \{x \in \R : x \notin A\}$.
\end{definition}

\begin{theorem}
  A set $A \subseteq \R$ is closed if and only if $A^c$
  is open.
\end{theorem}

\begin{proof}
  $(\Rightarrow)$ Suppose $A$ is closed. Take any $x \in A^c$.
  Since $A$ is closed and $x \notin A$, we know $x$ is not a
  limit point of $A$. So there is $\epsilon > 0$ such that
  $(V_\epsilon(x) \cap A) \setminus \{x\} = \emptyset$.
  Since $x \notin A$, this means that
  $V_\epsilon(x) \cap A = \emptyset$, which means that
  $V_\epsilon(x) \subseteq A^c$. So $x$ is an interior point
  of $A^c$, and thus $A^c$ is open.

  $(\Leftarrow)$ Suppose $A^c$ is open. Let $x$ be a
  limit point of $A$. Assume that $x \notin A$, i.e.~$x \in A^c$.
  Since $A^c$ is open, there is $\epsilon > 0$ such that
  $V_\epsilon(x) \subseteq A^c$. But then
  $V_\epsilon(x) \cap A = \emptyset$, which is a contradiction
  with $x$ being a limit point of $A$. So $x \in A$, and
  thus $A$ is closed.
\end{proof}

\begin{corollary}
  A set $A \subseteq \R$ is open if and only if $A^c$
  is closed.
\end{corollary}

\begin{remark}
  This is used in topology, where a collection of open sets
  that satisfies certain conditions
  \footnote{The collection must be closed under finite intersection and arbitrary union.}
  is called a \textit{topology}
  of a space, and closed sets are defined as their complements.
  Furthermore, metrics are not necessary in this setting.
\end{remark}

\begin{theorem}\leavevmode
  \begin{enumerate}
    \item Let $A_1, A_2, \dots, A_n \subseteq \R$ be closed.
      Then $\bigcup_{i = 1}^n A_i$ is closed.
    \item Let $A_\lambda \subseteq \R$, $\lambda \in \Lambda$
      be a family of closed subsets of $\R$ indexed by
      $\lambda \in \Lambda$, where $\Lambda$ is an index
      set. Then $\bigcup_{\lambda \in \Lambda} A_\lambda$
      is closed.
  \end{enumerate}
\end{theorem}

\begin{proof}
  Left as an exercise.
\end{proof}

\section{Compactness}

\begin{definition}
  A set $A \subseteq \R$ is \textbf{compact} if any sequence
  $\{a_n\}$ in $A$ has a convergent subsequence
  $\{a_{n_k}\}$ such that $\lim_{k \to \infty} a_{n_k} \in A$.
  \footnote{This definition is sometimes called \textit{sequential compactness}.}
\end{definition}

\begin{remark}
  Suppose we want to solve the differential equation
  \[
  \begin{cases}
    x' = f(t, x) \\
    x(t_0) = x_0
  \end{cases}
  .\]
  We can first transform this into the integral equation
  \[
    x(t) = x_0 + \int_{t_0}^t f(s, x(s))\, ds
  \]
  with a test function $f$.
  Then we perform Picard iterations to continue. However, this
  method requires $f \in C^{1}(\R)$ (i.e.~$f$ is continuously
  differentiable), since that is what
  guarantees that the sequence of functions converges
  (closedness).
  But even without this condition (if $f$ is only continuous),
  if we can
  show that the set of functions lies in a compact set,
  then we can find a subsequence of functions that do converge
  (though solutions may no longer be unique).
\end{remark}

\begin{example}
  The interval $(0, 1]$ is not compact since it is not closed:
  it does not contain all of its limit points.
\end{example}

\begin{example}
  The set $\R$ is not compact since it is not bounded:
  an arbitrary sequence may not even have a convergent
  subsequence.
\end{example}

\begin{theorem}
  \label{thm:compactness}
  A set $A \subseteq \R$ is compact if and only if
  $A$ is bounded and closed.
  \footnote{This applies to all Euclidean spaces (and pretty much only Euclidean spaces).}
\end{theorem}

\begin{proof}
  $(\Rightarrow)$\, Assume $A$ is not bounded. Then for any
  $n \in \N$, there exists an $x_n \in A$ such that
  $|x_n| > M$. Then $\{x_n\}$ is a sequence in $A$.
  Since $A$ is compact, there exists a subsequence
  \[\{x_{n_k}\} = \{x_{n_1}, x_{n_2}, x_{n_3}, \dots\}\]
  such
  that $x = \lim_{k \to \infty} x_{n_k} \in A$. But this
  implies that $\{x_{n_k}\}$ is bounded, which contradicts
  the fact that $|x_{n_k}| > n_k$ and $n_k \to \infty$
  and $k \to \infty$. Hence $A$ must be bounded.

  Now assume $A$ is not closed. Then there exists a limit
  point $x$ of $A$ such that $x \notin A$. Since $x$ is
  a limit point of $A$, there exists a sequence
  $\{a_n\}$ such that $a_n \in A$ and
  $\lim_{n \to \infty} a_n = x$. Now since $A$ is compact,
  there exists a convergent subsequence $\{a_{n_k}\}$ such that
  $\lim_{k \to \infty} a_{n_k} \in A$. But since
  $\{a_n\}$ converges, we have
  \[x = \lim_{n \to \infty} a_n = \lim_{k \to \infty} a_{n_k} \in A.\]
  This is a contradiction with $x \notin A$. Hence $A$ is closed.

  $(\Leftarrow)$\, Suppose $A$ is bounded and closed. Let
  $\{a_n\}$ be a sequence in $A$. By the Bolzano-Weierstrass
  theorem, there exists a convergent subsequence $\{a_{n_k}\}$.
  Let $x = \lim_{k \to \infty} a_{n_k}$. But $\{a_n\}$ is
  a convergent sequence in $A$, which is closed. So $x \in A$,
  and hence $A$ is compact.
\end{proof}

\begin{example}
  The union of intervals $[1, 2] \cup [3, 4]$ is compact.
\end{example}

\begin{example}
  The set
  \[A = \left\{\frac{1}{n} : n \in \N\right\} \cup \{0\}\]
  is compact (since we added the limit point $0$).
\end{example}

\begin{remark}
  Why do we need the concept of compactness? Because Theorem
  \ref{thm:compactness} is no longer true in infinite dimensions.
\end{remark}

\begin{theorem}
  Suppose
  \[k_1 \supseteq k_2 \supseteq k_3 \supseteq \dots\]
  are non-empty compact sets. Then
  \[
    \bigcap_{n = 1}^\infty k_n \ne \emptyset
  .\]
\end{theorem}

\begin{proof}
  Since $k_n \ne \emptyset$ for all $n$, there exists
  $a_n \in k_n$. Then for any $m \in \N$,
  $\{a_n\}_{n = m}^\infty$ is a sequence
  in $k_m$, which is compact. For $m = 1$,
  there exists a convergent
  subsequence $\{a_{n_k}\}$ such that
  $x = \lim_{k \to \infty} a_{n_k} \in k_1$. For any
  $m \in \N$, there is $k \in \N$ such that for all
  $k \ge k_m$, $n_k \ge m$. So
  \[a_{n_k} \in k_{n_k} \subseteq k_m,\]
  which means that $\{a_{n_k}\}$ is a convergent sequence
  in $k_m$, which is closed. Thus $x \in k_m$ for all
  $m$, which means that $x \in \bigcap_{n = 1}^\infty k_n$.
  So $\bigcap_{n = 1}^\infty k_n$ is nonempty.
\end{proof}

\begin{example}
  For $U_n = (0, 1 / n)$. Then
  $U_1 \supseteq U_2 \supseteq U_3 \supseteq \dots$, but
  their intersection is empty.
\end{example}

However, this is not so surprising, since the measure of
the sets $U_n$ tends to 0.

\begin{example}
  The sequence $V_n = (n, \infty)$ also satisfies
  $V_1 \supseteq V_2 \supseteq V_3 \supseteq \dots$,
  but their intersection is empty, despite each $V_n$ having
  infinite measure.
\end{example}

\section{Another Definition of Compactness}

\begin{theorem}
  A set $A \subseteq \R$ is compact if and only if
  it satisfies the following property:
  \begin{quote}
    (Covering property)\,
    For any family $U_\lambda$, $\lambda \in \Lambda$ of
    open subsets of $\R$ such that
    $A \subseteq \bigcup_{\lambda \in \Lambda} U_\lambda$,
    \footnote{This is called an \textit{open cover} of $A$.}
    there exists $n \in \N$ and
    $\lambda_1, \dots, \lambda_n \in \Lambda$ such that
    $A \subseteq \bigcup_{k = 1}^n U_{\lambda_k}$.
    \footnote{I.e.~there exists a finite subcover.}
  \end{quote}
\end{theorem}

\begin{proof}
  $(\Leftarrow)$\, Assume that the covering property holds for
  $A$. For boundedness, let
  \[A \subseteq \bigcup_{n = 1}^\infty (-n, n) = \R.\]
  By the covering property, there exist
  $n_1 < n_2 < \dots < n_k$ such that
  \[A \subseteq \bigcup_{i = 1}^k (-n_i, n_i) = (-n_k, n_k).\]
  So $A$ is bounded. Now for closedness, suppose otherwise
  that $A$ is not closed. So there exists a limit point
  $x \notin A$. Then
  \[A \subseteq \R \setminus \{x\} = \bigcup_{n = 1}^\infty \left\{y : |y - x| > \frac{1}{n}\right\}.\]
  By the covering property, there exist
  $n_1 < n_2 < \dots < n_k$ such that
  \[A \subseteq \bigcup_{i = 1}^k \left\{y : |y - x| > \frac{1}{n_i}\right\} = \left\{y : |y - x| > \frac{1}{n}\right\}.\]
  But $x$ is a limit point of $A$, so there exists a
  $z \in V_{1 / n_k}(x) \cap A$. This is a contradiction.
  Hence $A$ is also closed, and thus $A$ is compact.

  $(\Rightarrow)$\, Assume $A$ is compact and let
  $U_\lambda$, $\lambda \in \Lambda$ be an open cover
  of $A$. Suppose otherwise that there does not
  exist a finite subcover.
  Since $A$ is compact, $A$ is bounded. So there exists
  an $M$ such that $A \subseteq [-M, M]$. Let $A_1 = A$
  and define a sequence $A_n$ of sets inductively by
  $A_n = A \cap [a_n, b_n] \ne \emptyset$ with
  \[
    |b_n - a_n| = \frac{2M}{2^{n - 1}}
  ,\]
  and $A_n$ can't be covered by a finite subcollection
  of $U_\lambda$. Suppose such $A_n$ is defined.
  Then
  \[A_n = \underbrace{\left(A \cap \left[a_n, \frac{b_n + a_n}{2}\right]\right)}_{= k_1} \cup \underbrace{\left(A \cap \left[\frac{a_n + b_n}{2}, b_n\right]\right)}_{= k_2}.\]
  So either $k_1$ or $k_2$ can't be covered by a finite
  subcollection of $U_\lambda$. Let this one be $A_{n + 1}$.
  Since $A_n$ is closed and bounded, $A_n$ is compact.
  Then we have a sequence of compact sets
  \[A_1 \supseteq A_2 \supseteq A_3 \supseteq \dots,\]
  so $\bigcap_{n = 1}^\infty A_n \ne \emptyset$.
  Since $|a_n - b_n| \to 0$ (prove this as exercise),
  there exists $x \in \R$ such that
  $\cap_{n = 1}^\infty A_n = \{x\}$. But there exists
  $\lambda_0$ such that $x \in U_{\lambda_0}$, which is open.
  So there exists $n$ such that
  $V_{1 / n}(x) \subseteq U_{\lambda_0}$, which means that
  $A_{n + 1} \subseteq U_{\lambda_0}$. This is a
  contradiction.
\end{proof}

\begin{example}
  Let $A = (0, 1)$, $\Lambda = (0, 1)$, and
  $U_\lambda = (0, \lambda)$. We have
  $A \subseteq \bigcup_{\lambda \in (0, 1)} U_\lambda$,
  but there does not exist a finite subcover.
\end{example}

  \chapter{Oct.~3 -- Perfect Sets}

\begin{definition}
  A set $P \subseteq \R$ is \textbf{perfect} if it is closed
  and contains no isolated points.
\end{definition}

\begin{example}
  The closed intervals $[a, b]$ are perfect.
\end{example}

\section{The Cantor Set}
\begin{example}
  Define a sequence of sets inductively by $C_0 = [0, 1]$ and
  removing its middle third to get
  $C_1 = C_0 \setminus (\frac{1}{3}, \frac{2}{3})$, i.e.
  \[C_1 = \left[0, \frac{1}{3}\right] \cup \left[\frac{2}{3}, 1\right].\]
  Then we have
  \[C_2 = \left(\left[0, \frac{1}{9}\right] \cup \left[\frac{2}{9}, \frac{1}{3}\right]\right) \cup \left(\left[\frac{2}{3}, \frac{7}{9}\right] \cup \left[\frac{8}{9}, 1\right]\right),\]
  and so on, removing the middle third of each interval at
  each step. Note that $C_n$ is a set consisting of $2^n$
  closed intervals each of length $\frac{1}{3^n}$. Then
  the (middle third) \textit{Cantor set} $C$ is defined as
  \[C = \bigcap_{n = 0}^\infty C_n.\]
\end{example}

\begin{remark}
If we consider the sum of the lengths of the intervals that we
removed, we get
\[\frac{1}{3} + 2 \cdot \frac{1}{9} + 4 \cdot \frac{1}{27} + \dots + 2^{n - 1} \cdot \frac{1}{3^n} + \dots = \frac{1}{3}\left(\frac{1}{1 - \frac{2}{3}}\right) = \frac{1}{3} \cdot 3 = 1.\]
So, in some sense, the ``size''
\footnote{The (Lebesgue) \textit{measure}.}
of the Cantor set $C$ is 0.
However, $C$ is uncountable. In particular, the cardinality of
$C$ is the same as the cardinality of $\R$. A lot of
counterexamples in real analysis come from this Cantor set.
\end{remark}

\begin{remark}
  This means that the usual measure is not a good way to
  ``catch'' the Cantor set. Instead, we can consider
  \textit{fractional} (or \textit{fractal}) dimensions.
\end{remark}

\begin{theorem}
  The Cantor set $C$ is perfect.
\end{theorem}

\begin{proof}
  First note that $C$ is a countable intersection of
  closed sets, so $C$ is closed as well.
  To see that $C$ has no isolated points, take an arbitrary
  $x \in C$. Since $x \in C_n$ for all $n$, we can find
  $x_n \in C_n$ such that $x_n \ne x$ and
  $|x_n - x| < \frac{1}{3^n}$. Then $\lim_{n \to \infty} x_n = x$
  and $x_n \ne x$, so $x$ is a limit point of $C$.
  Thus $C$ is perfect.
\end{proof}

\section{Perfect Sets and Countability}

\begin{theorem}
  A nonempty perfect set is uncountable.
\end{theorem}

\begin{proof}
  Note that if $P$ is perfect, then $P$ is infinite
  (if we only have finitely many points, then they must be
  isolated). Now suppose that $P$ is only countably infinite.
  Then we can write
  \[P = \{x_1, x_2, \dots\}.\]
  Take $I_1$ to be a closed interval such that $x_1 \in I_1$
  and $x_1$ is not an endpoint of $I_1$.
  Since $x_1$ is not isolated in $P_1$, we can find $y_2 \in P$
  with $y_2 \ne x_1$
  such that $y_2 \in I_1$ and $y_2$ is not an endpoint of $I_1$.
  Then let $I_2 \subseteq I_1$ be a closed interval centered at
  $y_2$ such that $x_1 \notin I_2$. For example, we can do this
  by setting
  \[
    \epsilon = \frac{1}{2}\min\{y_2 - a, b - y_2, |x_1 - y_2|\}
  \]
  and letting $I_2 = [y_2 - \epsilon, y_2 + \epsilon]$.
  Since $y_2 \in P$, $y_2$ is not isolated, so we can find
  $y_3 \in P$ such that $y_3$ is not an endpoint of $I_2$ and
  $y_3 \ne x_2$. Pick $I_3$ centered at $y_3$ such that
  $x_2 \notin I_3$ and $I_3 \subseteq I_2$. Note that
  $I_3 \cap P \ne \emptyset$ since $y_3 \in I_3 \cap P$.
  From here, we continue by constructing
  $I_{n + 1} \subseteq I_n$ with $x_n \notin I_{n + 1}$ and
  $I_{n + 1} \cap P \ne \emptyset$. Let
  $k_n = I_n \cap P$. Clearly $k_n$ is closed, and $k_n$
  is also bounded since $k_n \subseteq I_n$. So $k_n$ is
  compact. By construction, $k_{n + 1} \subseteq k_n$, so
  by the nested interval property of compact sets,
  $\bigcap_{n = 1}^\infty k_n \ne \emptyset$. But
  $k_n \subseteq I_n$ and $x_n \notin I_{n + 1}$
  so $x_n \ne k_{n + 1}$. Since $k_n \subseteq P$, we must
  have $\bigcap_{n = 1}^\infty = \emptyset$. This is
  a contradiction, so $P$ must be uncountable.
\end{proof}

\end{document}
